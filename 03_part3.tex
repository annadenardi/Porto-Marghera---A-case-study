\section{The existing problems of Porto Marghera}
As seen up to now, Porto Marghera suffers from a wide range of problems, creating a complex system that cannot be analyzed in a simplistic way. In this section a summary of the existing problems is made.

\begin{itemize}[leftmargin=*, noitemsep]
\item \textbf{Soil and groundwater pollution at an extreme level}: the construction of Porto Marghera and the activities that operated on it since the XX century have caused pollution of soil, groundwater and sediments of the lagoon, creating a negative one-of-a-kind situation.
\item \textbf{Slow bureaucracy}: the works for pollution decontamination have been stalled for years and very few hectars of land actually have a completely carried out decontamination. The multiplicity of entities that exist on the same land contributed in slowing the process more.
\item \textbf{The outcome of the remediation is unsure}: while being an area full of potential, located in a place that is economically important, the polluted condition of the area does not attract private investors that are discouraged in starting the remediation process due to its complexity and unsureness of success.
\item \textbf{Partially an active zone}: the area is still partially active since it hosts port, logistics companies, chemical companies and oil companies, but this makes the decontamination process more difficult because public entities have to discuss also with private entities.
\end{itemize}

It is clear that Porto Marghera’s case is unique, therefore it is very difficult to find examples in other locations where problems were worked out in a perfect way. However, some examples that have partially similar characteristics to Porto Marghera can be found and some inspiration can be taken from them.

\section{National and international virtuous examples}
\subsection{Emscher Park}
The case of the Emscher Park in Germany is one of the most complex yet successful cases of brownfield renovation, considering that it regards the vast and largely polluted area of the Ruhr district. The strength of the program was that the various objectives established were divided in the span of the ten years of activity of the project (1989-1999), and that it was carried out through single-focused projects that stayed between the previously defined guidelines.

Including main focuses such as the restoration of the hydrographic system, new forms of housing and cultural activities, the conservation of cultural heritage and the reconstruction of the landscape, the Emscher Park case gave a new life to the rust belt, and became and example to follow all across Europe.

\subsubsection{The Planungsgesellschaft IBA Emscher Park GmbH (Internationale Bauausstellung (IBA) or International Architecture Exhibition, German device for urban engineering and architecture)} 
One of the key points of the Emscher Park operation was the definition of an authority with no legal power but with the aim to coordinate the action of all the parties involved, both public and private – including 17 different Municipalities, and supporting partners such as Emschergenossenschaft and Lippeverband, Kommunalverband Ruhrgebiet, Deutsche Bahn, RAG AG and many others \cite{rhur_iuav}.
The society, officially inactive since 1999, was organised in a board of directors, with members such as political, economic, social and environmental authorities, and a steering committee headed by the minister of urban planning and transportation, and made up of the representatives of the region, of the main municipalities, of TU Dortmund, of the professional orders and of the freelancers of the involved professions. Its employees were less than thirty, and its headquarters were located in Gelsenkirchen.
Its main characteristic was the total lack of legal power, that made it impossible for it to fine authorities or to force them to take part into initiatives. Its aim was to be a meeting platform for dialogue and exchange of ideas between the different parties involved, that could each come up with their own projects.

\textbf{Application for Porto Marghera}: One of the main problems that Porto Marghera has to face is the presence on its territory of many different legal actors, such as the Municipality of Venice, the Port Authority of the Northern Adriatic Sea, the Veneto Region and more. This plethora of actors, along with the extended length of the bureaucratic procedure, has proved to be quite disadvantageous to the making of actual plans for the remediation of Porto Marghera. Therefore, the Emscher Park case could be a valid example to follow, with the ad hoc creation of a non-legal authority aimed at the coordination of the various ideas and projects for Porto Marghera.

The presence of said authority would ensure that all of the ideas from the different parties are shared and heard, and at the same time it would maintain a cooperative and non-threatening environment, with the impossibility of giving sanctions or forcing the actors to take part into a specific project. With shared ideas and a more punctual organization of them, the future of Porto Marghera could be made different than its present.

\subsubsection{Guidelines and single-focused projects} 
The project for Emscher Park followed seven main guidelines developed by the IBA, identified as:
\begin{itemize}[leftmargin=*, noitemsep]
\item Reconstruction of landscape – the Emscher Landscape Park
\item Ecological restoration of the river Emscher system
\item Rhein-Herne Canal – an adventure space
\item Industrial cultural heritage as national treasure
\item Working in the park
\item New forms of houses and housing
\item New options for social, cultural and sports activities
\end{itemize}
These general principles were the base that had to be integrated into the single-focused projects that the actors had to come up with, each registered in a short, medium or long term period of time.
This allowed the plans to be followed more strictly, and to have a clearer vision of what the use destination of each part of the park should have been. 

\textbf{Application for Porto Marghera}: Porto Marghera lacks a clear vision of what it wants to be in the future. It is apparent that remediation works are needed, but an overall masterplan can be too large of a project for an area that needs such a conspicuous amount of work.

Main guidelines to follow and single-focused projects could be an efficient solution to this problem, as they would give a more defined and detailed vision of what each part of the area would and could look like. The remediation works wouldn’t just exist in a vacuum but would have a clear objective as to why it is done, apart from the obvious environmental constraint. The short, medium and long term division for the single-focused project would allow the overall plans to proceed much faster and the first results to be seen in a relatively closer window of time, to make room for future adjustments. The definition of the guidelines would be carried out by the aforementioned ad hoc authority, while the single-focused projects would be proposed by single actors.

\subsubsection{Population participation} 
What distinguishes the Emscher Park project, amongst other things, is the fact that the population of the area had such a central part in the processes of creation of what would become the park itself. In fact, the wishes and desires of the inhabitants on what they wanted the area to look like, what they wanted to celebrate and preserve, which functions were needed in their opinion, how the landscape had to transform to fulfill its full potential in their minds. Numerous projects were proposed and the population’s contribution to each varied from one to another.
It is fundamental to understand that without the population’s willingness to participate and without the plan to include it, the results wouldn’t have been as successful.

\textbf{Application for Porto Marghera}: Porto Marghera is a heavily polluted area that because of this issue and of its industrial past, doesn’t leave much space for the population and its needs, sometimes even basic ones. It is noted, for example, that a problem in previous years was the one of the lack of public housing, as well as the lack of jobs. Emscher Park serves as an example for how the ideas and the desires of the population helped shape its final form.

The same could be applied to Porto Marghera, for example letting the population decide, within the list of the abandoned industrial buildings, which ones to restore to keep as a part of the industrial heritage of the area for a process of valorization, and which ones to demolish or restore to have a new function, according to the ones needed the most by the population itself.
The inhabitants could also be proposing their own ideas for projects to realize on the areas and buildings subject to remediation, and such ideas could be explored in dedicated workshops.

\begin{minipage}{0.55\textwidth}
\centering
\includegraphics[width=\textwidth]{images/03_emscher.png}
\captionof{figure}{Emscher park, by Peron I.}
\end{minipage}
\hspace{20pt}
\begin{minipage}{0.39\textwidth}
\centering
\includegraphics[width=\textwidth]{images/03_genova.png}
\captionof{figure}{Porto di Genova, by Peron I.}
\end{minipage}

\subsection{Genova's harbor}
Genova is a perfect example of redevelopment of a dismissed area and reconnection of the city with the sea. Historically, Genova was always characterized by a dualism: the port and the city. The geography of the area was a great constraint: the mountains on the north, the valleys of the sides and the sea at south obliged the city to tight itself inside the available space. Inside this narrow area, the municipality of Genova and the Port Authority coexisted and regulated strictly separate zones. 
\subsubsection{Previous situation}
The area was terribly divided into two parts and the city didn’t have access to the sea. Starting in the 1960s, Genoa's “old port” experienced a gradual decline in activity due to the construction of new ports in the West and the relocation of main activities to the areas of Voltri (bulk cargo) and Multedo (crude mineral oils). With the transformation of maritime trade and the spread of container use, the old port increasingly lost its role in the economic balance of the city, resulting in the abandonment of several buildings. In addition to this, the construction of the Sopraelevata street in the late ‘60 strengthened the sense of separation between the two parts of Genova \cite{storiaportoantico}. 
\subsubsection{What was done}
The difficult situation witnessed a change when Renzo Piano designed the “Porto Antico”: this was the first time for Genova that an area was taken away from the port jurisdiction and given back to the city to be destined to urban settlements. In 1995 the Porto Antico S.p.A was created ad hoc to manage the whole area. In its project, Renzo Piano also planned the redevelopment of dismissed buildings present there: the idea was reconverting them into areas that were missing inside the historical city centre (such as areas for cultural events or commercial areas). It is estimated that Porto Antico created 1000 jobs and is one of the most visited areas of Genova, especially because of its famous aquarium \cite{mic_portoantico}. 
\subsubsection{Current situation}
After more than 20 years, Porto Antico still attracts visitors from different parts of Italy. A great number of cultural events are organized yearly in the area, giving also to citizens job and leisure opportunities. 

\subsection{Rotterdam's harbor}
Rotterdam is a strong example of how a large industrial port can be transformed while remaining economically active. Historically, the port developed mainly for industrial and logistical efficiency, with little attention to its relationship with the city or the environment. Over time, port activities moved westward toward the North Sea, creating a physical and functional separation between Rotterdam and its harbor.
\subsubsection{Previous situation}
During the 20th century, Rotterdam’s port was dominated by heavy industries such as oil refining, petrochemicals, and bulk transport. The port expanded over a very large area, reaching more than 40 km from the city centre \cite{rotterdam}. While this growth was economically successful, it caused serious environmental damage and left several inner-port areas abandoned. The port also increasingly conflicted with protected natural areas, later included in the Natura 2000 network.
\subsubsection{What was done}
Rotterdam chose an integrated approach combining port development, environmental protection, and urban renewal. The construction of Maasvlakte 2 (an expansion of the port), considered a project of national importance, was accompanied by strong ecological compensation measures. Coastal dunes were recreated, marine habitats were restored, fishing was limited in sensitive areas, and compensation zones were made much larger than the areas that were damaged \cite{rotterdam2}.

In parallel, the city redeveloped former port areas close to the centre, such as Kop van Zuid and Stadshavens. Old docks and industrial zones were converted into mixed-use areas with housing, offices, cultural spaces, and innovation hubs, reconnecting the city with the waterfront \cite{rotterdam3}.
\subsubsection{Current situation}
Today, Rotterdam remains one of Europe’s most important ports while also being a leader in sustainability and innovation. Requalified waterfront areas attract residents, visitors, and new economic activities. Environmental restoration and monitoring are still ongoing, showing a long-term commitment to ecological balance.
\subsubsection{Similarities with Porto Marghera}
Like Rotterdam, Porto Marghera developed as a large industrial and petrochemical area, creating strong separation between the port and the city and causing environmental degradation. Both areas include abandoned industrial zones and sensitive ecosystems. The Rotterdam experience indicates that ecological compensation in lagoon environments can be technically achievable in areas surrounding port infrastructure, without necessarily interrupting port activities, offering relevant insights for the redevelopment of Porto Marghera.

\begin{figure}[H]
\centering
\includegraphics[width=\linewidth]{images/03_rotterdam.png}
\caption{Map of the evolution of Rotterdam in history \cite{rotterdam}}
\end{figure}

\section{Project proposals}
For a territory characterized by such complexity, a general masterplan isn’t the right choice. Instead, the redevelopment of a number of areas inside Porto Marghera could be the starting point for a new use and look of it, following the examples of the projects that were carried out in the last decade (paragraph 2.5).
In addition, considering the high level of pollution both for soil and for the aquifer, the outcome of a possible remediation is unknown at the start. For this reason, the land use creation process cannot be disassociated to the remediation process. In other words, the land use must be decided according to the outcome of the remediation process, that is also the achievable level of remaining pollution. In the following paragraphs, the project proposals of this work are presented. A first part delves into the remediation techniques and explains the vision of the landuse evolution in time according to remediation (taking inspiration from the proposal of PhD Irene Peron \cite{peron}). Instead, a second part shows the project proposals about landuse and building reconversion for Porto Marghera. 

\subsection{About remediation}

\subsubsection{The techniques}
There is a wide range of techniques that can be taken to assess the pollution problem. In the presented scheme, the different solutions are linked with its effectiveness and consequent use limitations.

\underline{MISE (messa in sicurezza emergenziale)} [Emergency safety measures]: 
Any immediate or short-term intervention, to be implemented in the event of sudden contamination of any kind, aimed at containing the spread of primary sources of contamination, preventing their contact with other matrices present on-site, and removing them, pending any further remediation interventions or operational or permanent safety measures.

\underline{MISO (messa in sicurezza operativa)} [Operational Safety Measures]
The set of interventions carried out on an active site intended to guarantee an adequate level of safety for people and the environment, pending further permanent safety measures or remediation to be implemented upon the cessation of activity. These also include contamination containment interventions to be implemented on a transitional basis until the execution of remediation or permanent safety measures, in order to prevent the spread of contamination within the same matrix or between different matrices. In such cases, suitable monitoring and control plans must be prepared to verify the effectiveness of the solutions adopted.

\underline{MISP (messa in sicurezza permanente)} [Permanent Safety Measures]:
The set of interventions aimed at permanently isolating pollution sources from the surrounding environmental matrices and guaranteeing a high and definitive level of safety for people and the environment. In such cases, monitoring and control plans must be provided, as well as land-use restrictions in accordance with urban planning regulations

\underline{Remediation}: The set of interventions aimed at eliminating pollution sources and polluting substances, or reducing their concentrations present in the soil, subsoil, and groundwater to a level equal to or lower than the Risk-based Threshold Concentrations (CSR). Remediation techniques are divided into two main categories: \textit{in-situ} techniques (pollution is treated in its location) and \textit{ex-situ} techniques (the polluted matrix is removed and treated separately). 

In accordance with ARPA Veneto chosen remediation techniques, this work wants to highlight the need of in-situ remediation. In fact, ex situ remediation proves to be more costly than in-situ actions: there is a need to find a site (landfill) where the waste can be treated, resulting in a greater environmental and health impact stemming from the risks of transport and disposal. Landfilling may be an option for small areas characterized by a specific and localized source of pollution, but it proves impracticable in situations as vast and complex as that of Porto Marghera. 

\begin{figure}[H]
\centering
\includegraphics[width=\linewidth]{images/03_remediation.png}
\caption{Visualization of the different techniques to assess the protection of the environment. Image extracted from Peron I. (2016). \cite{peron}}
\end{figure}

\subsubsection{The vision}
This work does not aim at projecting new ways to carry out the remediation process, but it tries to propose an alternative vision on the \textit{iter} of decontamination of the soil. Since the Masterplan on remediation, the slowness of the procedures deriving from the complexity of the matter have left a large quantity of terrains completely unusable by the citizens or new companies: obviously, this was due to the contaminated condition of the land, but there might be a different way to give public access to the areas, even if limited, and this is what the vision here presented consists of. 

The basic concept here proposed is that the intended land use of a contaminated soil needs to be temporary and legally capable of change according to the progress of the remediation. This approach stems from the fact that when a remediation process starts, the outcome (i.e. the final condition after the operation) is unknown: this implies that a terrain might not have the qualities required for the intended land use, even though the planned remediation was completely carried out. Instead, if a piece of land doesn't have a fixed landuse but a changing-with-time landuse, its final destination can be assessed later accordingly. 

Thus, the project vision is tightly interwoven with the progress and monitoring of the site's remediation process; a time-schedule example with a monitoring window of 5 years articulated across four temporal scenarios is here presented. In the example the chosen remediation technique is phytoremediation. It is given for granted that the necessary MISE works have already been implemented (see paragraph 1.6.2 about soil containment).
\begin{enumerate}[leftmargin=*, noitemsep]
\item Initial scenario(t=$t_0$): at this point the area faces its maximum level of contamination, the use of the site is highly restricted and the remediation process is at its starting point. The whole area is decided to be treated with phytoremediation. The existing dismissed buildings are still in abandoned condition.
\item First scenario (t=5 years): another sampling is carried out and the level of contamination has decreased, thus the access to the area is consented but limited (e.g. the area is equipped to become a park but with lifted walkable paths - even if placed outside the urban area it can function as a recreational area for workers in their breaks).
\item Second scenario (t=10 years): a further sampling is carried out and the level of contamination has decreased, up to a level at which it is permitted to give the area to logistic use (e.g. the dismissed building can start to be restored to the necessary use)
\item Third scenario (t=15 years): a further sampling is carried out and the level has decreased enough to let the area be used for commercial use (e.g. the dismissed building can be restored to welcome offices)
\end{enumerate}

It is important to state clearly that the residential use of the remediated land is not considered in the majority of situations, due to the fact that the pollution level is so high that any kind of remediation could never completely restore the complete salubrity of soil and groundwater \cite{grilliVenezia}. In addition to this, the land located in the artificial macro-islands is not adequate to residential use because of infrastructure and location (it is too far from the urban facilities and the connection to the mainland happens only through a small number of bridges).

\subsection{About landuse}
The idea for a project in Porto Marghera, a hub full of potential but at the same time extremely difficult to coordinate in terms of actors and environmental interventions, would be focused on green hydrogen, environmental regeneration and training of workers and students.
The main objectives of the project would be to build and further expand a centre for the distribution and use of green hydrogen for transport, industry and port, and to make the port operations more sustainable with the aid of electric infrastructure and to add facilities. The brownfields would be used for the cultural and historical valorization of the area, and according to the needs of the inhabitants of Porto Marghera. The whole project should be enriched by numerous workshops and training activities for students and workers, to make them more aware of the processes of environmental regeneration, with a hands-on approach.

This idea is mainly applied to the closest areas to the urban fabric that are currently in state of abandonment. This choice is justified by the fact that there is an unmotivated separation between industrial area and city are along Via Fratelli Bandiera, therefore using the abandoned spaces on the eastern side of the area would constitute a first reconnection between the two parts, following the virtuous example of Genova's harbor. An outline of the most interesting dismissed areas to consider is presented in figure \ref{fig:dismessi}. A more detailed presentation of this is given in the first part of this section.

In the second part, the project identifies the areas on the macro-islands that are currently empty or abandoned in order to choose them as the place where to apply the remediation vision presented in the previous section.

In the following paragraphs, a list of ideas is presented together with a map of the possible location of said project-idea. The plan involves only buildings shown in figure \ref{fig:dismessi}, instead for the use of the brownfields an additional set of abandoned areas in the macro-isles is considered.

\begin{figure}[h]
\centering
\includegraphics[width=\linewidth]{images/03_dismessi.jpeg}
\caption{Map of dismissed industrial buildings (highlighted in orange) near the city center of Marghera. The map was created by the authors using QGIS.}
\label{fig:dismessi}
\end{figure}


\subsubsection{The green hydrogen hub}
First of all, the project would be compatible with the already existing Hydrogen Valley Venezia in Porto Marghera, under the funding of the PNRR (see paragraph 2.5.4).
The idea would be to continue with the building of green hydrogen plants based on photovoltaic panels and other renewable energy sources considered most suitable for the space. This would allow for more production, which would make the green hydrogen available not only for transport and industry, which are the main objectives of the Hydrogen Valley Venezia, but also for heavy goods vehicles and naval vehicles.
This would be a perfect opportunity for workers and Marghera inhabitants to learn more about green hydrogen in workshops, and for university students to write dissertations and to fulfill internship duties. 

A possibile placement of the hydrogen hubs is presented in figure \ref{fig:hydrohubs}, considering dismissed buildings that could be restored and are the closest to the green hydrogen plant Sapio, in the southern part of Marghera.

\begin{figure}[h]
\centering
\includegraphics[width=\linewidth]{images/03_hydrohubs.jpeg}
\caption{Map of a possible location of the hydrogen hubs considering two of the already highlighted dismissed buildings near the city center. The two buildings were chosen because of their vicinity to the Sapio green hydrogen plant (shown in the map). The map was created by the authors using QGIS.}
\label{fig:hydrohubs}
\end{figure}

\subsubsection{The docks and the monitoring}
The subsequent step would be linked to the ongoing project that aims at the electrification of the port dock driven by the Port Authority, in order to reduce the emissions of the ships that are anchored in the port.
To this, there could be the addition of systems of noise and air quality, that would be useful for the assessment of the results of the operations that have been carried out. The monitoring could be, again, done with the collaboration of students that could rejoice in doing internships. Furthermore, an information point could be considered, to spread the awareness among inhabitants of the neighborhood.

\subsubsection{The brownfields on the mainland}
Regarding the brownfields, there is a distinction to be defined between the areas located in the mainland and the areas located on the macro-islands. It goes without saying that a different purpose in land use has to be considered for the two zones given their different vicinity to the urban fabric, the infrastructure facilities and the dissimilar risk of pollution (both atmospheric and acoustic). Thus, this project considers that areas inside the macro-islands can only be destined to become green areas, commercial areas or in some cases cultural spaces, instead, for the brownfields on the mainland other destination of use can be considered.

\begin{figure}[H]
\centering
\includegraphics[width=\linewidth]{images/03_projectlanduse.jpeg}
\caption{Map of a possible destination of use of the remaining abandoned areas. The map was created by the authors using QGIS.}
\label{fig:newlanduse}
\end{figure}

Porto Marghera is, as already established, an area rich in cultural value for the impact its industrial nature has had since it was built, but this aspect lacks valorization. This is why the first proposal of use in the mainland would be for a Marghera Museum, a restored industrial building that would serve as a time capsule for how production worked through the years and what legacy it left. The museum would be articulated in different parts, each one related to the different type of industry that is being represented: there would be a part of historical retelling and historical documentation through pictures and artifacts, and a part where some of the most important mechanisms for production are built again in order for people to further inspect them and use them. Each section would also be filled with the words of the people that worked in the different industries. A part would then be dedicated to the historical accounts of the worker protests during the years, and the final section would be dedicated to Marghera as a part of the Venetian conglomerate but through the eyes of the people that live in it, with accounts from the past and from today. This last section would also contain a part dedicated to the ongoing projects in Porto Marghera, in order to keep the inhabitants accountable of its current development. The museum would be useful, again, for students of Ca'Foscari and IUAV Universities that could offer guided tours as part of their internships. In addition, the museum could be the perfect space where to hold different workshops to learn more about environmental regeneration and sustainability.

The considered building, located along Via Fratelli Bandiera (Figure \ref{fig:newlanduse}), is a place that has been in state of abandonment since 2000s and photos of Google Earth testify that it is a place that the citizens asked for its renovation (Figure \ref{fig:museum}).
\begin{figure}[H]
\centering
\includegraphics[width=\linewidth]{images/03_museum.jpg}
\caption{Photos of the building that could be renovated to become museum. Source: Google Earth Pro.}
\label{fig:museum}
\end{figure}
 
Another proposal of use of one of the dismissed buildings of the area is a new university hub, either for Univeristà Ca'Foscari, IUAV or H-FARM business school. The dismissed building located close the railway in the northen part of Marghera presents a great number of infrastructural advantages: vicinity to train station Venezia Mestre, close to the city center and near an empty green area that could become the park of the university hub itself. The architecture of the building is very basic, but suitable to host university classrooms.
\begin{figure}[H]
\centering
\includegraphics[width=\linewidth]{images/03_university.jpeg}
\caption{Photos of the building that could be renovated to become a new university hub with its own park. Source: Google Earth Pro.}
\label{fig:uni}
\end{figure}

When it comes to the housing, the aim would be to solve a problem that characterizes Porto Marghera and that has been lamented multiple times by its inhabitants \cite{ripensarevenezia}: the lack of public housing and student housing. 

Public housing: Marghera and Mestre suffer from the shortage of public housing (ERP — Edilizia Residenziale Pubblica) since many years \cite{ripensarevenezia}. The existent ERP are old and already full and none of the recent governments have tried to solve this problem. Therefore, it seems necessary to include in this project the renovation of dismissed buildings that could function as public housing. The considered building for this is the one presented in figure \ref{fig:housing}.

Shared student-elderly housing: In recent years, projects that encourage distant generations to share a home have surfaced all over the world, including on university campuses in Canada, California and the Netherlands, as the BBC article reports \cite{itergen-house}. A similar idea could be applied also in Porto Marghera, considering its closeness to the Universities in Venice. The shared housing between students and elderly is thought as an experience that would benefit both categories and would guarantee them a better living situation. Students would benefit from it as they would have rooms to rent at a reduced price, sensible to their needs since the vast majority of them doesn’t work or works lesser paid jobs, and in a location that allows them to get to Venice rather swiftly with public transport. The elderly, on the other hand, frequently lack company and assistance. In this project, each person would have their own bedroom and bathroom facilities, and the other spaces would be shared, in order to guarantee companionships and little helps from the students to the elderly and vice versa – though not in terms of medical assistance. The common areas could also be filled with monthly organized cultural activities. The spaces would have their set of rules, such as ones for silence, privacy and volunteering. The space would also have to have, in close proximity, facilities for social and medical aid, which could ultimately create new spots in the job market.

The chosen building for this idea is the same one used for the public housing, presented in figure \ref{fig:housing}, located very close to the train station of Venezia Mestre and near the other planned hub for university. The building structure is suitable for the shared housing because it has internal gardens, small internal spaces and overall a large area. 
\begin{figure}[H]
\centering
\includegraphics[width=\linewidth]{images/03_housing.png}
\caption{Photos of the building that could be renovated to become a shared student-elderly housing. Source: Google Earth Pro.}
\label{fig:housing}
\end{figure}

Finally, a proposal of landuse for one of the considered buildings is to build a kindergarten. As a matter of fact, the \textit{Livelli Essenziali di Prestazione} (Essential Performance Levels) of Porto Marghera regarding the offer of places of education before elementary school are very low compared with other cities in Italy (39 places for 100 newborns in 2020) \cite{lep}. Thus, the wide green area located along Via Fratelli Bandiera could be the perfect spot for the construction of a new small kindergarten with a big green area all around it. 
\begin{figure}[H]
\centering
\includegraphics[width=\linewidth]{images/03_asilo.jpeg}
\caption{Location of the empty abandoned area that could be used to build a small kindergarten with a big garden.}
\label{fig:asilo}
\end{figure}

The last part of the proposal for the brownfileds in the mainland is dedicated to the conversion of buildings into spaces for co-working and startup hubs, the aim of which would be to give an actual space for sharing ideas and innovation in a space like Porto Marghera that so desperately needs it. This could be done starting from the VeGa, expanding its area. 

\begin{figure}[H]
\centering
\includegraphics[width=\linewidth]{images/03_macroisole.jpeg}
\caption{Map of the areas inside the macro islands considered in the project for a new destination. Map created by the authors using QGIS}
\label{fig:macroisole}
\end{figure}

\subsubsection{The brownfields on the macroislands}
Considering now the areas in the macro islands, the main project proposal is to identify the empty areas and the dismissed areas (land that hosts abandoned buildings) as in figure \ref{fig:macroisole}. The former are to be treated with phytoremediation, thus blocking its use for at least 5 years, as in the example presented in paragraph 3.3.1. Meanwhile, the latter are to be remediated as well, but will include in the future the renovation of the currently abandoned buildings. The use destination has to be defined later, in accordance with the vision of the project, choosing among logistics, cultural or commercial use.

\subsubsection{The funding}
When it comes to the funding of the project, many routes could be chosen.
As already established, the PNRR gives funds for dock electrification and renewable energy production such as green hydrogen (already implemented through the MASE funding notice). Furthermore, the Veneto Region has calls for bids for applications of environmental regeneration and requalification of polluted sites. Green investors and PMI can benefit from funding from partnered banks as well.
European programs could also be a part of the funding, such as the Interreg for urban and industrial regeneration, and the Horizon Europe for research on renewable energy and hydrogen as a source of it.
Last but not least, private investors would also be included, especially involving them in startups and sustainable logistics hubs.

\subsubsection{The coordination of stakeholders}
In order to achieve this, taking into account the large number of actors present on the territory and cited in the previous sections, the suggestion would be the one of creating an authority that would coordinate the process.
The Municipality of Venice, the Veneto Region and the Port Authority of the Norther Adriatic Sea would be the institutional stakeholders, while the projects and training would be the responsibility of universities and research centres.
Private investors and PMI would act on actuation and investments.

\begin{figure}[H]
\centering
\includegraphics[width=\linewidth]{images/03_peron.png}
\caption{One of the dismissed buildings in the northen part of Porto Marghera, a possible place of renovation while keeping memory of the history of the area. Extracted from Peron I.(2016), Potenzialità contese. Porto Marghera: una questione di metodo. \cite{peron}}
\end{figure}
