\section{Historical notes on Porto Marghera}
At the beginning of the XX century the port of Venice, which at the time was located in the insular area, had reached the maximum level of traffic, showing its structural limitations, such as the absence of rail and road connections to the mainland and the lack of land which turned into a severe physical constraint to any possible expansion. Therefore, the development of a commercial port and an adjacent industrial area on the mainland appeared to be a profitable opportunity. In addition to this, it was also considered the availability of a labour force at low cost, as at that time the area was mostly rural and economically depressed. 

In 1917 the Municipality of Venice, the “Società Porto Industriale” created by the entrepreneur Giuseppe Volpi and the Italian Boselli government signed the “Decreto Luogotenenziale 26 luglio 1917, n. 1191” in order to build the new port. In 1919 the Vittorio Emanuele III Canal, connecting the Giudecca to Marghera, was dug, while the Sade company started building the first thermal power plant. In 1920, the Società Porto Industriale started building infrastructures for the industrial and commercial port, the rail and road links needed for transporting the goods and raw materials for the works. The construction started in 1919 and in 1928 fifty-eight firms were already settled in the first industrial zone of Porto Marghera. Porto Marghera, which was conceived and designed as a coastal industrial area right from the start, experienced a time of notable growth. In 1925 there were 33 companies and 3,440 workers \cite{zazzara}. 

In 1925 a new Port Master Plan was drawn up providing for the extension of industrial areas up to the Brenta canal (Fusina), securing enough space required for the enlargement of the area throughout the sixties. Starting from the '30s, the metallurgy and non-ferrous material industry (aluminium and its alloys, zinc) grew vigorously and a large plant producing synthetic ammonia for nitrogen fertilizers was created. The area flourished in a very short time: not even the II World War stopped its growth. Even though some plants were hit during aerial bombing, they were soon rebuilt after the post-war adjustment. 
In the early '50s, because of the saturation of the first industrial area with 128 companies and 22,500 workers, the project for a second industrial area was designed, establishing productions of petrochemicals, refractory materials, precision carpentry, power plants and food oil refineries. This second industrial area rose mainly on areas taken away from the Lagoon through fills or embankments of the ground level, using industrial waste and scraps from factories based in the first industrial area and materials from the excavation of canals \cite{interreg}.

During the '60s, the development of Porto Marghera and the growth of industrial trades required a new stage of interventions. Following the increase in industrial trades, the San Leonardo oil port was built in 1966 and, three years later, the excavation of the Malamocco-Marghera canal was completed, in order to allow all oil tankers to reach the port of San Leonardo and the industrial areas through the Malamocco port entrance, avoiding the San Marco basin and the city's historical centre. At this point, in 1965, Porto Marghera was at its peak of employment, with 33.000 workers in the area. Porto Marghera was one of the major industrial concentrations of that time and thus had a great number of working class employees that took part in the great conflicts between capital and labour of the late 60s \cite{zazzara}. 

Since the '70s Porto Marghera fell into a crisis, due to the increase in raw material price and the decline of the raw chemical industry. This obliged many firms to downsize and to lay many workers off. The crisis continued all along the '80s and the '90s, reaching less than half of workers compared to the peak year: in 2000, only 12.727 people were employed in Porto Marghera \cite{interreg}. It was at this time during the 90s that the concern for the economically and environmentally critical condition started to arise and a lot of legislation and agreements between the parts were made. A brief recap of its evolution is here summarized: 

\textbf{1994-1996}: The environmental remediation and industrial reconversion project for Porto Marghera found its first definition in the Variation to the General Regulatory Plan (PRG) for Porto Marghera. This tool, later discussed in this work, not only was a new and precise urban planning tool for the area, but also an important set of indications for implementing planned interventions. 

\textbf{1998-1999}: The progressive emergence of the environmental issue (confirmed site pollution, risk of pollutant spills in the lagoon, and the presence of industrial plants posing risks to urban centers near Porto Marghera) occurred alongside a deepening structural crisis in the industrial sectors characterizing the hub (particularly petrochemicals). This led to a preventive search for a balance between environmental protection and the continuation and development of high-risk production processes. In this period Porto Marghera was placed among the Siti di Interesse Nazionale (SIN - Sites of National interest) of Italy. 
The Program Agreement for Chemistry in Porto Marghera (formally a DPCM of February 1999) proposed to establish and maintain optimal conditions for the coexistence of environmental protection and the productive development of the chemical sector. It conditioned the continued presence of industrial activities on the implementation of interventions that guaranteed their sustainability.

\textbf{2001}: A supplement to the Agreement (DPCM 15.11.2001) subsequently defined criteria for harmonizing the approval procedures for investment projects presented by the signatory companies of the first agreement. In particular, this supplementary act provided for the drafting of a "Masterplan for Remediation": the goal was to guide the creation of projects consistent with an environmental redevelopment program for the entire area, ensuring coherence, timeliness, and solutions tailored to the specific characteristics of the sites. This Master Plan for Porto Marghera received final approval in 2004.

\textbf{2004-2005}: The revitalization of a debate about the future of Porto Marghera, which occurred simultaneously with the debate over the environmental sustainability of petrochemical activity, has meant that the general theme of the Porto Marghera project has been reviewed comprehensively from the recognition of its economic value in Venice and the Veneto region. The "Intesa per Porto Marghera" signed in late 2005 between the Region, local authorities, business associations, and trade unions, provided for the development of activities aimed at defining a "development pact" for Porto Marghera based on fundamental aspects of production analysis and development capacity (launch of an environmental redevelopment process for the area, prevention of industrial risk, redevelopment of the chemical hub, construction of infrastructure necessary to decongest municipal and provincial arteries).

\textbf{2006-2007}: On December 14, 2006, the State, the Veneto Region, the Province and Municipality of Venice, social partners, and companies signed a new Program Agreement for Chemistry in Porto Marghera, which aimed to maintain conditions of managerial certainty for companies operating there, combining these with environmental protection needs starting from the integrated petrochemical cycle. This led to the "Protocol for sharing strategic guidelines for the redevelopment and development of Porto Marghera," released on September 21, 2007.

\textbf{2011-2015}: In May, 2011, a DPCM by the Ministry of the economic development included Marghera inside the areas of industrial crisis. This later led to the signing of a new Program Agreement for the environmental remediation and redevelopment of Porto Marghera between the ministry, the Port Authority and the local institutions, with the aim of simplifying the procedures of remediation. Always in 2012, the Progetto Integrato Fusina (PIF) was approved: the construction of a multifunctional platform to treat contaminated groundwater resulting from the safety measures implemented in the area. In 2014 the Veneto Region and the Municipality of Venice bought 110 hectares of dismissed industrial areas (previously owned by Syndial) for re-industrialization. This was followed by a new Program Agreement in 2015 that allocated 152 million euros for Porto Marghera. 

\textbf{2018-present}:
In January 2018, a new agreement was signed to set up a five-year contract between local institutions for the management of the remediation process in the Porto Marghera SIN. This was followed by an agreement signed on 23rd October 2018 by the Ministry of Economic Development, the Region, and the Municipality to provide for investment in reconversion measures for an amount of 27 million euros. This achievement was also be confirmed with the approval in September 2019 of a three-year extension to the contract (i.e., extension of the deadline until 2022) for completion of the redevelopment interventions provided for in the latest program agreement.

\begin{figure}[h]
\centering
\includegraphics[width=\linewidth]{images/02_timeline.png}
\caption{Timeline of the most important plans and laws for Porto Marghera. Source: Peron I., Potenzialità contese. Porto Marghera, una questione di metodo, scheda 28}
\end{figure}
% ----------------------- sezione overview -----------------------------------------------
\section{The legislative framework for Porto Marghera}
In this section, an overview of the existing legislation acting on Porto Marghera is given. This area is characterized by a complex system of entities that act upon it. Considering a hierarchical structure from the farthest actor to the nearest to the area, Porto Marghera sees at the same time: 
\begin{itemize}[leftmargin=*]
\item The European Directives
\item The Italian laws (that receive and apply the European Directives)
\item The Region with its regional plans and laws
\item The Port Authority, that is a separate entity from the Region (therefore on the same level as the region)
\item The Municipality of Venice and its local plans
\end{itemize}
This complex network results into the overlapping of the jurisdiction of the different entities, that to work together with a unified vision need to sign Program Agreements. However, there is quite a distance between the objectives of the different authorities and the effective action they can pursue unitarily. 

\begin{figure}[h]
\centering
\includegraphics[width=0.8\linewidth]{images/02_structure.png}
\caption{Visual representation of the hierarchy of the different entities that act upon Porto Marghera. Image created by the authors.}
\end{figure}

\subsection{On the European and national level}
\subsubsection{Funds and Programs}
The most important European funds and programs applied to Porto Marghera are the following. 

\textbf{ERDF: European Regional Development Fund}: The ERDF promotes regional economic growth and territorial cohesion. It finances infrastructure, innovation, environmental and energy projects. Its aim is to reduce regional disparities and support sustainable development. Actions focus on competitiveness, green transition, and digital transformation.

\textbf{PNRR: The National Recovery and Resilience Plan}: The PNRR is a strategic plan that aims to modernize and improve the Italian economy, making it more sustainable, innovative, and competitive. The Green Revolution and Ecological Transition component of Italy's PNRR has a total allocation of €55.52 billion (28.56\% of the total plan), instead restoration and protection of seabeds and marine habitats has €400 million \cite{pnrr}.

\textbf{Interreg Program}: Under the European funding programme \textit{Connecting Europe Facility}, the ports of Venice, Trieste, Koper e Ravenna, obtained a new co-funding for the implementation of the project called ACCESS2NAPA, for a total expenditure of over 14 million Euros. The North Adriatic Sea Port System Authority (NASPA) will contribute to the achievement of the growth objectives of the North Adriatic Ports Association (NAPA) by promoting two actions in particular, namely with the design of interventions in Porto Marghera for the strengthening of the last mile railway infrastructure (in the connection between via dell’Elettricità and the Petrochemical Peninsula) and road accessibility (the so-called Malcontenta node). NASPA will have a budget of € 880.000, 50\% co-financed and 36 months to complete the activities \cite{access2napa}.

\subsubsection{Directives}
\begin{enumerate}

\item \textbf{Directives 92/43/EEC (Habitats) and 2009/147/EC (Birds)} : It establishes the Natura 2000 network for biodiversity and habitat protection. All redevelopment projects must undergo an environmental assessment of impacts.\\
\textit{Porto Marghera is in the immediate vicinity of Natura 2000 sites, therefore they fully apply
Any redevelopment, remediation, port expansion or dredging in Porto Marghera must undergo a mandatory Natura 2000 Appropriate Assessment. Projects that risk significant impacts on protected species or habitats must be redesigned, mitigated, or compensated. If such impacts cannot be avoided, authorization can only be granted for imperative reasons of overriding public interest and in the absence of alternatives.}\\
$\rightarrow$ \textbf{The Presidential Decree 357/2003, transposes the Habitats Directive into Italian law}.

\item \textbf{Directive 2000/60/EC, Water Framework Directive (WFD)} : It establishes a common framework for water protection and management in the EU.\\
\textit{For Porto Marghera it includes the remediation. Since it is a SIN (Site of National Interest) the Italian State has direct responsibility for the site's remediation, regardless of any redevelopment projects.}\\

\item \textbf{Directive 2001/42/EC, Strategic Environmental Assessment (SEA)}: it ensures that environmental aspects are integrated into plans and programs at an early stage.\\
\textit{The SEA Directive (2001/42/EC) applies to strategic planning for the redevelopment of Porto Marghera, ensuring that environmental priorities are integrated upstream into zoning and industrial reconversion choices. The EIA Directive (2011/92/EU), on the other hand, applies to each specific project implementing the plan.  So applies to the masterplan.}\\
$\rightarrow$ \textbf{Transposed into Legislative Decree 152/2006.}

\item \textbf{Directive 2004/35/EC, Environmental Liability Directive (ELD)}: Introduces the “polluter pays” principle and the obligation to prevent or repair environmental damage.\\
\textit{In the absence of a clearly identifiable or solvent polluter, Directive 2004/35/EC stipulates that the State becomes responsible for remediation as a last resort. Porto Marghera was therefore classified as a SIN in 1998, transferring responsibility for remediation to the Ministry of the Environment (now MASE). Since then, the operations have been financed by national and European public funds (ERDF, LIFE, etc.), and not by the companies that originally carried out the remediation.}\\
$\rightarrow$ \textbf{European obligations (WFD + ELD) are transposed into Italian law (Legislative Decree 152/2006).}

\item \textbf{Regulation (EC) No 850/2004, POPs Regulation}: Aims to eliminate and restrict Persistent Organic Pollutants (POPs) such as dioxins, PCBs, PAHs and other long-lasting contaminants.\\
\textit{POPs are historically present in Porto Marghera due to petrochemical and metallurgical activities, therefore this applies to strict remediation obligations. Any redevelopment project must ensure that excavated contaminated materials are safely treated, disposed of or destroyed, following POPs elimination objectives.}\\
$\rightarrow$ \textbf{Directly applicable in Italy (EU Regulation) with national monitoring by the Ministry and ARPA Veneto}

\item \textbf{Regulation (EC) 1907/2006 (REACH)}:  it regulates the registration, evaluation and authorization of chemical substances in the EU to ensure high protection of human health and the environment.\\
\textit{In Porto Marghera, REACH applies to remaining industrial facilities and to the management of contaminated materials during remediation (soil, sludge, groundwater).
Any reuse, treatment, or transportation of polluted sediment and soil must comply with REACH obligations, particularly regarding hazardous substances and exposure prevention.
REACH strengthens control over chemicals historically released in the area (chlorinated solvents, hydrocarbons, metals).}\\
$\rightarrow$ \textbf{ Implemented in Italy through national enforcement under the Testo Unico Ambientale (D.Lgs. 152/2006) and specific REACH compliance monitoring by ISPRA/ARPA.}

\item \textbf{Directive 2008/50/EC, Ambient Air Quality Directive}: it sets limits for key air pollutants (NO$_2$, SO$_2$, PM$_10$, etc.)\\
\textit{The Veneto Region must integrate these obligations into its regional air quality plan (and potentially local plans for highly polluted areas like Marghera). Local authorities (municipalities, ARPA) are required to monitor air quality according to the national criteria of Decree 155. If an industrial zone exceeds the limit values, corrective measures must be adopted through an action plan (closure, emission reduction, traffic control, etc.). Citizens have a right to information: the data must be publicly accessible, which can serve as a tool for exerting pressure.}\\
$\rightarrow$ \textbf{In Italy, this is Legislative Decree 155/2010.}

\item \textbf{Directive 2010/75/EU, Industrial Emissions Directive (IED)}: it regulates major industrial installations to minimize emissions to air, water, and soil. Useful for remaining or new industrial facilities in Porto Marghera that require compliance or closure.\\
\textit{In Porto Marghera, existing industrial facilities are required to comply with the Industrial Emissions Directive (IED); otherwise, they risk suspension or termination of their activities. For any new installation, an Integrated Environmental Permit (IPPC/AIA in Italy) must be issued by the regional authority. This permit defines the specific Emission Limit Values (ELVs) for the plant, based on the Best Available Techniques and Associated Emission Levels (BAT-AELs) established at EU level, while also considering local environmental conditions.}\\
$\rightarrow$ \textbf{In Italy, Legislative Decree No. 152/2006 (“Testo Unico Ambientale”) implements the IED.}

\item\textbf{Directive 2011/92/EU, Environmental Impact Assessment}: it requires an environmental impact assessment for major public and private projects. Applies to infrastructure, remediation, and redevelopment projects in the port area.\\
\textit{Any significant redevelopment project in Porto Marghera must comply with the Directive and undergo a full Environmental Impact Assessment (EIA), given the sensitive environmental context and the presence of high-risk industrial activities. The EIA includes cumulative impact analysis, public participation, alternatives assessment and monitoring obligations. Projects may be rejected if impacts are considered unacceptable or insufficiently mitigated. General studies pre change.}\\
$\rightarrow$ \textbf{In Italy this is contained in the decree 152/2006 part 2 among other things}

\item \textbf{Directive 2012/18/EU, SEVESO III Directive}:it prevents and mitigates major industrial accident risks involving dangerous substances.\\
\textit{Several facilities in Porto Marghera are classified as SEVESO establishments → they must implement safety management systems, risk assessment and emergency planning.
Redevelopment cannot reduce safety standards: any land-use change must undergo a compatibility check with industrial risk constraints. Public right to information on risks is strengthened (risk transparency obligation)}. \\
$\rightarrow$ \textbf{Implemented in Italy through Legislative Decree 105/2015, coordinated by Prefecture and Regional authorities.}

\end{enumerate}

\subsection{On the regional level}
\subsubsection{A general overview of the plans of the region}
\begin{figure}[h]
\centering
\includegraphics[width=\linewidth]{images/02_PTRC.png}
\caption{Hierarchy of the regional plans and its territorial validity. Image extracted from the official website of the Veneto region.} 
\label{fig:gerarchia}
\end{figure}
Porto Marghera is part of the Municipality and Province of Venice, which in turn is part of the Veneto Region. The latter has to provide the whole territory with a series of plans \cite{regioneveneto}, as all Regions in Italy must do. In particular, the Veneto Region must arrange: 
\begin{itemize}
\item \textbf{Piano territoriale regionale di coordinamento (PTRC)}: it represents the regional tool to govern the territory and its use. It indicates the objectives and the main lines of organization and structure of the regional territory, as well as the strategies and actions aimed at their realization. It is structurally above a series of more specific tools that apply on smaller scales as in image \ref{fig:gerarchia}.

\textbf{The latest version was approved by the Giunta Regionale on the 06/30/2020}

\item \textbf{Piano Regionale di Tutela e Risanamento dell’Atmosfera (PRTRA)}:  it represents the regional tool to regulate activities that create pollutant emissions in the atmosphere.

\textbf{The latest version was approved by the regional Council on 04/19/2016. Currently a new version is under development, since the update approval of the Giunta Regionale of 11/11/2021. }

\item \textbf{Piano di Tutela delle Acque (PTA)}: it is the regional tool that regulates the protection and preservation of hydrological resources. It establishes the procedures to protect all types of water bodies and the methods to guarantee a sustainable use of water. It is divided into three parts: a descriptive document, a document with the plan directions and finally a document with the technical regulations.

\textbf{The latest version was approved by the deliberazione del Consiglio Regionale n.107 on 11/05/2009, and with the last modification approved with the D.G.R.V. n. 1023 on the 07/11/2018.}

\item \textbf{Piani di Gestione delle Acque dei Distretti idrografici delle Alpi Orientali e del fiume Po}: parallel tool regulated by the hydrological district. 

\item \textbf{Piano Regionale per la Bonifiche delle Aree Inquinate (PRBAI)}:  it is the functional instrument for the analysis of critical situations and the identification of priority interventions, with which the Region, in implementation of current legislation, carries out a sustainable management of its territory and its resources. It is important to note that this plan does not apply to Porto Marghera, since the presence of the SIN regulation. 

\end{itemize}

\subsection{Port Authority level}
The Port Authority in Italy is a separate entity with respect to the region and it refers directly to the national regulations. In other words, it is parallel to the region, thus it has its own set of laws for the territory of its property. In the following subsection an overview of its most important regulations is provided. 

\begin{itemize}
\item \textbf{The Piano Regolatore Portuale dell’Autorità Portuale di Venezia}: approved by the Ministero dei Lavori Pubblici with the Decreto n.319 of 15/05/1965. With the Riforma della Legislazione Portuale of D.Lgs.169/2016 the new Authorities of the Harbour System have been established, in this specific context with the Autorità di Sistema Portuale del Mare Adriatico Settentrionale (AdSPMAS), regarding the ports of Venice and Chioggia, now seen as a unified system. 
\item \textbf{Three-year operational plan}: The Three-Year Operational Plan is a planning document issued by the North Adriatic Sea Port Authority, required by the Law No. 84/1994. It is revised annually and it is prepared every three years. It regards the planning of the port activities, actions and interventions in order to achieve previously set objectives.

\end{itemize}

\subsection{On the Province level (Città metropolitana di Venezia)}
As mentioned above, Porto Marghera forms part of the Province of Venice, whose specific name is “Metropolitan city of Venice” since law 56/2014. This authority must arrange a series of important plans and regulations on its territory both in the environmental and urban field. 

\textbf{Il Piano Territoriale Generale Metropolitano}:
Since the switch from the Province of Venice to the Metropolitan City of Venice in 2014, the previously existing Piano Territoriale di Coordinamento Provinciale (PTCP) was renamed as “Piano Territoriale Generale (PTG) della Città Metropolitana di Venezia” keeping the contents unvaried. Therefore, the PTG dates back to 2019, but it actually contains the same contents of the latest version of PTCP, which in turn dates back to 2014. 

\textbf{Il Piano Territoriale di Coordinamento Provinciale}
The PTCP outlines development and governance strategies for the Province of Venice. Adopted in 2008, the Plan is defined as open, flexible, and concerted, born out of a broad participatory process aimed at addressing a rapidly changing economic and regulatory environment.

\textbf{Piano di Assetto del Territorio}
The Piano di Assetto de Territorio (PAT) outlines strategic choices for the structure and development of the territory, identifying specific characteristics and constants of a geological, geomorphological, hydrogeological, landscape, environmental, historical, monumental, and architectural nature. The PAT is a “structural plan,” or a planning document that:
\begin{itemize}[leftmargin=*, noitemsep]
\item adopts the general guidelines of higher-level instruments (PTRC, PTCP, PALAV) and municipal instruments relating to the wider area (Strategic Plan, Urban Mobility Plan).
\item outlines major choices for the territory and strategies for sustainable development;
\item defines the functions of the different parts of the municipal territory;
\item identifies areas to be protected and enhanced due to their environmental, landscape, and historical-architectural importance;
\end{itemize}

%---------------sezione Accordi di Programma ----------------------------------------------
\section{"Accordi di Programma" for Porto Marghera}
Given the legislative evolution for Porto Marghera presented in paragraph 2.1, a section of explanation with a bit more detail about the \textit{Accordi di Programma} seems much needed. Thus, the following sub-paragraphs present the objectives of the most relevant Program Agreements signed between 1998 and 2012. 
\subsection{Accordo di Programma per la Chimica (1998)}
The Program Agreement for Chemicals in Porto Marghera was signed on October 21, 1998, and approved by DPCM of February 12, 1999. Its objective was to establish and maintain optimal conditions for coexistence between environmental protection and production development in the chemical sector.
The 1999 Chemicals Agreement set the following objectives:
\begin{itemize}[leftmargin=*, noitemsep]
\item Rehabilitate and protect the environment through cleanup, remediation, and site safety measures, reduction of emissions into the atmosphere and the lagoon, and prevention of major accident risks (SIMAGE)
\item Innovation, competition, employment: Inducing adequate industrial investment, with the aim of equipping existing plants with the best environmental technologies and making them competitive at European level, ensuring their long-term viability and ensuring the maintenance, revitalization, and qualification of employment.
\end{itemize}
A supplementary Act was signed in November 2001 by Ministries, Region, local institutions, trade unions and the most important firms working on the area.

\subsection{Accordo di Programma per l'Idrogeno (2005)}
The Program Agreement for the Porto Marghera hub stems from the availability of hydrogen already present in the area, derived as a byproduct of petrochemical industrial processes. This resource represents a competitive advantage that enables immediate and low-cost procurement, essential for fueling research. The project leverages the consolidated expertise of local companies, already united in the "Hydrogen Park – Marghera per l’Idrogeno" consortium, which serves as a strategic driver for the district's development and guarantees excellent expertise across the entire gas supply chain.

The four-year initiative is designed as an open-air innovation laboratory aimed at testing cutting-edge solutions for the production, storage, and use of hydrogen. The central objective is the integration of fuel cells in both stationary electricity generation and sustainable mobility. Through this synergy between industry and research, the Agreement aims to transform Porto Marghera into a national hub for the energy transition and the use of zero-emission technologies. In addition to this, the Program Agreement for Hydrogen led to the creation of plants for CO$_2$ catchment and reuse. The Program Agreement was later extended with the Addendum of December 2009.

\subsection{Accordo di Programma Vallone Moranzani (2008)}
The "Accordo di Programma per la gestione dei sedimenti di dragaggio dei canali di grande navigazione e la riqualificazione ambientale, paesaggistica, idraulica e viabilistica dell'area di Venezia - Malcontenta – Marghera" (also known as "Accordo Moranzani", was signed in 2008 by a number of entities. The "Moranzani" Program Agreement arose from the need to identify a permanent disposal site for dredged sediment from the port canals, an alternative to the site initially envisaged by the Progetto Integrato Fusina. More is presented in paragraph 2.5.1.

\subsection{Accordo di Programma per le Bonifiche (2012)}
Following reflections on the administrative burden, highlighted as one of the main critical issues in the remediation process, a key administrative act was adopted that represents a new beginning in the Porto Marghera reclamation process: on April 16, 2012, a Program Agreement for the remediation and environmental redevelopment of the Porto Marghera and surrounding areas was signed by the Ministry of the Environment, the Venice Water Authority, the Veneto Region, the Province of Venice, the Municipality of Venice, and the Venice Port Authority. The signed Program Agreement had two fundamental objectives:
\begin{itemize}[leftmargin=*, noitemsep]
\item to accelerate and simplify the remediation procedures for the Porto Marghera, supporting businesses in accessing credit for the implementation of the interventions
\item to define an initial list of new projects to be implemented in the area with simplified procedures, open to further participation
\end{itemize}

%----------------sezione detailed -------------------------------------------------------------------
\section{Detailed study of the relevant plans for Porto Marghera}
In this section, more insights are provided about some of the planning tools presented in the previous part. More specifically, the analysis delves into four regulatory plans:
\begin{itemize}[leftmargin=*, noitemsep]
  \item \textbf{Piano Regolatore Portuale per il Porto di Venezia} (1965): its the most important regulatory tool for the area managed by the Port Authority and it is still valid today, in spite of its outdatedness.
  \item \textbf{Variante al Piano Regolatore Generale per la zona di Porto Marghera} (1999): the very first serious regulatory plan created by the Municipality of Venice that affirmed its decisional power on the area. It is also the first official document where the critical condition of the environment in Porto Marghera is confirmed.
  \item \textbf{Masterplan per la bonifica dei siti inquinati} (2004): its the only tool that assessed in a technical way how to operate the remediation of the polluted soil and groundwater. 
  \item \textbf{Piano di Assetto del Territorio} (2014): its the most recent regulatory plan for the area of Venice that gives a picture of the evolution of Porto Marghera during the beginning of the new century in addition to showing the expectations for the future of the area.
\end{itemize}
%---------------- Porto -------------------------------------------------------------------------------
\subsection{Piano Regolatore Portuale per il Porto di Venezia}
The Port Regulatory Plan is the planning tool for the system of ports that are under the control of the Port System Authority, in the specific case of Porto Marghera of the North Adriatic Sea Port Authority. It is defined as such by the Port Reform Law 85/1994 and its subsequent amendments and integrations, such as the Legislative Decree No. 169/2016 of August 4, 2016 and the Legislative Decree No. 232/2017 of December 31, 2017, known as “Port Corrections”.
The current PRP dates back to the year 1965, but a new plan is currently undergoing drafting.
\subsubsection{Structure}
The Piano Regolatore Portuale (PRP) is made up of two different levels:
\begin{itemize}[leftmargin=*, noitemsep]
\item The System Strategic Planning Document (DPSS): it is the document that describes the complete vision of the port area managed by the Port Authority of the Adriatic Sea.
\item The Port Regulatory Plans: these are the specific plans for each port and port areas, such as Porto Marghera.
\end{itemize}
The current 1965 plan for the port industrial area includes the site plan, the excavation section and the use destinations.
\begin{figure}[h]
\centering
\includegraphics[width=0.77\linewidth]{images/02_portplan.png}
\caption{Planimetry of the Port Regulatory Plan (1965), available through the Port Authority of the North Adriatic Sea} 
\label{fig:portplan}
\end{figure}
\subsubsection{The new plan}
The Port Authority of the North Adriatic Sea is currently drafting two new RPPs, one for Venice, which will include Porto Marghera, and one for Chioggia, which has a plan dating back to 1908.

As of 2025, the DPSS is going to finish its draft and be approved. The new plan is being thought out by a technical group from AdSP and by a group of enterprises, such as Rina, StudioPaolaViganò, Acquatecno, Mtbs.
The DPSS defines the main functions as:
\begin{itemize}[noitemsep]
\item Port functions, such as warehouses, docks and cargo terminals
\item Port hinterland areas functions, such as logistics, industries and interactions with the city
\item Infrastructure, such as railways.
\end{itemize}

\subsubsection{Objectives}
\begin{itemize}[leftmargin=*, noitemsep]
\item Environmental regeneration of the brownfield areas of Porto Marghera, that are mostly non-efficient or disused
\item Sustainability, with more efficient land and soil use and the reduction of the use of new soil.
\item Implementation of new terminals, such as a container terminal in the Montesyndial area, through a series of strategic investments
\item Implementation of new railway connections to improve freight transport, more specifically in the Via della Chimica hub.
\end{itemize}

\subsubsection{Relationship with other plans}
The PRP has to operate with the Master Plan per la bonifica dei siti inquinati di Porto Marghera, as the actions described in the Master Plan have to take place in the area which is regulated by the PRP.

For the same reason, within the Accordo di programma per la riconversione e riqualificazione dell’area di crisi industriale complessa di Porto Marghera, there are relationships between the AdSP and the Ministero dello Sviluppo Economico, the Veneto Region, and the Municipality of Venice.
The area division as planned with the PRP is also utilized by the Piano di Raccolta e Gestione dei Rifiuti of the Port Authority.

\subsubsection{Main issues}
The fact that the current PRP dates back to 60 years ago, which brings the disadvantage of it not being specifically designed for the modern needs of sustainability and regeneration, as well as new functions.

The PRP has to be coordinated with infrastructure investments, land use plans and remediations plans, in order to plan for an efficient industrial conversion. Furthermore, there has to be balance between the port function and the urban regeneration function, making sure that there are areas that efficiently carry on with the classic port functions, and other ones that are converted to city areas.

The revisions to the plan need a wide number of steps, both technical and administrative, such as the conference of services, the VAS and the VINCA, which takes a considerable amount of time.

\begin{figure}[H]
\centering
\includegraphics[width=\linewidth]{images/02_newportplan.png}
\caption{Most recent planimetry as defined by the Decreto n. 865 of 27/12/2022, to define areas that have lost the port function and are considered areas of port-city interaction. Extracted from the official website of Port Authority.} 
\label{fig:portplan}
\end{figure}
%---------------------------- Venice Masterplan --------------------------------
\subsection{Masterplan for Porto Marghera}
The Variante to the General Master Plan PRG of Porto Marghera (1999) represents one of the most important urban planning instruments adopted by the City of Venice and approved by the Veneto Region, specifically designed to reorganize and redevelop a large industrial and port area. Historically, the planning of Porto Marghera had been dominated by the State and the Port Authority due to the area's particular legal regime and to the undefined jurisdiction of the Municipality of Venice, which went through the many legislative changes that followed the creation of the Italian Republic. As a matter of fact, since the construction of Porto Marghera in 1917, the area witnessed a great variety of historical events (such as the II World War, the fascist regime and the aforementioned birth of the Italian Republic) during which territorial regulation wasn’t a priority. 
\subsubsection{Assessed problems}
\textbf{1. No previous planning}: Porto Marghera has not, until 1995, been the subject of explicit planning action by the Municipality, which has limited itself in its urban planning instruments to merely sanctioning its designated land use, without addressing the merits of its physical internal organization. The plans elaborated in the past for the industrial zone—in 1925 and 1960, respectively for the first and second zones—were drawn up and sanctioned by the State and subsequently modified repeatedly by the Provveditorato al Porto (Port Authority) through its own Port Master Plan (Piano Portuale). This was due to the particular legal regime under which the zone itself had been constituted at the time, which made the Municipality's jurisdiction uncertain until 1995. It’s only with the Variante al Piano Regolatore Generale per la zona di Porto Marghera (VPRG) that the Municipality of Venice, in accordance with the PALAV, can affirm its authority and its planning jurisdiction on the industrial area.
This implies a series of problems that are the result of the lack of planning:
\begin{itemize}[noitemsep]
\item Urban isolation
\item Inadequate infrastructure
\item Environment degradation
\end{itemize}

\textbf{2. Economic and industrial crisis}: The VPRG seeks to drive a restructuring process designed to overcome the industrial contraction experienced since the ‘70s. The VPRG states that Porto Marghera continues to perform vital functions for the local economy. Its importance has not been undermined by previous restructuring cycles, since it guarantees employment and it gives diversification to Venice’s economy. However, at the time of the VPRG, Porto Marghera is in the middle of an economic crisis: the area has, in fact, been included among the "Aree di crisi" by the Government's Task Force for the Economy to counter the ongoing "deindustrialization. The causes of this crisis are multiple and complex, but it is possible to highlight mainly that: 
\begin{itemize}[noitemsep]
\item there are structural difficulties in the traditional ("historical") sectors of the hub, such as basic chemistry, aluminum, and coke processing, which were the main industrial activities to develop at the time of Porto Marghera’s birth.
\item there is absence of new and significant entrepreneurial initiatives
\item Poor control over the ripple effects of the crises faced by the hub's large basic companies
\end{itemize}
At the level of public governance, negative trends have been fueled by a series of unresolved issues and critical decision-making failures, notably the failure to dredge the industrial canals and absence of a unitary project for the industrial hub's reconversion. This complex combination of factors and the excessive bureaucracy, has prevented the creation of necessary synergies for development. 

\subsubsection{Planning solutions}
The VPRG’s proposed solutions for overcoming the aforementioned problems are listed in the following part: 

\textbf{1. No previous planning $\rightarrow$ General reorganization of the area, creation of reclamation and civil protection plans}: 
\begin{itemize}[leftmargin=*, noitemsep]
\item new zoning on the basis of studies on the problems of the area, on the existing infrastructure for gas and water and the presence of cultural heritage sites. 
\item redevelopment of the Venice Port:  the restoration of 50 ha occupied by abandoned industrial plants, with the idea of relocating logistics enterprises inefficiently located in the area. 
\item creation of new streets for Via dell’Elettricità: the main objective is reducing heavy traffic caused by the harbor
\end{itemize}
\textbf{2. Economic and industrial crisis $\rightarrow$ Creation of space for innovation and new small enterprises}
\begin{itemize}[leftmargin=*, noitemsep]
\item Creation of space for small and mid-sized companies: a project prepared by the Municipal Administration in order to entrust territorial lots to chosen enterprises. 
\item Establishment of “Parco Scientifico Tecnologico”: This project was born in 1992 under the protection of the European Community, with the participation of public institutions and private companies interested in technological innovation. The project establishes a gradual recovery of abandoned areas in the 1st industrial zone. The main objective is connecting this area with the University of Venice, in order to create a direct connection between the enterprises and the technological innovations.
\end{itemize}

\subsubsection{Strategy}
The “piano di intervento” considered by the VPRG is constituted of 5 main points: 
\begin{enumerate}[leftmargin=*]
\item The strategic use of available resources:
\item The promotion of memoranda of understanding between the Municipality of Venice and other entities with jurisdiction over the area (Veneto Region, Province of Venice, Port Authority/Port Superintendence) or stakeholders (State Railways, landowners, etc.). Two other parties are especially considered: the "Provveditorato di Porto" and the existing companies. 
\item The definition of infrastructure interventions of territorial significance (new southern access to the port area; separation of urban traffic from industrial traffic; connection between the industrial area and Via Torino; etc.).
	\begin{itemize}[noitemsep]
		\item a new set of streets to connect Porto Marghera with the outside
		\item reorganization of the streets inside the hub
		\item expansion of the Terminal Fusina
		\item street connection between the university and the industrial hub
	\end{itemize}
\item A zoning of the industrial hub's areas functional to the valorization of Porto Marghera's specific potential.
\item The definition of terms and methods of land use (Technical Implementation Standards) able to ensure the flexibility necessary for the remediation processes that are intended to be promoted and, at the same time, capable of providing the certainties required by entrepreneurial forces.
\end{enumerate}
\subsubsection{Evaluation of achieved results}
The VPRG succeeded in realizing some of the projects included in its objectives: in this subsection an assessment of the most relevant achievements is made:
\begin{itemize}
\item The creation of the "Parco Scientifico Tecnologico VeGa": In 1993, the company Venice Gateway for Science and Technology (VeGa) was instituted. VeGa is a "Parco Scientifico Tecnologico",  a compound of innovative firms of high technology. It was originally founded by 34 public and private partners to promote the development of the Porto Marghera industrial area through the creation of research centers, activities, and services. It started to be build in the late 90s, on the territory that was previously occupied by "Enichem Agricoltura", an abandoned area since 1986. In its first 10 years of activity, VeGa promoted the redevelopment of 35.000 m$^2$ of land through the construction of new buildings (mainly for office and university use) thanks to the EU structural funds. Nowadays it hosts companies working in ICT, green economy and environmental sustainability. 
\item Parco San Giuliano: Parco San Giuliano was a former marshland area that was used as a dumping ground for industrial sludge and waste during the peak years of production in Porto Marghera. In accordance with the objective of the PRG of creating green spaces, it was transformed into a park in 2003 with the aim of conserving and safeguarding the lagoon's habitat, fauna, and flora, thereby combating environmental degradation. Nowadays it is still an important green area for the inhabitants of the city of Mestre.
\item SIMAGE system: SIMAGE is the acronym of "Sistema Integrato per il Monitoraggio Ambientale e la Gestione delle Emergenze" a prevention system to manage industrial risk in Porto Marghera. It consists of a monitoring network made up of sensors placed directly inside the industrial plants. Its functionality over the years has been guaranteed by ARPAV, EZI (Porto Marghera Industrial Zone Authority), APV (Port Authority of Venice), and companies of the ENI group in Marghera. Its creation is one of the achievements of the VPRG, which included it inside its plan of civil protection, and of the Accordo di Programma per la Chimica of 1998. It is still working today and its functioning is currently regulated through the latest Accordo di Programma for the triennium 2023-2026.
\end{itemize}
\begin{figure}[H]
\centering
\includegraphics[width=\linewidth]{images/02_achievements.jpeg}
\caption{Location of the aforementioned achievements. Map created by the authors with QGIS.} 
\end{figure}

%--------------------------- Remediation Masterplan ----------------------------
\subsection{Masterplan per la bonifica}
The Masterplan per la bonifica dei siti inquinati di Porto Marghera (2004) is the main planning instrument dedicated to the coordination and regulation of environmental remediation interventions within the Porto Marghera National Priority Site (Sito di Interesse Nazionale – SIN). It was developed to provide a unified strategic and operational framework for the clean-up of contaminated soils, groundwater and sediments, ensuring coherence between remediation processes, territorial planning tools and redevelopment strategies.

\subsubsection{Main structure}
The Masterplan is structured as an integrated technical and planning document composed of several complementary components.
First, it includes a territorial and environmental classification framework, based on site characterization data, which subdivides Porto Marghera into homogeneous areas according to contamination typologies, risk levels and hydrogeological conditions.

Secondly, the plan defines a zoning system for remediation priorities, identifying areas that require urgent intervention and those that can be addressed in later phases, depending on environmental risk and redevelopment perspectives.

Finally, the Masterplan contains a set of technical guidelines and operational procedures, regulating the different phases of remediation, from preliminary investigations to monitoring after intervention, and defining the institutional responsibilities of the involved actors.

This structure allows the Masterplan to act both as a strategic coordination tool and as a technical reference framework for individual remediation projects.

\subsubsection{Existing issues}
The Masterplan addresses a series of structural and operational issues that characterize remediation processes in Porto Marghera.
One of the main challenges is the complexity and heterogeneity of contamination patterns, which involve different pollutants, environmental matrices and spatial distributions. This complexity requires differentiated remediation approaches and limits the applicability of standardized technical solutions.
Another major issue is the strong interaction between soil, groundwater and surface water systems, due to the presence of canals and hydraulic connections with the lagoon. This condition increases the risk of pollutant dispersion and complicates containment and treatment strategies.
In addition, the fragmentation of land ownership and institutional responsibilities represents a critical governance challenge. The involvement of multiple public authorities and private stakeholders increases administrative complexity and can slow down decision-making and implementation processes.
Finally, remediation activities are affected by high economic costs and long implementation timelines, which may delay redevelopment projects and reduce the attractiveness of private investments in the area.

\subsubsection{Objectives}
The general objective of the Masterplan is to ensure the systematic and coordinated remediation of contaminated areas in Porto Marghera, reducing environmental and health risks while enabling the future reuse of industrial brownfields. More specifically, the plan aims to
\begin{itemize}[leftmargin=*, noitemsep]
\item reduce environmental and health risks associated with contaminated soils, groundwater and sediments in Porto Marghera;
\item enable the safe reuse and redevelopment of brownfield areas, removing environmental constraints to territorial transformation;
\item restore environmental quality and ecological functionality in heavily impacted industrial zones;
\item ensure long-term environmental safety, preventing future pollutant dispersion towards the lagoon and surrounding urban areas;
\item support the structural reconversion of the industrial district by providing a stable environmental framework for new economic activities.
\end{itemize}

\subsubsection{Strategies and actions}
The Masterplan defines a set of coordinated remediation strategies aimed at ensuring environmental safety while supporting the progressive redevelopment of Porto Marghera. The main strategic guidelines can be summarized as follows:
\begin{itemize}[leftmargin=*, noitemsep]
\item The adoption of a risk-based remediation approach, prioritizing interventions according to exposure pathways, environmental sensitivity and potential impacts on human health, rather than relying exclusively on concentration threshold values.
\item The phased remediation process, articulated through different levels of intervention: Messa in Sicurezza d’Emergenza (MISE), aimed at the immediate containment of active pollution sources; Messa in Sicurezza Permanente (MISP), focused on long-term containment and isolation of contaminated matrices; Bonifica, consisting of the removal or treatment of contaminated soils and groundwater where technically and economically feasible.
\item The integration between remediation and land-use planning, ensuring that clean-up strategies are compatible with current and future land uses defined by urban, port and territorial planning instruments.
\item The coordination between soil, groundwater and sediment management, particularly in canal areas and waterfront zones, where contamination processes are strongly interconnected with hydraulic dynamics and dredging activities.
\item The promotion of technically flexible remediation solutions, encouraging the adoption of site-specific approaches, including in-situ treatments, hydraulic barriers, soil containment systems and selective excavation techniques, depending on local environmental conditions.
\end{itemize}

In addition, the Masterplan establishes technical planning instructions and regulatory rules governing site characterization methodologies, project approval procedures, monitoring requirements and post-remediation control phases. These provisions are supported by a comprehensive cartographic system, including contamination maps, remediation priority zoning, hydrogeological vulnerability layers and land-use compatibility maps, which facilitate coordination between environmental remediation and territorial planning processes.

\subsubsection{Way of action and implementation}
The implementation of the Masterplan is based on a multi-level governance framework involving national, regional and local institutions. The Ministry of Environment plays a central coordination role, while the Veneto Region, the Municipality of Venice and the Port Authority are responsible for the integration of remediation actions with territorial planning and port activities. Environmental monitoring and control functions are carried out by specialized agencies such as ARPAV (Agenzia Regionale per la Prevenzione e Protezione Ambientale del Veneto).

Operationally, remediation projects are implemented through programmatic agreements and administrative procedures that define responsibilities, funding mechanisms and execution timelines. 

\subsubsection{Evaluation}
Despite the structured framework introduced by the Masterplan in 2004, the progress of remediation activities in Porto Marghera has been relatively slow. By 2016, although remediation procedures had been activated on more than 90\% of the area, only about 14\% of the industrial district had effectively completed the clean-up process (see figure \ref{fig:SINfalda} and \ref{fig:SINterreni}). A large share of the sites remained at intermediate administrative or characterization stages. This situation highlights the difficulty of translating strategic planning objectives into concrete remediation results, mainly due to technical complexity, long authorization procedures and high intervention costs.
%-------------------------- PAT--------------------------------
\subsection{Piano di Assetto del Territorio}
The Piano di Assetto del Territorio (PAT), as defined in Article 13 of Regional Law 11 of 2004, establishes the objectives and sustainability conditions for eligible interventions and transformations and is drawn up by Municipalities based on ten-year forecasts. Venice's PAT was approved on 10/10/2014 and it is still valid: it refers to the PRG of 1999 by evaluating the evolution of the territory throughout the years and by setting the objectives for the future years. 
\subsubsection{Structure}
The PAT of the municipality of Venice is subdivided into three main parts: first, a knowledge framework (\textit{quadro conoscitivo}), secondly, a strategic environmental evaluation (\textit{Valutazione Ambientale Strategica (VAS)}) and finally the recollection of technical regulations together with the cartography (see figure \ref{fig:convertibility} for a re-design of the convertibility map of the area, one of the most interesting products of the PAT).  For the goals of this work, it is relevant to consider only what is mentioned regarding the environment and Porto Marghera specifically.
\subsubsection{Objectives}
The objectives of the PAT of Venice are stated in an assigned paragraph, where it is underlined that all of its objectives are in accordance with national and regional legislation, as clearly explained in the previous paragraphs. All objectives are divided into subgroups according to their common theme. These subgroups are: 
\begin{itemize}[leftmargin=*, noitemsep]
\item Environmental protection
\item Protection of cultural, historical and architectural heritage
\item Settlement system
\end{itemize}
Porto Marghera has a set of objectives of its own inside the section of "business activities", located inside the "settlement system part". Its condition of decommissioning of the plants is considered, but it is still regarded as one the core of economic activity in the municipality. Marghera's set of goals reads: 
\begin{enumerate}[leftmargin=*, noitemsep]
\item consolidation and possible expansion of productive activities (especially shipbuilding and new environmentally sustainable technologies)
\item physical and functional redevelopment of the area (introduction of new urban functions especially in the area near Via Fratelli Bandiera which could represent a potential extension of the city and its services)
\item productive reconversion of disused or underutilized areas with the expansion of port-related functions
\item improvement of accessibility from the south (connection with "Romea commerciale" and SP81)
\end{enumerate}
In addition, as a project choice, in the PAT it is stated that Porto Marghera is considered as a place of development of logistic integrated platform and/or as an opportunity for innovation and research. Instead, regarding the environmental clean-up, the PAT declares complete accordance with the \textit{"Protocollo di condivisione delle linee strategiche per la riqualificazione e lo sviluppo di Porto Marghera"} signed on the 21/09/2007 by the veneto Region, the Province of Venice, the Municipality of Venice, the ARPAV, the "unione degli Industriali"  and trade union organization. 
\subsubsection{Strategy: Ambito Territoriale Omogeneo di Porto Marghera}
The PAT subdivides the whole territory in "Ambiti Territoriali Omogenei (ATO)", which means areas that have common territorial features. For the ATO of Porto Marghera, the PAT considers the following planning choices :
 \begin{itemize}[leftmargin=*, noitemsep]
    \item the consolidation and strengthening of port functions, including the possible location of a new cruise ship terminal
    \item the identification of redevelopment and/or conversion areas across much of the district, provided they are not already involved in implementation plans, given the potential Porto Marghera expresses within the territorial and infrastructural context 
    \item the potential functional reconversion, while remaining consistent with the \textit{Accordo di Programma per la Chimica}, primarily aimed at establishing innovative and environmentally sustainable industrial production
     \item the identification of the areas between via dell'Elettricità and via F.lli Bandiera as strategic for the improvement of urban and territorial quality, both because they are in direct contact with the residential areas of Marghera (to the point of suggesting a small residential quota), and because their redevelopment/transformation is functional to improving accessibility from the South
    \item the enhancement of the waterfront through a project aimed at mitigating the impact of the industrial skyline through environmental and landscape redevelopment works.
\end{itemize}
Finally, it is interesting to note that in the part about the numerical sizing of the PAT, in the non-residential part, the additional settlement load (\textit{carico insediativo aggiuntivo}) is mainly attributed to Porto Marghera, with a value of 4.550.000 m$^2$ on a total of 7.146.000 m$^2$ distributed on the total area of the municipality of Venice. 
\subsubsection{Evaluation}
The PAT succeeds in establishing a set of goals that consider the population and its possible use of former industrial areas. However, it lacks in considering the population's real needs, that should be assessed with participatory processes. 

When it comes to evaluating what has actually been accomplished, the list remains quite short: no changes where made to the port, the reconversion of many buildings is still unaccomplished and no landscape redevelopment works were started. Nevertheless, some steps forward were made: about reconversion, in 2017 Venice Heritage Tower was created; a new cultural space was born from a former cooling tower \cite{venicetower}. 
Regarding the environment, the PAT does not really assess the problem directly and mainly focuses on the industrial reconversion of the area. 
\begin{figure}[H]
\centering
\includegraphics[width=\linewidth]{images/02_venicetower.png}
\caption{Picture of the renovation of the Venice Heritage Tower. Extracted by \cite{venicetower}} 
\label{fig:tower}
\end{figure}

%--------------------------------------sezione projects--------------------
\newpage
\section{Projects}
\subsection{Progetto Integrato Fusina}
The Progetto Integrato Fusina (PIF) is a complex project that concerns the construction of a plant for the collection, treatment and safe discharge of dangerous water of various origin. The PIF is presented as a high-profile engineering work of strategic important that serves a preeminent public function. Its core objectives are: 
 \begin{itemize}[leftmargin=*, noitemsep]
 \item Pollution reduction within the drainage basin of the Venice Lagoon by drastically limiting discharges, even those that have been treated.
 \item The remediation of contaminated sites in Porto Marghera, where the PIF serves as the key component for the water cycle.
 \item The optimization of water resource management by implementing extensive recycling of water used for industrial purposes.
 \end{itemize}
Technically, the project plans a treatment within a single functional platform of industrial discharges, contaminated groundwater resulting from the safety measures implemented at the Porto Marghera site, runoff water from potentially contaminated sites and all domestic sewage and rainwater from Mestre and Marghera. This should sum up to a total of 160.000 m$^3$ per day of treated wastewater, that is later put under refinement in a phytodepuration wetland ("Cassa di Colmata A"). When water is renewed, it is destined to industrial use in cooling plants, saving hundreds of thousands liters of drinkable water \cite{pif}. 
Furthermore, the project includes the construction of a final discharge into the open sea via an underwater pipeline. This pipeline transfers treated water from the final section of the plant to a point located approximately 10 km offshore from the Venice Lido, adhering to limits that are even stricter than those imposed by the European Community.
In addition, the project includes also the creation of a didactic - ludic park of 10 hectares on the South of Porto Marghera, in accordance with the objectives of the Accordo "Moranzani". 
\begin{figure}[H]
\centering
\includegraphics[width=0.85\linewidth]{images/02_pif.jpg}
\caption{Map of water collection channels of the PIF. Extracted by \cite{pif}} 
\label{fig:portplan}
\end{figure}

\subsection{Veritas SpA Eco Park}
In the southern part of Porto Marghera, in the same area of the development of the Progetto Integrato Fusina, a relevant example of Industrial Symbiosis (IS) is operating since 2017. An Industrial Symbiosis is a production system that engages traditionally separate entities in a collective approach involving physical exchange of by-products to substitute raw material input \cite{is}. The IS in Porto Marghera is an example of urban and industrial symbiosis network of national relevance: it enables a high percentage of materials recycling thanks to the collaboration between the public waste collection company Veritas and other connected industries that work in the sorting and recycling of materials.

The Ecodistretto di Porto Marghera was born in 2017, after the interruption of the former waste-to-energy plant. Its creation was driven by two different companies, Ecoprogetto Venezia S.r.l. and Eco-Ricicli Veritas, which converged into one in 2022, thus becoming Eco+Eco S.r.l. . This company forms part of the Veritas group, a municipally owned S.p.A. and one of the largest multiutility in Italy. The former Ecoprogetto Venezia S.r.l. was an integrated facility for processing unsorted waste; meanwhile, Eco-Ricicli Veritas operated as a platform for sorting waste from separate collection. Now, the two plants are still divided, but together they provide an environmentally-safe management of waste for approximately 890.000 people \cite{ecodistretto}. The Ecodistretto also welcomes third-party companies engaged in recycling materials at the early stages of the value chain.
\begin{figure}[H]
\centering
\includegraphics[width=\linewidth]{images/02_ecoprogetto.png}
\caption{Map of the industries engaged in the Eco Park. Extracted from \cite{ecodistretto}} 
\label{fig:ecodistretto}
\end{figure}


\subsection{VeGa waterfront}
VeGa's Venice Waterfront project is the first part of a development project for the northern macro-area of Porto Marghera. Twenty years after the construction of the Venice Science and Technology Park, Expo 2015 became a driving force for the VeGa project, which reclaimed a disused area along the Brentella canal, the first of four planned development scenarios.
The pretext for the regeneration of this initial site was the \textit{Venice Green Dream 50x50}, a collateral event at the 2012 Architecture Biennale, which triggered the reclamation process of a portion of the area.
Thanks to the extraordinary legislative tools provided by Expo, the area was reclaimed, the route connecting it to the Porto Marghera train station was redeveloped, and the\textit{ Expo Aquae Pavilion} was built, with its experimental showcase for phytoremediation.

\begin{minipage}{0.65\textwidth}
\vspace{0pt}
The project, as described by VeGa, is part of a broader vision: \textit{the green tree strategy}, a metaphor that sees Venice islands as the roots of a tree, and its mainland as the green branches and leaves, the symbol of development opportunities. This urban-environmental regeneration strategy does not appear to have any local implications. The leaves  have no geographical connection. The Green Tree Strategy and the strategic development of Venice's port system trace the boundaries of many "Porto Margheras", which are irreconcilable. Both visions, while interesting, should perhaps clarify and question the nature of the points where the geography of flows (and the geography of economic interests) meets the geography of places. 

{\footnotesize (Translated from: Peron I. (2016), Potenzialità contese. Porto Marghera, una questione di metodo.) }
\end{minipage}
\hspace{10pt}
\begin{minipage}{0.3\textwidth}
\centering
\includegraphics[width=\linewidth]{images/02_treestrat.pdf}
\captionof{figure}{Representation of the Green Tree Strategy by VeGa.}
\end{minipage}

\subsection{Hydrogen Valley in dismissed area (ex Sapio)}
Thanks to PNRR funds, in 2022 the Italian Ministry of the Environment and Energetic Security (MASE) published a funding notice to promote the construction of "hydrogen valleys" in dismissed industrial areas \cite{mase_hydrogen}. Porto Marghera adhered to this notice and so the project for the Venetian hydrogen valley began. 

The chosen dismissed plant that is subject of the renovation is the Sapio plant, located in the western part of Porto Marghera (see image ...). The project considers also using part of the already existing equipment of the previous firm: as a matter of fact, the Sapio used to work in the mixing of gases for petrochemical production. Furthermore, together with the Sapio reconversion, a photovoltaic plant is build in the area of Fusina, in order to sustain the hydrogen production using renewable resources.

The project plans the hydrogen production mainly for industrial use, in order to sustain the decarbonization of heavy plants of the area, but hydrogen is also destined to logistical use for the public transport in the mainland. The total nominal capacity  of the electrolytic cell is planned to be 5 MW, which correspond to a hourly capacity around 1.000 Nm$^3$/h of hydrogen. \cite{hydrogentech}

\section{Other proposals}
The following proposals are the result of the latest efforts made by citizens, activists, researchers and professionals to deal with the issues of Porto Marghera. As it is apparent, there is not a wide variety of them, a clear sign of the great difficulty that all of the subjects have to face trying to make changes to the current situation.

\subsection{100 idee per Porto Marghera}
100 idee per Venezia is the condensed effort of a participatory process that involved about 250 citizens of Venice since 2023, published as a book by La Toletta edizioni in 2024. The projects presented concern the entire city of Venice, and some sections are focused on the issues of regeneration and requalification of the area of Porto Marghera. The volume contains ideas formed after workshops, talks and seminars, and it could serve as a starting point for understanding the needs of the population.


\subsection{The Marghera Gamble}
The Marghera Gamble is a project by architects Martina Bertani and Charles André, exploring future scenarios for Porto Marghera. The discussion brought together voices such as: Gabriella Chiellino (environmental entrepreneur), Cristiana Colli (journalist and curator), Massimiliano De Martin (Venice Deputy Mayor for Environment and Urban Planning), Stefano Zeli of Terrapreta (collective specialising in soil regeneration), and Jane da Mosto for WahV, sharing perspectives on sustainable territorial development based on regenerating the lagoon system and associated values for society. The debate took place on the 20th of September 2025 within the Biennale Architettura, in the ENS public programme: Relational landscape: development without growth in Venice.

\subsection{Ecogiustizia subito}
Ecogiustizia Subito is a campaign promoted by  ACLI, AGESCI, ARCI, Azione Cattolica Italiana, Legambiente e Libera. The associations underline the need for funds for infrastructure for safety and environmental remediation for sites such as Nuovo Petrolchimico and Fusina; the need for a higher quantity of scientific and environmental analyses and for stricter controls on the management of times, procedures and right use of the resources for the environmental remediation; the need for the promotion of community participation for the inhabitants to define projects of economic and social redevelopment focusing on the industrial conversion of the SIN and on the creation of new job opportunities with green economy.

\section{Final evaluation}
Porto Marghera presents four main regulatory plans that affect the management of its territory: the Piano Regolatore Portuale per il Porto di Venezia (1965), the Variante al Piano Regolatore Generale per la zona di Porto Marghera (1999), the Masterplan per la bonifica dei siti inquinati (2004) and the Piano di Assetto del Territorio (2014), each, as previously covered, in both their assets and liabilities.

The general evaluation of the contribution of said plans to the territory of Porto Marghera, in both its port area and its residential one, and to the activities carried out on it, those being of commerce, of environmental regeneration or of production, cannot be said to be overwhelmingly positive.

It would be unfair to say that progress hasn’t been made: a new Port Regulatory Plan is currently in the works, to better serve a port area that has most certainly changed since 1965; projects like the Parco Scientifico Tecnologico VeGa, the Parco San Giuliano and the SIMAGE system have been successfully implemented and are an important part of the life of the area; remediation activities have been activated in 90\% of the targeted area; the Venice Heritage Tower was created.

However, for an area that has been the direct focus of such numerous debates, and that has continuously been signalled as an area of crisis from the economic, environmental, industrial points of view, as well as from the quality of life of its inhabitants.

Especially analysing the state of the environmental regeneration that should be carried out, only the 14\% of the industrial district has been actually regenerated, as of 2016. Not only this is a problem as it allows pollution to persist, but it also further prevents any other changes that could take place, as they would need remediation first. This is also aggravated by the fact that, due to the lack of participation, the needs of the population are often neglected, both on the long and short term, and most citizens are often uninformed of the condition of the industrial fabric.

Furthermore, it is important to underline that Porto Marghera was never completely given up on during the last decades, as the projects carried out testify. As much as scattered, one must recognize that innovative solutions are present: the system of recollection of wastewater PIF was completed, the urban-industrial symbiosis network was built and is operative, the green hydrogen valley is underway and the debate on Porto Marghera never extinguished. Nevertheless, this scattering slows the general renovation process and results in a physical oxymoron: polluted and dismissed areas are placed right next to “wannabe” centres of innovation, that most of the times seem misplaced and end up being under-achieving (VeGa \cite{altrochemestre} and the Venice Tower). 

In conclusion it can be argued that the plans do exist, however much outdated, and innovative projects are carried out, but this doesn’t seem to be sufficient for the needs of the territory. 