% ---- Sezione 1------
\section{Territorial area} % okay just missing images
\subsection{Topography}
\begin{minipage}{0.37\textwidth}
Porto Marghera is located within a 40 km wide and 120 km long area enclosed by the Alps, with peaks rising over 3000 meters to the northwest, North, and East, and bordered by the Adriatic Sea to the South. It is connected to the Padana Plain (the economic heart of Italy) only from the West. Within this region, the industrial hub of Porto Marghera lies on the inland shore of the Venetian Lagoon, between the lagoon and the mainland. It is positioned 5 km NW of the historical center of Venice, between the urban areas of Marghera and Mestre. It spans an area of 2000 ha representing, in terms of extension and importance, one of the principal italian industrial sites, centrally located within the Venetian Lagoon, one of UNESCO's world heritage sites \cite{unesco}. 
\end{minipage}
\hfill
\begin{minipage}{0.55\textwidth}
    \centering
    \includegraphics[width=\linewidth]{images/01_intro.png}
    \captionof{figure}{Topographical map of the Veneto region highlighting the location of Porto Marghera.}
    \label{fig:intro}
\end{minipage}
\vspace{2pt}

The industrial zone, entirely man-made, is completely flat. Landfills can reach 3–4 meters in height at central points, which is the highest elevation in the area. The zone is crisscrossed by canals, forming several artificial islands that are part of the industrial complex.

\subsection{Geology and hydrography}
Porto Marghera is a coastal area that consists of part mainland and a set of artificial islands. This area is part of the greater area of the Venice lagoon, a peculiar area in the Adriatic sea that is 50 km long an 11 km large, therefore it is the largest lagoon in Italy \cite{waterlands}. 

The contribution of sediments from the rivers (in order of influence: the Adige, the Sile, and the Brenta) is of fundamental importance for the life and morphology of the lagoon: the original river mouths (which are now the present-day inlets) had accumulated sandy deposits that, after being invaded by the sea, became coastal barriers. These are the three strips of land that currently separate the lagoon from the sea: \textit{Sant’Erasmo, Vignole, Cavallino, Lido, Pellestrina}, and \textit{Sottomarina}. Other islands were formed by the accumulation of sediments at the river mouths, which gradually receded inland; the islands of Venice, San Giorgio, Mazzorbo, Burano, and Torcello are remnants of ancient river formations \cite{hydrogeology}. According to \textit{Istituto Superiore per la Protezione e Ricerca Ambientale}, the mainland where Porto Marghera sits is made of diluvial deposits, whereas the majority of the islands of the lagoon are formed of calcareous silt and fine sands (see image \ref{fig:geology} for reference)

The lagoon is divided into three basins, each with different characteristics and separated by watershed zones. Each basin is served by a port inlet through which the water that flows in during the rising tide is the same that flows out through the same inlet during the ebb tide (Figure \ref{fig:inlets}). 
\begin{itemize}
	\item \textbf{The northern lagoon} (connected to the sea through the Lido inlet, which accounts for $40\%$  of total water exchange) most closely preserves the lagoon’s original character, that is, its relationship with the rivers. 
	\item \textbf{The central part} is served by the Malamocco inlet and is the most polluted due to higher human presence (Marghera and Venice). 
	\item Finally, \textbf{the southern lagoon}, served by the Chioggia inlet, is characterized by strong hydrodynamics, resulting from the presence of numerous navigable channels.
\end{itemize}
The entire drainage basin extends over more than 1.800 km$^2$ across the provinces of Venice $(52\%)$, Padua $(40\%)$, and Treviso $(8\%)$, and directly involves the municipalities of Jesolo, Musile di Piave, Quarto d’Altino, Venice, Mira, Campagna Lupia, Codevigo, Chioggia, and Cavallino-Treporti \cite{veneziavela}.

\begin{figure}[h]
    \centering
    \includegraphics[width=1.1\textwidth]{images/01_geology.png}
    \caption{Geological map from Istituto Superiore per la Protezione e la Ricerca Ambientale, 1:100.000.}
    \label{fig:geology}
\end{figure}

\begin{figure}[h]
    \centering
    \includegraphics[width=0.8\textwidth]{images/01_inlets.jpg}
    \caption{Map of the main inlets of the Venetian Lagoon, Ufficio Idrografico del Magistrato delle Acque di Venezia, 1:75000}
    \label{fig:inlets}
\end{figure}

\begin{figure}[H]
    \centering
    \includegraphics[width=0.75\textwidth]{images/01_hydrography.png}
    \caption{Map of rivers flowing out in the lagoon. Map created by the authors with QGIS.}
    \label{fig:inlets}
\end{figure}

\subsection{Soil}
Venice lagoon consists mainly of alluvial and marine sediments (layers of clay, silt, and sand deposited over centuries by the Po, Adige, Brenta, and Sile rivers). Therefore, the soil of this area is actually a soft, water-saturated sediment, which has very low bearing capacity \cite{hydrogeology}. However, in the industrial area of Porto Marghera, the currently existing land is completely of artificial origin: the islands of the area were created using soil removed from the canal construction and in some cases mixed with industrial waste coming from the existing firms \cite{grilliVenezia}. The creation of these new artificial  islands destroyed the former present marshes and tidal flats.

Instead, in the urban fabric, concrete, asphalt, and artificial pavements dominate, leaving very little permeable surface. This has led to limited soil biodiversity, restricted infiltration, and increased runoff, influencing both hydrology and pollution dispersion in the lagoon system.

% ---- Sezione 2------
\newpage
\section{Environmental characteristics} % okay just missing images and table translation
\subsection{Climate}
Porto Marghera, located on the mainland edge of the Venice Lagoon, experiences a humid subtropical climate, typical of the northern Adriatic region. The climate is mild and humid, influenced by the proximity of the sea and the extensive lagoon system, which moderates temperature extremes throughout the year.
The winter period is relatively short and mild, with average January temperatures around 3–4 °C, and only occasional frosts. Snowfall is rare and generally light, melting within a few hours or days. Fog is frequent in the colder months, especially in the early morning, often covering the industrial and lagoon areas with a dense haze \cite{atlanteclimatico}.

The summer season is warm, long, and humid. In July and August, daily maximum temperatures average around 28 °C, and heat waves can occasionally push values above 33–34 °C. The combination of heat and humidity often produces a feeling of heaviness. For reference, when the temperature reaches 33 °C and the relative humidity is about 70 $\%$, the heat index indicates potentially dangerous conditions for outdoor activity. Thunderstorms are common in late summer, bringing short but intense rainfall.

Over the course of the year, precipitation totals about 800 mm, distributed fairly evenly, with slightly wetter periods in spring and autumn (see table 1.1). Heavy rains can cause temporary flooding in low-lying areas of Marghera and Mestre, a recurring issue in the lagoon environment. The relative humidity averages around 75 $\%$, and the prevailing winds come from the North-East (Bora) in winter and the South-East (Scirocco) in autumn. These winds have a strong influence on local weather conditions and on the hydrodynamics of the lagoon, sometimes contributing to episodes of \textit{acqua alta} (high water).
The annual average temperature in Porto Marghera is approximately 13,8 °C, and the frost-free period extends for more than eight months of the year. Despite the moderate climate, the interaction between humidity, air stagnation, and industrial emissions often affects local air quality, particularly during summer heat episodes \cite{atlanteclimatico}.

In summary, the climate of Porto Marghera is characterized by mild winters, hot and humid summers, and frequent rainfall, a pattern that reflects its transitional position between the continental plains of the Veneto and the maritime influence of the Adriatic.
\begin{table}[htbp]
\centering
\renewcommand{\arraystretch}{1.25}
\setlength{\tabcolsep}{3.5pt}
\small

\resizebox{\textwidth}{!}{%
\begin{tabular}{c
                c c c c c
                c c c
                c c c c c}
\rowcolor{brandColor}
\color{white}\textbf{Month} &
\multicolumn{5}{c}{\color{white}\textbf{Monthly temperatures (°C)}} &
\multicolumn{3}{c}{\color{white}\textbf{Rainfall}} &
\multicolumn{5}{c}{\color{white}\textbf{Wind}} \\

\rowcolor{brandColor}
\color{white} &
\color{white}\textbf{Mean min} &
\color{white}\textbf{Min extr} &
\color{white}\textbf{Mean max} &
\color{white}\textbf{Max extr} &
\color{white}\textbf{Mean} &
\color{white}\textbf{Month} &
\color{white}\textbf{mm} &
\color{white}\textbf{Raindays} &
\color{white}\textbf{Month} &
\color{white}\textbf{Main dir} &
\color{white}\textbf{II dir} &
\color{white}\textbf{V mean} &
\color{white}\textbf{V max} \\

\midrule
1  & 1,1 & -5  & 6,4 & 11,5 & 3,7  & 1  & 59 & 7 & 1  & N  & NE & 3 & 4 \\
2  & 2,4 & -3  & 8,5 & 14,2 & 5,4  & 2  & 46 & 6 & 2  & N  & E  & 3 & 3 \\
3  & 5,9 & 0   & 12,1& 18,7 & 9,0  & 3  & 61 & 7 & 3  & E  & N  & 3 & 3 \\
4  & 9,7 & 4,7 & 16,2& 22,4 & 12,9 & 4  & 64 & 8 & 4  & E  & S  & 3 & 4 \\
5  & 13,9& 9   & 20,6& 25,8 & 17,3 & 5  & 73 & 9 & 5  & E  & S  & 3 & 4 \\
6  & 17,8& 12,5& 24,7& 30,0 & 21,3 & 6  & 70 & 7 & 6  & S  & E  & 3 & 3 \\
7  & 20,3& 15,2& 27,8& 32,5 & 24,0 & 7  & 53 & 5 & 7  & SE & S  & 3 & 3 \\
8  & 20,1& 15,2& 27,5& 32,0 & 23,8 & 8  & 76 & 6 & 8  & SE & S  & 3 & 3 \\
9  & 16,5& 11  & 23,8& 28,6 & 20,2 & 9  & 62 & 6 & 9  & S  & E  & 3 & 3 \\
10 & 11,3& 5,6 & 18,3& 23,7 & 14,8 & 10 & 67 & 6 & 10 & N  & E  & 3 & 3 \\
11 & 6,2 & 0,7 & 11,7& 17,5 & 9,0  & 11 & 79 & 8 & 11 & N  & E  & 3 & 4 \\
12 & 1,9 & -3  & 7,2 & 12,0 & 4,6  & 12 & 61 & 7 & 12 & N  & NE & 3 & 4 \\

\midrule
\textbf{Year} &
\textbf{10,6} & \textbf{-5} &
\textbf{17,1} & \textbf{32,5} &
\textbf{13,8} &
\multicolumn{2}{c}{\textbf{771}} &
\textbf{82} &
\multicolumn{3}{c}{} &
\textbf{3} &
\textbf{4} \\

\bottomrule
\end{tabular}%
}

\caption{Climate data from the municipality of Venice. Source: Archivio climatico ENEA DBT}
\label{tab:clima-venezia}
\end{table}


\subsection{Ecosystems and protected areas}
The whole Venetian lagoon is considered a protected site since 2007 under the Birds Directive (2009/147/EC) and the Habitats Directive (92/43/CEE), thus forming part of the EU protected sites system Natura 2000. It is one of the most significant transitional environments in the Mediterranean sea, including:
\begin{itemize}
	\item salt marshes: natural filters for pollutants, habitat of migratory birds
	\item mudflats: exposed only at low tides, sustain benthic communities and shorebirds
	\item coastal barriers: regulate the circulation of water and protect the lagoon from the open sea
\end{itemize}
The construction of the industrial port drastically altered the natural lagoon ecosystem, replacing wetlands and salt marshes with artificial landfills and canals. However, remnants of the original lagoon environment persist in peripheral zones and have been incorporated into regional conservation strategies. The most relevant example of a strictly protected area inside the Venetian lagoon is the WWF Oasis "Valle Averto", a wetland of international importance under the Ramsar Convention, located in the southern part of the Venice Lagoon.

\begin{minipage}{0.48\textwidth}
\centering
\includegraphics[width=\textwidth]{images/01_mapnatura2000_2.jpeg}
\captionof{figure}{Map of the Sites of Common Interest in Venice. Map created by the authors. Source: Natura2000}
\end{minipage}
\hspace{25pt}
\begin{minipage}{0.48\textwidth}
\centering
\includegraphics[width=\textwidth]{images/01_valleaverto.jpeg}
\captionof{figure}{Map of the perimeter of the WWF Valle Averto. Map created by the authors using QGIS.}
\end{minipage}

% ---- Sezione 3 ------
\section{Socio-economic characteristics} % okay just missing images

\subsection{Demography}
The number of people living in the entire municipality of Venice is on a medium negative rate since various years (Figure 1.8): in 1997, the population accounted for almost 294.000 inhabitants, whereas at the beginning of 2025 this number turned into approximately 252.000. However, even though a decrease of forty thousand people is seen on the whole territory, looking only at Marghera, the population changes from 29.120 in 1997 to 28,181 in 2025 (Figure 1.7). This shows that Marghera's population remained almost stable, since it kept its a local importance as a pole for occupation. 

The big shift on the general scale is ascribable to the increasing number of people moving from Venice to the mainland due to the rising prices of housing, the inconveniences caused by the increasing number of tourists and the emigration of firms to larger areas in the mainland.

\begin{minipage}{0.49\textwidth}
\centering
\includegraphics[width=\textwidth]{images/01_demography.png}
\captionof{figure}{Source: Comune di Venezia - Servizio Servizio Elettorale e Leva Militare, Statistica su dati di Anagrafe Comunale}
\end{minipage}
\hspace{10pt}
\begin{minipage}{0.49\textwidth}
\centering
\includegraphics[width=\textwidth]{images/01_demographyVen.png}
\captionof{figure}{Source: Comune di Venezia - Servizio Servizio Elettorale e Leva Militare, Statistica su dati di Anagrafe Comunale}
\end{minipage}

\subsection{Occupation}
Employment patterns in the Venice metropolitan area have undergone a profound transformation. During the 20th century, Porto Marghera functioned as the industrial core of the region, employing tens of thousands of workers in petrochemical, metallurgical, and shipbuilding industries. Since the 1970s, however, deindustrialization and automation have drastically reduced industrial employment, leading to a shift toward port logistics, environmental remediation, and service sectors (Figure 1.9).
Nowadays, 78,4 \% of the people living in the Municipality of Venice work in the tertiary sector, leaving just 21,6 \% to work in the industrial sector. Looking at the subdivisions of the latter, the majority is employed in the transportation and storage sector, as the main firms currently operating in Porto Marghera belong to. See figure 1.10 for the complete data. 

\begin{minipage}{0.43\textwidth}
\centering
\includegraphics[width=\textwidth]{images/01_sectors.png}
\captionof{figure}{Sectors division in job occupation. Source: Istat Data 2022 - settori economici (Ateco 3 cifre)}
\end{minipage}
\hspace{20pt}
\begin{minipage}{0.5\textwidth}
\centering
\includegraphics[width=\textwidth]{images/01_jobs.png}
\captionof{figure}{Detail of the job occupation data in the industrial sector. Source: Istat Data 2022 - settori economici (Ateco 3 cifre)}
\end{minipage}

\subsection{Tourism}
Tourism represents the leading economic activity in the municipality of Venice, generating a major share of local income and employment. This sector encompasses accommodation, hospitality, cultural services, and transport, forming a complex network of touristic operators. While the historic center of Venice concentrates most cultural and hospitality activities, the mainland districts of Mestre and Marghera have become increasingly integrated into the tourism economy, offering accommodation, parking, and transport infrastructure for mass tourism at affordable prices. More specifically, Porto Marghera has seen the effects of tourism increase with the construction of the Terminal Fusina, a touristic port located in the southern part of the industrial area that offers parking and direct connection through ferries to the historical centre of Venice.

% ---- Sezione 4 ------
\section{Territorial and urban characteristics}

\subsection{Landcover and landtake}
\subsubsection{Landcover}
The study area is characterized by a high degree of artificial land occupation: the territory welcomes a part of urban fabric (Mestre, Marghera and Venice islands) and part of industrial fabric, in the core of Porto Marghera. Land utilization in the latter is dominated by port, logistics, and chemical-industrial functions. In the mainland, outside the patchworked urban fabric, land is used for cultivation. Small parts are destined to public parks. See figure \ref{fig:landcover} for reference.

\begin{figure}[h]
    \centering
    \includegraphics[width=0.96\textwidth]{images/01_landcover.png}
    \caption{Corine Landcover of Porto Marghera in 2018, extracted from Copernicus.}
    \label{fig:landcover}
\end{figure}

\subsubsection{Landtake}
The areas near Porto Marghera witnessed an increase in landtake, defined as natural land that gets covered in concrete or becomes part of constructions. The new constructions form part of three separate classes: road, industrial or commercial  construction.  Commercial areas, like the big shopping center "Nave de Vero", were built under the pressure of the popularity of department stores and fast fashion firms. 
Some constructions are also due to touristic reasons caused by the always increasing fame of the area, both for Venice and for the sea (Figure \ref{fig:camping}).

\begin{figure}[H]
    \centering
    \includegraphics[width=0.94\textwidth]{images/01_camping.png}
    \caption{Example of landtake increase: expansion of the area of a camping resort near Marghera. The image on the left refers to 11/27/2000, image on the right refers to 31/7/2025. Images by Google Earth historical satellite photos.}
    \label{fig:camping}
\end{figure}

\begin{figure}[h]
 \centering
  \includegraphics[width=\textwidth]{images/01_landtake.jpeg}
    \caption{Map of the landtake increase between the years 2010 and 2025. The map was created by the authors using QGIS software and by comparing images taken by Google Earth historical satellite photos.}
\end{figure}

\subsection{Housing}
When Porto Marghera was built (starting in 1917), there was no coordinated housing plan for workers. As a result, most workers became commuters, living in the rural hinterland (Mestre, Mira, Dolo) and biking long distances to work \cite{zazzara}. Around 1920, the construction of a residential area was decided: the plan evolved around the \textit{garden city} model, first created by  Ebenezer Howard, in which houses had to be salubrious and characterized by big green spaces. A large tree-lined avenue, 80 meters wide and 700 meters long, was planned, together with side streets with flowerbeds separating the street from the sidewalk. Each building could not be closer than 15 meters from neighboring buildings and could not exceed three stories. All public services, such as schools, municipal offices, theaters, hospitals, libraries, and markets, were to be located in the center of the neighborhood, thus fostering the aggregation of a solid community \cite{cittagiardino}. 

Unfortunately, construction works were delayed from the beginning and the municipality of Venice stopped in financing the site. This resulted in a delayed end of works and an increase in the price of houses: only the heads of the existing firms could afford houses in \textit{città giardino}.  
The post-war housing policies transformed Marghera from an industrial suburb into a working-class district: large social housing estates, such as Villaggio San Marco and Catene, provided accommodation for industrial workers, yet often lacked adequate infrastructure and services. \\
Nowadays, housing in Marghera sees a majority of flat complexes and some remaining single houses. The existing road map of Marghera still shows the garden city plan (Figure 1.14). 

\begin{minipage}{\textwidth}
\centering
\includegraphics[width=\textwidth]{images/01_housingnow.jpeg}
\captionof{figure}{On the left: Map of Marghera urban fabric in 2025. 1:10.000. Map created by the authors using QGIS. On the right: Pietro Emilio Emmer (1920). Piano regolatore del centro urbano di Marghera. Extracted by \cite{cittagiardino}.}
\end{minipage}

\subsection{Infrastructure}
Porto Marghera operates as part of the Port System Authority of the Northern Adriatic Sea, which also includes the terminals of Venice and Chioggia. The port handles both commercial cargo and industrial shipments, serving petrochemical, metallurgical, and energy sectors. Its docks are connected to the Vittorio Emanuele Canal and to the “oil tanker channel”, which together link the industrial area to the maritime station and to the wider Adriatic Sea. The oil tanker canal was mostly dug in a straight line to allow oil tankers and other large ships to reach the port of Marghera through the mouth of Malamocco. These canals were dug in the last century, before understanding its damages to life in the lagoon \cite{shipeffects}. The port infrastructure includes approximately 30 km of quays and over 130 hectares of operational space dedicated to loading, storage, and logistics.

A dense railway network ensures the movement of goods within the industrial zone and provides direct connections to the national rail system through Mestre. Two main industrial lines (the Linea Petroli and Linea Enichem) connect the main production sites and terminals to the freight network. The port’s rail freight terminal, modernized in 2012 with European co-financing under the TEN-T program, facilitates the transfer of containers and bulk materials \cite{access2napa}.

The road system connects Porto Marghera to the A57 motorway (Tangenziale di Mestre), the A4 (Turin–Trieste), and the regional road SR11, providing access to Venice, Padua, and Treviso. Heavy vehicle traffic is concentrated along dedicated industrial corridors to minimize interference with residential areas. The main road that separates the industrial fabric to the urban fabric is Via Fratelli Bandiera, located on the West side of Porto Marghera. Access to the biggest industrial island is allowed by Via Alessandro Volta, which continues on the Ponte Strallato di Porto Marghera \cite{strade}.

Passenger mobility relies on a combination of urban bus lines, tram connections between Marghera, Mestre, and Venice, and ferry services operating across the lagoon. The nearest airport, Venice Marco Polo International Airport, is located about 15 km away and provides national and international connections.
Hence, although constrained by its lagoonal setting, Porto Marghera remains a strategic multimodal node within the Padua–Treviso–Venice (PATREVE) metropolitan area, integrating maritime, rail, and road transport networks to support industrial activity and urban mobility.

\begin{minipage}{\textwidth}
\centering
\includegraphics[width=\textwidth]{images/01_infrastructure.png}
\captionof{figure}{Map of the infrastructure of the area of Porto Marghera. Map created by the authors.}
\end{minipage}

% ---- Sezione 5 ------
\newpage
\section{Socio-political context}
\subsection{Institutional and socio-political context}
\subsubsection{The Municipality of Venice and the Italian government}
Porto Marghera, being a part of Marghera, one of the municipalities in the city of Venice, and being an industrial area undergoing numerous changes regarding urban planning and environmental reconversion, deals with a great variety of actors, both social and institutional.
 
The Municipality of Venice is the local government that Porto Marghera refers to, and it therefore holds the responsibility for everything that regards the projects and programs regarding the area, its urban planning and land use, and also its social ramifications.

The Veneto region, where Venice, and subsequently Marghera and Porto Marghera, are located, holds its own policies, which are present in Porto Marghera especially regarding the economy and the environment. The Region is also responsible for coordinating specific plans for the area, such as the \textit{Nuovo Patto per Marghera} (New Pact for Marghera) for its regeneration and development.

The \textit{Ente della Zona Industriale di Porto Marghera} (Authority for the Industrial Zone of Porto Marghera) is an association that represents all of the industrial actors in Porto Marghera, and its role is to manage the industrial area with its technical services and infrastructure and to monitor its environmental state and quality to develop it in a sustainable manner – this also carried out by the ESG (Environmental, Social, Governance) policies applied.

The \textit{Autorità di Sistema Portuale del Mare Adriatico Settentrionale} (Port Authority of the Northern Adriatic Sea) is the authority that coordinates, plans, controls and promotes the activities of the ports of Venice and Chioggia, with their logistics, maritime traffic and infrastructure. It is therefore actively involved in the projects that regard the port area of Porto Marghera.
More generally, the Italian Central Government and its Ministries have an impact on Porto Marghera through national laws and policies that regard industrial development and the environment, as well as funding.

\begin{minipage}{\textwidth}
\centering
\includegraphics[width=0.9\textwidth]{images/01_municipalities.png}
\captionof{figure}{Map of the municipalities inside the Comune di Venezia. Source: Comune di Venezia}
\end{minipage}

\subsubsection{The stakeholders}
\begin{minipage}{0.55\textwidth}
The industrial nature of the area attracts a lot of public and private stakeholders - numerous private companies that work in sectors such as energy and large manufacturing (Versalis, Siad Macchine Impianti, Fincantieri), as well as logistics operators and, more recently, investors in the sector of sustainability and industrial conversion \cite{adspmas}.
\textit{Piccole e medie imprese} (PMI) and local businesses are often part of the production chains, but also in new and autonomous activities, especially regarding green technology and the circular economy.
Innovation and development are sectors where the presence of universities and research centres is felt, with the realization of projects such as the VeGa Science and Technology Park (further explained in paragraph 2.4.2). The presence of many industrial activities makes Porto Marghera an area with active trade unions, in order to provide workers with a safe environment and rights protection, and to also take part in the plans and conversations on environmental issues. Citizen associations of volunteers and local NGOs, such as Legambiente, serve the similar purpose of managing social and environmental issues and being an active presence in support of the community.
\end{minipage}
\hspace{10pt}
\begin{minipage}{0.42\textwidth}
\centering
\includegraphics[width=\linewidth]{images/01_stakeholders}
\captionof{figure}{Stakeholders inside the Port area. Source: \cite{adspmas}}
\end{minipage}


\subsubsection{The inhabitants and their activities}
The vulnerability of Porto Marghera as an area concerns not only its environmental aspects, but also the life of its inhabitants. The awareness of such fragility has led citizens and associations to work together on projects and community-based actions.
The welfare of the inhabitants is one of the key problems that are being tackled, acting in aid of struggling families, providing education by collaborating on projects like “Costruire una comunità signific-attiva”, promoted by the Volontari del Fanciullo OdV association in 2021 \cite{inhabitants}.
Still regarding education and youth engagement, programs such as “Futuro Prossimo, Patto per la Comunità Educante di Mestre e Marghera” (Next Future, Path for the Educational Community of Mestre and Marghera) is also worth noting, it being a manner to contrast educational poverty and social exclusion \cite{bambini}.

Immigration and the integration of immigrants in the local community is fundamental to maintain cohesion between citizens, as immigrants are about one third of the total population in Porto Marghera. The ACLI provinciali (Christian Associations for Italian Workers) of the city of Venice have promoted the 2020 SQUERI project, offering Italian language courses for the A0, A1 and A2 levels and digital citizenship courses \cite{inhabitants}.

For the environment and its protections being one of the main issues of Porto Marghera, associations have been active in the protection of the area and of its inhabitants from pollution, with solutions such as the Patto di Comunità (Community Pact) promoted by ACLI, Legambiente, Azione Cattolica and AGESCI, as a part of the Ecogiustizia campaign, to stress the importance of recovery and decontamination of the area \cite{ecogiustizia}.

\subsection{Legislative context} 
The area of Porto Marghera is characterized by a wide legislative framework, due to its very particular nature of being part of the city of Venice and being a highly industrialized area undergoing changes and evolution in the sustainability department, as well as a residential one.
\subsubsection{Laws regarding the environment}
From the environmental point of view, the first law that needs to be taken into account is the Legge 426/1998 defined the Siti di Interesse Nazionale (SIN), or Sites of National Interests, areas of the country that are considered particularly polluted and in need of recovery. Porto Marghera, with pollution problems regarding air, water and soil, is listed as one of the SIN.
The D. Lgs. 152/2006 (Legislative Decree), Italy’s Environmental Code, sets the legal guidelines for the actions for the decontamination of sites, as well as for the monitoring of polluted air, water and soil. The Ministry of Environment also defines plans and actors for the decontamination plans.
Regarding industrial emission with the aim to prevent and reduce pollution, Italy, and consequently Porto Marghera, refers to the Industrial Emissions Directive of the European Union (IED 2010/75/EU).

\begin{figure}[H]
    \centering
    \includegraphics[width=\textwidth]{images/01_SINperimeter.jpeg}
    \caption{Perimeter of the Site of National Interest Venezia-Porto Marghera. Map created by the authors. Shape file source: Ministry of Environment and Energetic Safety}
    \label{fig:perimeter}
\end{figure}

\subsubsection{Laws regarding the industrial area}
The industrial area of Porto Marghera is undergoing processes of economic development and urban transition, and is regarded as a ZLS or Zona Logistica Semplificata (Simplified Logistics Zone) as per D.L. 91/2017, decree that was initially created to enhance the development of ports in the South of the country and that was then further expanded to the North. The areas defined by the decree benefit from tax benefits, administrative simplification and fast-track investment processes.
Porto Marghera is also defined as an area of complex industrial crisis, and the Decreto MISE 9 June 2015 gives the framework for the project implemented in such zones.
The Legge 181/1989 makes the area eligible for financial incentives for business investments, as it is an area of industrial crisis.

\subsubsection{Laws regarding urban and territorial planning}
As for the urban and territorial planning, Porto Marghera is subject to the Legge Regionale Veneto n. 11/2004, which concerns regional urban planning, and to the Accordi di Programma (Program Agreements), agreements between Ministries, Regions, Municipalities and industries, in order to coordinate their actions.
There are also specific local development plans – Piano Particolareggiato dell’area di Marghera – adopted by the Municipality of Venice for regeneration of disused industrial areas and land use.

\subsubsection{Laws regarding the port}
The actual port of Porto Marghera is part of the Autorità del Sistema Portuale del Mare Adriatico Settentrionale, whose authority has been assigned by the Legge 84/1994, a reform of Italy’s ports.
The area is also subject to the Seveso Directive (D. Lgs. 105/2015) and to other permits regarding the safety of the workers and of the environment regulated by national bodies.

\subsubsection{Specific laws for Porto Marghera}
Porto Marghera also has policies and agreements that regard its specific area.

The Accordo di \textit{Programma per Porto Marghera}, first presented in 2015 and updated since, is an agreement between the Ministry of the Economic Development, the Ministry of Environment, the Veneto Region, the Municipality of Venice, the Porto Authority and other actors, and it main aims regard sustainable growth and development, environmental cleanup and investment attraction.

The \textit{Protocollo per le Linee Strategiche di Porto Marghera} (Protocol for the Strategic Lines of Porto Marghera), presented in 2021, concerns the reducing of environmental impacts through green and industrial transition, as well as an enhancement of social development. It is signed by the Veneto Region, the Municipality of Venice, trade unions and stakeholders.

\subsubsection{Specific laws for the Veneto Region}
The Veneto Region is also part of the Covenant of Mayors, introduced by the European Commission in the year 2008, with the aim to define the goals for climate and energy that the members of the European Union would have had to reach by the year 2020, and was then renewed in 2015 with new objectives for 2030.
The three main goals are  the reduction of 55\% of greenhouse gases emissions by 2030,  the strengthening of resilience and  the reduction of energetic poverty. The local governments therefore have to present an Action Plan for Sustainable Energy and Climate by the second year since adhering to the Covenant, with the actions that they are willing to partake in. 

\subsubsection{Specific laws for the city of Venice}
The city of Venice is under specific laws and regulations that apply to it, being a one of a kind case in the country of Italy and in the whole world.

With the Legge 16 aprile 1973 n. 71 the protection of Venice becomes an issue of national interest, with the State, the Region and the local authorities actively participating in the activities with such aim. Further improvement have been made with the Legge 29 novembre 1984, n. 798 (aimed at the architectural, urban, environmental and economic recovery of Venice between the years 1984 and 1986) and with the Legge 8 novembre 1991, n. 360. In said context, the Veneto Region had proposed the "\textit{Piano per la prevenzione dell’inquinamento e il risanamento delle acque del bacino idrografico immediatamente sversante nella laguna di Venezia}”, with the aim to expand the actions of prevention and decontamination of all the sources of pollution (civil, industrial, agricultural and zootechnical) and inside of the Drainage Basin of the Venice Lagoon. Said plan was updated in the year 2000, with the birth of the \textit{Piano Direttore 2000}, that established the most optimal strategies of decontamination for the waters coming into the Lagoon and for the Lagoon itself. The main guidelines of the Piano regard the prevention, reduction, auto-decontamination and, as a last resort, diversion.
The sectors of application of the Legge Speciale are the sewage and depuration, aqueducts, the Drainage Basin territory, agriculture and zootechnics, decontamination of contaminated areas, environmental monitoring and experimentation. Four provinces of the Veneto Region, including Venice (the others being Treviso, Padova and Vicenza) are a part of the territory of the Drainage Basin as defined by the D.C.R. n. 23 del 7 maggio 2003.

With the Legge Regionale n. 12 del 27 maggio 2024, the Veneto Region defined the regional management of the strategic environment evaluation, the environmental impact assessment, the environmental effects assessment and the integrated environmental authorization.

\begin{figure}[H]
    \centering
    \includegraphics[width=0.9\textwidth]{images/01_bacinoscolante.jpg}
    \caption{Perimeter the Drainage Basin inside the Venetian Lagoon. Source: \cite{moseecosistema}}
    \label{fig:bacino}
\end{figure}

% -- Sezione 6 --
\newpage
\section{Issues} 
\subsection{Pollution and its impact}
\begin{minipage}{0.6\textwidth}
The most prominent issue that the area of Porto Marghera has to face is the extremely high quantity of pollutants that can be found in its water, soil and sediments, and its devastating consequences for both the environment and the inhabitants \cite{grilliVenezia}.

Water pollution is widespread and varies according to the location of the different industries: in the oil area, PCBs are the main pollutants, with a higher concentration in the waste water than in the water table; the area of the Chemical Peninsula has a high concentration of polycyclic aromatic hydrocarbons and chlorinated organic compounds in the waste water, and chlorobenzenes in the water table.
Heavy metals, such as As and Pb, but also PCBs and polycyclic aromatic hydrocarbons are the main pollutants of the sediments of the water canals, in a much higher concentration than the one registered in the rest of the Venice lagoon.
As far as the soil is concerned, the level of pollution of 85$\%$ of the area of Porto Marghera is considered higher than the limit defined by national law.
\end{minipage}
\hspace{20pt}
\begin{minipage}{0.32\textwidth}
\centering
\includegraphics[width=\textwidth]{images/01_pollutantsphoto.png}
\captionof{figure}{Photo of the diffusion of pollutants, visible on the coast. Extracted by \cite{mag_acque}}
\end{minipage}

Such an alarming situation is the result of the lack of control of emissions of pollutants into soil and groundwater, of the air pollution produced by the industries over the years and by the use of production waste for coastal advance.

The impacts of this situation have been assessed by the SENTIERI (Studio Epidemiologico Nazionale dei Territori e degli Insediamenti Esposti a Rischio da Inquinamento) study on Porto Marghera and other SINs, and have been published in 2011.
The exposure of the population to these substances causes higher than usual rates of mortality for diseases such as lung cancer and pleural mesothelioma for men and asthma, cardiovascular diseases, chronic lung diseases and ovarian cancer for women.

\subsection{Slow bureaucracy in the remediation process}
Given the conditions of the pollution in the area of Marghera, the remediation of contaminated soil and groundwater is required. However, the expected process of decontamination is formed of 4 steps that found some difficulties in its application and have been stalled for years: 
\begin{enumerate}[leftmargin=*,noitemsep]
\item Soil and groundwater sampling: detailed soil sampling to determine the concentrations and types of pollutants introduced.
\item Emergency Safety Measures (MISE): comprehensive containment of the area using embankments and sheet piling to prevent contaminants from leaching into the water.
\item Remediation: a complex of techniques to extract pollutants (usually time and cost demanding). In some cases this step might be downgraded to simple soil containment.
\item Land reallocation: establishing the precise use of the decontaminated land.
\end{enumerate}

The soil containment system consists of anti-erosion physical barriers that have a wide range of forms (figure \ref{fig:sezioneA} shows an example of the simplest one). More specifically, the construction works include first, sheet piling (a physical barrier composed of interlocking steel sheet piles), then a groundwater and storm-water collection and drainage system and finally a conveyance system to channel the water into the Progetto Integrato Fusina (see paragraph 2.5.1) , where the actual water treatment process takes place \cite{mag_acque}. 

\begin{figure}[H]
\centering
\includegraphics[width=0.95\textwidth]{images/01_primadopo.png}
\caption{Photos of before and after the soil containment works. Picture extracted from \cite{mag_acque}.}
\label{fig:primadopo}
\end{figure}
The difficulties of application of the works of soil containment not only stemmed from the complexity of the work itself but also from the fact that its realization was assigned to multiple entities, these being the Port Authority, the existing firms, the Provveditorato and the government. This division provoked inequality in the progress of the construction works and a general difficulty in having clear information about the work evolution. In 2015, according to the report on the works assigned to the Magistrato delle Acque (Provveditorato), the construction of the soil containment was at 94\%, with the most problematic parts missing \cite{mag_acque}. Nevertheless, this amount accounted only for the Provveditorato's part of the works. Soil containment works on private industrial land is still in progress \cite{monitoraggiocivico}.

\begin{figure}[H]
\centering
\includegraphics[width=\textwidth]{images/01_marginamentimap.png}
\caption{Map of the soil containment works. Created by Magistrato delle Acque. Extracted from \cite{mag_acque}.}
\label{fig:marginamenti}
\end{figure}

Concerning the actual remediation, the SIN system provides for the following \textit{iter}. After the quantification of pollution through sampling, the area is labeled as contaminated if C>CSR (i.e. contamination exceeds the contamination limit). In case of contamination, a remediation project must be presented, approved and finally it can be carried out.

The last update from the Ministry of the Environment and Energetic Safety about soil and groundwater decontamination is presented in figure \ref{fig:SINfalda} and \ref{fig:SINterreni}. 
It shows that the majority of the territory has an approved remediation project but not carried out, while the completely remediated land and groundwater still constitutes a minor part. 

\begin{figure}[H]
\centering
\includegraphics[width=0.8\textwidth]{images/01_sezioneA.png}
\caption{Example of a section of soil containment technique.}
\label{fig:sezioneA}
\end{figure}
\section{Existing debates}
\subsection{Soil decontamination}

In January 2025, a group of citizens of the municipality of Venice protested to denounce that  the environmental cleanup in the National Interest Site (SIN) of Venice–Porto Marghera has been stalled for years due to bureaucratic delays and uncertain funding \cite{ecogiustizia}. With the protest, citizens stated a list of main requests:
\begin{itemize}[leftmargin=*,noitemsep]
\item Immediate funding for urgent safety and cleanup works 
\item Simplify bureaucratic procedures that delay cleanup on private lands
\item Strengthen environmental monitoring by ARPAV
\end{itemize}
\subsection{Social inequality}
Porto Marghera’s position with respect to the city of Venice, its lack of connections and its history, profoundly impacted by the industrial crisis and the subsequent lack of employment make difficult living conditions for its inhabitants.
While the young demographic suffers from the lack of jobs, finding them in the tourism sector when possible, the immigrant component of the population struggles to fit in.
The population has manifested discontent through protests for years, lastly for the sudden closing of the Altuglas factory.

\begin{figure}[h]
\centering
\includegraphics[width=0.92\textwidth]{images/01_SINfalda.pdf}
\caption{}
\label{fig:SINfalda}
\end{figure}

\begin{figure}[!ht]
\centering
\includegraphics[width=0.92\textwidth]{images/01_SINterreni.pdf}
\caption{}
\label{fig:SINterreni}
\end{figure}
\clearpage