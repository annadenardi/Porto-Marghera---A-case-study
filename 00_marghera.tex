\documentclass[
  11pt,
  a4paper,
  twoside,
  openright,
  DIV=12,
  numbers=noenddot
]{scrreprt}
\raggedbottom

% --- 1. IMPOSTAZIONI BASE ---
\usepackage[utf8]{inputenc}
\usepackage[T1]{fontenc}
\usepackage[english]{babel}
\usepackage{booktabs}
\usepackage{multirow}
\usepackage{colortbl}
\usepackage{xcolor}
\usepackage{array}
\usepackage{sidecap}
\usepackage{enumitem}
\usepackage{float}
\usepackage{url}

% --- 2. IMMAGINI ---
\usepackage{graphicx} 

% --- 3. COLORI & GRAFICA ---
\usepackage{xcolor}
\usepackage{tikz} 

% --- 4. FONT & TYPO ---
\usepackage[scaled]{helvet}
\renewcommand{\familydefault}{\sfdefault} 
\usepackage{microtype}      
\usepackage[parfill]{parskip} % Spazio tra paragrafi, no rientro

% --- 5. DEFINIZIONE COLORI ---
\definecolor{brandColor}{RGB}{230, 80, 10} % Arancione
\definecolor{darkGrey}{RGB}{40, 40, 40}

% --- 6. GEOMETRIA PAGINA ---
\usepackage[
    top=3cm, 
    bottom=3cm, 
    inner=3cm, 
    outer=2cm, 
    bindingoffset=0.5cm, 
    headheight=15pt
]{geometry}

% --- 7. HEADER E FOOTER ---
\usepackage{scrlayer-scrpage}
\clearpairofpagestyles

% Linea colorata
\newcommand{\headerline}{\color{brandColor}\rule{\textwidth}{1.5mm}}

% Header Pari (Sinistra) e Dispari (Destra)
\lehead[\vspace{0.5em}\headerline]{\textcolor{darkGrey}{\normalfont\small\thepage}\hspace{1em}\textcolor{gray}{\headmark}\\\headerline}
\rohead[\vspace{0.5em}\headerline]{\textcolor{gray}{\headmark}\hspace{1em}\textcolor{darkGrey}{\normalfont\small\thepage}\\\headerline}

% --- 8. STILE TITOLI (CORRETTO) ---
% Qui c'era l'errore: bisogna aggiungere [explicit] per usare #1
\usepackage[explicit]{titlesec}

% Comando per creare il box colorato
\newcommand{\colorchapter}[1]{%
    \tikz[baseline=(current bounding box.east)] {
        \node[
            rectangle, 
            fill=brandColor, 
            text=white, 
            inner sep=15pt, 
            text width=\dimexpr\textwidth-30pt\relax, 
            align=flush right,
            font=\huge\bfseries
        ] {#1};
    }
}

% Stile Capitolo
\titleformat{\chapter}[display]
  {\normalfont\sffamily}
  {} 
  {0pt}
  {\colorchapter{\thechapter \quad #1}} % Ora #1 funziona grazie a [explicit]
  []

% Stile Sezione (Con [explicit] dobbiamo rimettere #1 alla fine)
% --- section ---
\titleformat{\section}
  {\Large\bfseries\color{darkGrey}}
  {\textcolor{brandColor}{\vrule width 5pt}\hspace{0.5em}\thesection}
  {0.5em}
  {#1}

% --- subsection---
\titleformat{\subsection}
  {\large\bfseries\color{darkGrey}}
  {\textcolor{brandColor}{\vrule width 3pt}\hspace{0.5em}\thesubsection}
  {0.5em}
  {#1}

% --- subsubsection ---
\titleformat{\subsubsection}
  {\normalsize\bfseries\color{darkGrey}}
  {\textcolor{brandColor}{\vrule width 1.5pt}\hspace{0.5em}\thesubsubsection}
  {0.5em}
  {#1}

% --- 9. DIDASCALIE E LINK ---
\usepackage{caption}
\captionsetup{font={small,sf}, labelfont={bf, color=brandColor}}
\usepackage[hidelinks]{hyperref}
\KOMAoption{bibliography}{totocnumbered}

% ==========================================
\begin{document}

% --- FRONTESPIZIO ---
\begin{titlepage}
    \pagecolor{darkGrey} 
    \color{white}        
    \newgeometry{margin=2cm}
    
    \vspace*{1cm}
    
    % --- RIGA LOGO E TESTO ---
    \noindent
    \begin{minipage}[b]{0.7\textwidth} % Testo a sinistra
        \Large Sapienza Università di Roma \\
        \textbf{Exam}: Sustainable Development and Planning
    \end{minipage}
    \begin{minipage}[b]{0.3\textwidth} % Logo a destra
        \raggedleft
        \includegraphics[width=5cm]{images/00_logo.png} % Sostituisci con il nome del file
    \end{minipage}
    
    \vspace{4cm}
    
    % --- TITOLO (Box Arancione) ---
    \noindent\colorbox{brandColor}{
        \parbox{\dimexpr\textwidth-2\fboxsep}{
            \vspace{0.5cm}
            \centering\Huge\bfseries
            PORTO MARGHERA: A CASE STUDY
            \vspace{0.5cm}
        }
    }
    
    \vspace{0.5cm}
    \noindent{\huge Final report}

    % --- NUOVA IMMAGINE ALLINEATA A DESTRA ---
    \vspace{1cm} % Spazio dopo "Final report"
    \begin{flushright}
        \includegraphics[width=0.56\textwidth]{images/00_copertina.png} % Regola width per la dimensione
    \end{flushright}
    
    \vfill
    
    \noindent\textbf{Students:} Clément Cazajous, Martina Cumo, Anna De Nardi \\
    \textbf{Professor:} Carlo Cellamare \\
    \textbf{Year:} 2025/2026
    
\end{titlepage}
\restoregeometry
\nopagecolor 
\color{black} 

% --- INDICE ---
\tableofcontents

% --- CAPITOLO 1 ---
\chapter{Contextualization}
% ---- Sezione 1------
\section{Territorial area} % okay just missing images
\subsection{Topography}
\begin{minipage}{0.37\textwidth}
Porto Marghera is located within a 40 km wide and 120 km long area enclosed by the Alps, with peaks rising over 3000 meters to the northwest, North, and East, and bordered by the Adriatic Sea to the South. It is connected to the Padana Plain (the economic heart of Italy) only from the West. Within this region, the industrial hub of Porto Marghera lies on the inland shore of the Venetian Lagoon, between the lagoon and the mainland. It is positioned 5 km NW of the historical center of Venice, between the urban areas of Marghera and Mestre. It spans an area of 2000 ha representing, in terms of extension and importance, one of the principal italian industrial sites, centrally located within the Venetian Lagoon, one of UNESCO's world heritage sites \cite{unesco}. 
\end{minipage}
\hfill
\begin{minipage}{0.55\textwidth}
    \centering
    \includegraphics[width=\linewidth]{images/01_intro.png}
    \captionof{figure}{Topographical map of the Veneto region highlighting the location of Porto Marghera.}
    \label{fig:intro}
\end{minipage}
\vspace{2pt}

The industrial zone, entirely man-made, is completely flat. Landfills can reach 3–4 meters in height at central points, which is the highest elevation in the area. The zone is crisscrossed by canals, forming several artificial islands that are part of the industrial complex.

\subsection{Geology and hydrography}
Porto Marghera is a coastal area that consists of part mainland and a set of artificial islands. This area is part of the greater area of the Venice lagoon, a peculiar area in the Adriatic sea that is 50 km long an 11 km large, therefore it is the largest lagoon in Italy \cite{waterlands}. 

The contribution of sediments from the rivers (in order of influence: the Adige, the Sile, and the Brenta) is of fundamental importance for the life and morphology of the lagoon: the original river mouths (which are now the present-day inlets) had accumulated sandy deposits that, after being invaded by the sea, became coastal barriers. These are the three strips of land that currently separate the lagoon from the sea: \textit{Sant’Erasmo, Vignole, Cavallino, Lido, Pellestrina}, and \textit{Sottomarina}. Other islands were formed by the accumulation of sediments at the river mouths, which gradually receded inland; the islands of Venice, San Giorgio, Mazzorbo, Burano, and Torcello are remnants of ancient river formations \cite{hydrogeology}. According to \textit{Istituto Superiore per la Protezione e Ricerca Ambientale}, the mainland where Porto Marghera sits is made of diluvial deposits, whereas the majority of the islands of the lagoon are formed of calcareous silt and fine sands (see image \ref{fig:geology} for reference)

The lagoon is divided into three basins, each with different characteristics and separated by watershed zones. Each basin is served by a port inlet through which the water that flows in during the rising tide is the same that flows out through the same inlet during the ebb tide (Figure \ref{fig:inlets}). 
\begin{itemize}
	\item \textbf{The northern lagoon} (connected to the sea through the Lido inlet, which accounts for $40\%$  of total water exchange) most closely preserves the lagoon’s original character, that is, its relationship with the rivers. 
	\item \textbf{The central part} is served by the Malamocco inlet and is the most polluted due to higher human presence (Marghera and Venice). 
	\item Finally, \textbf{the southern lagoon}, served by the Chioggia inlet, is characterized by strong hydrodynamics, resulting from the presence of numerous navigable channels.
\end{itemize}
The entire drainage basin extends over more than 1.800 km$^2$ across the provinces of Venice $(52\%)$, Padua $(40\%)$, and Treviso $(8\%)$, and directly involves the municipalities of Jesolo, Musile di Piave, Quarto d’Altino, Venice, Mira, Campagna Lupia, Codevigo, Chioggia, and Cavallino-Treporti \cite{veneziavela}.

\begin{figure}[h]
    \centering
    \includegraphics[width=1.1\textwidth]{images/01_geology.png}
    \caption{Geological map from Istituto Superiore per la Protezione e la Ricerca Ambientale, 1:100.000.}
    \label{fig:geology}
\end{figure}

\begin{figure}[h]
    \centering
    \includegraphics[width=0.8\textwidth]{images/01_inlets.jpg}
    \caption{Map of the main inlets of the Venetian Lagoon, Ufficio Idrografico del Magistrato delle Acque di Venezia, 1:75000}
    \label{fig:inlets}
\end{figure}

\begin{figure}[H]
    \centering
    \includegraphics[width=0.75\textwidth]{images/01_hydrography.png}
    \caption{Map of rivers flowing out in the lagoon. Map created by the authors with QGIS.}
    \label{fig:inlets}
\end{figure}

\subsection{Soil}
Venice lagoon consists mainly of alluvial and marine sediments (layers of clay, silt, and sand deposited over centuries by the Po, Adige, Brenta, and Sile rivers). Therefore, the soil of this area is actually a soft, water-saturated sediment, which has very low bearing capacity \cite{hydrogeology}. However, in the industrial area of Porto Marghera, the currently existing land is completely of artificial origin: the islands of the area were created using soil removed from the canal construction and in some cases mixed with industrial waste coming from the existing firms \cite{grilliVenezia}. The creation of these new artificial  islands destroyed the former present marshes and tidal flats.

Instead, in the urban fabric, concrete, asphalt, and artificial pavements dominate, leaving very little permeable surface. This has led to limited soil biodiversity, restricted infiltration, and increased runoff, influencing both hydrology and pollution dispersion in the lagoon system.

% ---- Sezione 2------
\newpage
\section{Environmental characteristics} % okay just missing images and table translation
\subsection{Climate}
Porto Marghera, located on the mainland edge of the Venice Lagoon, experiences a humid subtropical climate, typical of the northern Adriatic region. The climate is mild and humid, influenced by the proximity of the sea and the extensive lagoon system, which moderates temperature extremes throughout the year.
The winter period is relatively short and mild, with average January temperatures around 3–4 °C, and only occasional frosts. Snowfall is rare and generally light, melting within a few hours or days. Fog is frequent in the colder months, especially in the early morning, often covering the industrial and lagoon areas with a dense haze \cite{atlanteclimatico}.

The summer season is warm, long, and humid. In July and August, daily maximum temperatures average around 28 °C, and heat waves can occasionally push values above 33–34 °C. The combination of heat and humidity often produces a feeling of heaviness. For reference, when the temperature reaches 33 °C and the relative humidity is about 70 $\%$, the heat index indicates potentially dangerous conditions for outdoor activity. Thunderstorms are common in late summer, bringing short but intense rainfall.

Over the course of the year, precipitation totals about 800 mm, distributed fairly evenly, with slightly wetter periods in spring and autumn (see table 1.1). Heavy rains can cause temporary flooding in low-lying areas of Marghera and Mestre, a recurring issue in the lagoon environment. The relative humidity averages around 75 $\%$, and the prevailing winds come from the North-East (Bora) in winter and the South-East (Scirocco) in autumn. These winds have a strong influence on local weather conditions and on the hydrodynamics of the lagoon, sometimes contributing to episodes of \textit{acqua alta} (high water).
The annual average temperature in Porto Marghera is approximately 13,8 °C, and the frost-free period extends for more than eight months of the year. Despite the moderate climate, the interaction between humidity, air stagnation, and industrial emissions often affects local air quality, particularly during summer heat episodes \cite{atlanteclimatico}.

In summary, the climate of Porto Marghera is characterized by mild winters, hot and humid summers, and frequent rainfall, a pattern that reflects its transitional position between the continental plains of the Veneto and the maritime influence of the Adriatic.
\begin{table}[htbp]
\centering
\renewcommand{\arraystretch}{1.25}
\setlength{\tabcolsep}{3.5pt}
\small

\resizebox{\textwidth}{!}{%
\begin{tabular}{c
                c c c c c
                c c c
                c c c c c}
\rowcolor{brandColor}
\color{white}\textbf{Month} &
\multicolumn{5}{c}{\color{white}\textbf{Monthly temperatures (°C)}} &
\multicolumn{3}{c}{\color{white}\textbf{Rainfall}} &
\multicolumn{5}{c}{\color{white}\textbf{Wind}} \\

\rowcolor{brandColor}
\color{white} &
\color{white}\textbf{Mean min} &
\color{white}\textbf{Min extr} &
\color{white}\textbf{Mean max} &
\color{white}\textbf{Max extr} &
\color{white}\textbf{Mean} &
\color{white}\textbf{Month} &
\color{white}\textbf{mm} &
\color{white}\textbf{Raindays} &
\color{white}\textbf{Month} &
\color{white}\textbf{Main dir} &
\color{white}\textbf{II dir} &
\color{white}\textbf{V mean} &
\color{white}\textbf{V max} \\

\midrule
1  & 1,1 & -5  & 6,4 & 11,5 & 3,7  & 1  & 59 & 7 & 1  & N  & NE & 3 & 4 \\
2  & 2,4 & -3  & 8,5 & 14,2 & 5,4  & 2  & 46 & 6 & 2  & N  & E  & 3 & 3 \\
3  & 5,9 & 0   & 12,1& 18,7 & 9,0  & 3  & 61 & 7 & 3  & E  & N  & 3 & 3 \\
4  & 9,7 & 4,7 & 16,2& 22,4 & 12,9 & 4  & 64 & 8 & 4  & E  & S  & 3 & 4 \\
5  & 13,9& 9   & 20,6& 25,8 & 17,3 & 5  & 73 & 9 & 5  & E  & S  & 3 & 4 \\
6  & 17,8& 12,5& 24,7& 30,0 & 21,3 & 6  & 70 & 7 & 6  & S  & E  & 3 & 3 \\
7  & 20,3& 15,2& 27,8& 32,5 & 24,0 & 7  & 53 & 5 & 7  & SE & S  & 3 & 3 \\
8  & 20,1& 15,2& 27,5& 32,0 & 23,8 & 8  & 76 & 6 & 8  & SE & S  & 3 & 3 \\
9  & 16,5& 11  & 23,8& 28,6 & 20,2 & 9  & 62 & 6 & 9  & S  & E  & 3 & 3 \\
10 & 11,3& 5,6 & 18,3& 23,7 & 14,8 & 10 & 67 & 6 & 10 & N  & E  & 3 & 3 \\
11 & 6,2 & 0,7 & 11,7& 17,5 & 9,0  & 11 & 79 & 8 & 11 & N  & E  & 3 & 4 \\
12 & 1,9 & -3  & 7,2 & 12,0 & 4,6  & 12 & 61 & 7 & 12 & N  & NE & 3 & 4 \\

\midrule
\textbf{Year} &
\textbf{10,6} & \textbf{-5} &
\textbf{17,1} & \textbf{32,5} &
\textbf{13,8} &
\multicolumn{2}{c}{\textbf{771}} &
\textbf{82} &
\multicolumn{3}{c}{} &
\textbf{3} &
\textbf{4} \\

\bottomrule
\end{tabular}%
}

\caption{Climate data from the municipality of Venice. Source: Archivio climatico ENEA DBT}
\label{tab:clima-venezia}
\end{table}


\subsection{Ecosystems and protected areas}
The whole Venetian lagoon is considered a protected site since 2007 under the Birds Directive (2009/147/EC) and the Habitats Directive (92/43/CEE), thus forming part of the EU protected sites system Natura 2000. It is one of the most significant transitional environments in the Mediterranean sea, including:
\begin{itemize}
	\item salt marshes: natural filters for pollutants, habitat of migratory birds
	\item mudflats: exposed only at low tides, sustain benthic communities and shorebirds
	\item coastal barriers: regulate the circulation of water and protect the lagoon from the open sea
\end{itemize}
The construction of the industrial port drastically altered the natural lagoon ecosystem, replacing wetlands and salt marshes with artificial landfills and canals. However, remnants of the original lagoon environment persist in peripheral zones and have been incorporated into regional conservation strategies. The most relevant example of a strictly protected area inside the Venetian lagoon is the WWF Oasis "Valle Averto", a wetland of international importance under the Ramsar Convention, located in the southern part of the Venice Lagoon.

\begin{minipage}{0.48\textwidth}
\centering
\includegraphics[width=\textwidth]{images/01_mapnatura2000_2.jpeg}
\captionof{figure}{Map of the Sites of Common Interest in Venice. Map created by the authors. Source: Natura2000}
\end{minipage}
\hspace{25pt}
\begin{minipage}{0.48\textwidth}
\centering
\includegraphics[width=\textwidth]{images/01_valleaverto.jpeg}
\captionof{figure}{Map of the perimeter of the WWF Valle Averto. Map created by the authors using QGIS.}
\end{minipage}

% ---- Sezione 3 ------
\section{Socio-economic characteristics} % okay just missing images

\subsection{Demography}
The number of people living in the entire municipality of Venice is on a medium negative rate since various years (Figure 1.8): in 1997, the population accounted for almost 294.000 inhabitants, whereas at the beginning of 2025 this number turned into approximately 252.000. However, even though a decrease of forty thousand people is seen on the whole territory, looking only at Marghera, the population changes from 29.120 in 1997 to 28,181 in 2025 (Figure 1.7). This shows that Marghera's population remained almost stable, since it kept its a local importance as a pole for occupation. 

The big shift on the general scale is ascribable to the increasing number of people moving from Venice to the mainland due to the rising prices of housing, the inconveniences caused by the increasing number of tourists and the emigration of firms to larger areas in the mainland.

\begin{minipage}{0.49\textwidth}
\centering
\includegraphics[width=\textwidth]{images/01_demography.png}
\captionof{figure}{Source: Comune di Venezia - Servizio Servizio Elettorale e Leva Militare, Statistica su dati di Anagrafe Comunale}
\end{minipage}
\hspace{10pt}
\begin{minipage}{0.49\textwidth}
\centering
\includegraphics[width=\textwidth]{images/01_demographyVen.png}
\captionof{figure}{Source: Comune di Venezia - Servizio Servizio Elettorale e Leva Militare, Statistica su dati di Anagrafe Comunale}
\end{minipage}

\subsection{Occupation}
Employment patterns in the Venice metropolitan area have undergone a profound transformation. During the 20th century, Porto Marghera functioned as the industrial core of the region, employing tens of thousands of workers in petrochemical, metallurgical, and shipbuilding industries. Since the 1970s, however, deindustrialization and automation have drastically reduced industrial employment, leading to a shift toward port logistics, environmental remediation, and service sectors (Figure 1.9).
Nowadays, 78,4 \% of the people living in the Municipality of Venice work in the tertiary sector, leaving just 21,6 \% to work in the industrial sector. Looking at the subdivisions of the latter, the majority is employed in the transportation and storage sector, as the main firms currently operating in Porto Marghera belong to. See figure 1.10 for the complete data. 

\begin{minipage}{0.43\textwidth}
\centering
\includegraphics[width=\textwidth]{images/01_sectors.png}
\captionof{figure}{Sectors division in job occupation. Source: Istat Data 2022 - settori economici (Ateco 3 cifre)}
\end{minipage}
\hspace{20pt}
\begin{minipage}{0.5\textwidth}
\centering
\includegraphics[width=\textwidth]{images/01_jobs.png}
\captionof{figure}{Detail of the job occupation data in the industrial sector. Source: Istat Data 2022 - settori economici (Ateco 3 cifre)}
\end{minipage}

\subsection{Tourism}
Tourism represents the leading economic activity in the municipality of Venice, generating a major share of local income and employment. This sector encompasses accommodation, hospitality, cultural services, and transport, forming a complex network of touristic operators. While the historic center of Venice concentrates most cultural and hospitality activities, the mainland districts of Mestre and Marghera have become increasingly integrated into the tourism economy, offering accommodation, parking, and transport infrastructure for mass tourism at affordable prices. More specifically, Porto Marghera has seen the effects of tourism increase with the construction of the Terminal Fusina, a touristic port located in the southern part of the industrial area that offers parking and direct connection through ferries to the historical centre of Venice.

% ---- Sezione 4 ------
\section{Territorial and urban characteristics}

\subsection{Landcover and landtake}
\subsubsection{Landcover}
The study area is characterized by a high degree of artificial land occupation: the territory welcomes a part of urban fabric (Mestre, Marghera and Venice islands) and part of industrial fabric, in the core of Porto Marghera. Land utilization in the latter is dominated by port, logistics, and chemical-industrial functions. In the mainland, outside the patchworked urban fabric, land is used for cultivation. Small parts are destined to public parks. See figure \ref{fig:landcover} for reference.

\begin{figure}[h]
    \centering
    \includegraphics[width=0.96\textwidth]{images/01_landcover.png}
    \caption{Corine Landcover of Porto Marghera in 2018, extracted from Copernicus.}
    \label{fig:landcover}
\end{figure}

\subsubsection{Landtake}
The areas near Porto Marghera witnessed an increase in landtake, defined as natural land that gets covered in concrete or becomes part of constructions. The new constructions form part of three separate classes: road, industrial or commercial  construction.  Commercial areas, like the big shopping center "Nave de Vero", were built under the pressure of the popularity of department stores and fast fashion firms. 
Some constructions are also due to touristic reasons caused by the always increasing fame of the area, both for Venice and for the sea (Figure \ref{fig:camping}).

\begin{figure}[H]
    \centering
    \includegraphics[width=0.94\textwidth]{images/01_camping.png}
    \caption{Example of landtake increase: expansion of the area of a camping resort near Marghera. The image on the left refers to 11/27/2000, image on the right refers to 31/7/2025. Images by Google Earth historical satellite photos.}
    \label{fig:camping}
\end{figure}

\begin{figure}[h]
 \centering
  \includegraphics[width=\textwidth]{images/01_landtake.jpeg}
    \caption{Map of the landtake increase between the years 2010 and 2025. The map was created by the authors using QGIS software and by comparing images taken by Google Earth historical satellite photos.}
\end{figure}

\subsection{Housing}
When Porto Marghera was built (starting in 1917), there was no coordinated housing plan for workers. As a result, most workers became commuters, living in the rural hinterland (Mestre, Mira, Dolo) and biking long distances to work \cite{zazzara}. Around 1920, the construction of a residential area was decided: the plan evolved around the \textit{garden city} model, first created by  Ebenezer Howard, in which houses had to be salubrious and characterized by big green spaces. A large tree-lined avenue, 80 meters wide and 700 meters long, was planned, together with side streets with flowerbeds separating the street from the sidewalk. Each building could not be closer than 15 meters from neighboring buildings and could not exceed three stories. All public services, such as schools, municipal offices, theaters, hospitals, libraries, and markets, were to be located in the center of the neighborhood, thus fostering the aggregation of a solid community \cite{cittagiardino}. 

Unfortunately, construction works were delayed from the beginning and the municipality of Venice stopped in financing the site. This resulted in a delayed end of works and an increase in the price of houses: only the heads of the existing firms could afford houses in \textit{città giardino}.  
The post-war housing policies transformed Marghera from an industrial suburb into a working-class district: large social housing estates, such as Villaggio San Marco and Catene, provided accommodation for industrial workers, yet often lacked adequate infrastructure and services. \\
Nowadays, housing in Marghera sees a majority of flat complexes and some remaining single houses. The existing road map of Marghera still shows the garden city plan (Figure 1.14). 

\begin{minipage}{\textwidth}
\centering
\includegraphics[width=\textwidth]{images/01_housingnow.jpeg}
\captionof{figure}{On the left: Map of Marghera urban fabric in 2025. 1:10.000. Map created by the authors using QGIS. On the right: Pietro Emilio Emmer (1920). Piano regolatore del centro urbano di Marghera. Extracted by \cite{cittagiardino}.}
\end{minipage}

\subsection{Infrastructure}
Porto Marghera operates as part of the Port System Authority of the Northern Adriatic Sea, which also includes the terminals of Venice and Chioggia. The port handles both commercial cargo and industrial shipments, serving petrochemical, metallurgical, and energy sectors. Its docks are connected to the Vittorio Emanuele Canal and to the “oil tanker channel”, which together link the industrial area to the maritime station and to the wider Adriatic Sea. The oil tanker canal was mostly dug in a straight line to allow oil tankers and other large ships to reach the port of Marghera through the mouth of Malamocco. These canals were dug in the last century, before understanding its damages to life in the lagoon \cite{shipeffects}. The port infrastructure includes approximately 30 km of quays and over 130 hectares of operational space dedicated to loading, storage, and logistics.

A dense railway network ensures the movement of goods within the industrial zone and provides direct connections to the national rail system through Mestre. Two main industrial lines (the Linea Petroli and Linea Enichem) connect the main production sites and terminals to the freight network. The port’s rail freight terminal, modernized in 2012 with European co-financing under the TEN-T program, facilitates the transfer of containers and bulk materials \cite{access2napa}.

The road system connects Porto Marghera to the A57 motorway (Tangenziale di Mestre), the A4 (Turin–Trieste), and the regional road SR11, providing access to Venice, Padua, and Treviso. Heavy vehicle traffic is concentrated along dedicated industrial corridors to minimize interference with residential areas. The main road that separates the industrial fabric to the urban fabric is Via Fratelli Bandiera, located on the West side of Porto Marghera. Access to the biggest industrial island is allowed by Via Alessandro Volta, which continues on the Ponte Strallato di Porto Marghera \cite{strade}.

Passenger mobility relies on a combination of urban bus lines, tram connections between Marghera, Mestre, and Venice, and ferry services operating across the lagoon. The nearest airport, Venice Marco Polo International Airport, is located about 15 km away and provides national and international connections.
Hence, although constrained by its lagoonal setting, Porto Marghera remains a strategic multimodal node within the Padua–Treviso–Venice (PATREVE) metropolitan area, integrating maritime, rail, and road transport networks to support industrial activity and urban mobility.

\begin{minipage}{\textwidth}
\centering
\includegraphics[width=\textwidth]{images/01_infrastructure.png}
\captionof{figure}{Map of the infrastructure of the area of Porto Marghera. Map created by the authors.}
\end{minipage}

% ---- Sezione 5 ------
\newpage
\section{Socio-political context}
\subsection{Institutional and socio-political context}
\subsubsection{The Municipality of Venice and the Italian government}
Porto Marghera, being a part of Marghera, one of the municipalities in the city of Venice, and being an industrial area undergoing numerous changes regarding urban planning and environmental reconversion, deals with a great variety of actors, both social and institutional.
 
The Municipality of Venice is the local government that Porto Marghera refers to, and it therefore holds the responsibility for everything that regards the projects and programs regarding the area, its urban planning and land use, and also its social ramifications.

The Veneto region, where Venice, and subsequently Marghera and Porto Marghera, are located, holds its own policies, which are present in Porto Marghera especially regarding the economy and the environment. The Region is also responsible for coordinating specific plans for the area, such as the \textit{Nuovo Patto per Marghera} (New Pact for Marghera) for its regeneration and development.

The \textit{Ente della Zona Industriale di Porto Marghera} (Authority for the Industrial Zone of Porto Marghera) is an association that represents all of the industrial actors in Porto Marghera, and its role is to manage the industrial area with its technical services and infrastructure and to monitor its environmental state and quality to develop it in a sustainable manner – this also carried out by the ESG (Environmental, Social, Governance) policies applied.

The \textit{Autorità di Sistema Portuale del Mare Adriatico Settentrionale} (Port Authority of the Northern Adriatic Sea) is the authority that coordinates, plans, controls and promotes the activities of the ports of Venice and Chioggia, with their logistics, maritime traffic and infrastructure. It is therefore actively involved in the projects that regard the port area of Porto Marghera.
More generally, the Italian Central Government and its Ministries have an impact on Porto Marghera through national laws and policies that regard industrial development and the environment, as well as funding.

\begin{minipage}{\textwidth}
\centering
\includegraphics[width=0.9\textwidth]{images/01_municipalities.png}
\captionof{figure}{Map of the municipalities inside the Comune di Venezia. Source: Comune di Venezia}
\end{minipage}

\subsubsection{The stakeholders}
\begin{minipage}{0.55\textwidth}
The industrial nature of the area attracts a lot of public and private stakeholders - numerous private companies that work in sectors such as energy and large manufacturing (Versalis, Siad Macchine Impianti, Fincantieri), as well as logistics operators and, more recently, investors in the sector of sustainability and industrial conversion \cite{adspmas}.
\textit{Piccole e medie imprese} (PMI) and local businesses are often part of the production chains, but also in new and autonomous activities, especially regarding green technology and the circular economy.
Innovation and development are sectors where the presence of universities and research centres is felt, with the realization of projects such as the VeGa Science and Technology Park (further explained in paragraph 2.4.2). The presence of many industrial activities makes Porto Marghera an area with active trade unions, in order to provide workers with a safe environment and rights protection, and to also take part in the plans and conversations on environmental issues. Citizen associations of volunteers and local NGOs, such as Legambiente, serve the similar purpose of managing social and environmental issues and being an active presence in support of the community.
\end{minipage}
\hspace{10pt}
\begin{minipage}{0.42\textwidth}
\centering
\includegraphics[width=\linewidth]{images/01_stakeholders}
\captionof{figure}{Stakeholders inside the Port area. Source: \cite{adspmas}}
\end{minipage}


\subsubsection{The inhabitants and their activities}
The vulnerability of Porto Marghera as an area concerns not only its environmental aspects, but also the life of its inhabitants. The awareness of such fragility has led citizens and associations to work together on projects and community-based actions.
The welfare of the inhabitants is one of the key problems that are being tackled, acting in aid of struggling families, providing education by collaborating on projects like “Costruire una comunità signific-attiva”, promoted by the Volontari del Fanciullo OdV association in 2021 \cite{inhabitants}.
Still regarding education and youth engagement, programs such as “Futuro Prossimo, Patto per la Comunità Educante di Mestre e Marghera” (Next Future, Path for the Educational Community of Mestre and Marghera) is also worth noting, it being a manner to contrast educational poverty and social exclusion \cite{bambini}.

Immigration and the integration of immigrants in the local community is fundamental to maintain cohesion between citizens, as immigrants are about one third of the total population in Porto Marghera. The ACLI provinciali (Christian Associations for Italian Workers) of the city of Venice have promoted the 2020 SQUERI project, offering Italian language courses for the A0, A1 and A2 levels and digital citizenship courses \cite{inhabitants}.

For the environment and its protections being one of the main issues of Porto Marghera, associations have been active in the protection of the area and of its inhabitants from pollution, with solutions such as the Patto di Comunità (Community Pact) promoted by ACLI, Legambiente, Azione Cattolica and AGESCI, as a part of the Ecogiustizia campaign, to stress the importance of recovery and decontamination of the area \cite{ecogiustizia}.

\subsection{Legislative context} 
The area of Porto Marghera is characterized by a wide legislative framework, due to its very particular nature of being part of the city of Venice and being a highly industrialized area undergoing changes and evolution in the sustainability department, as well as a residential one.
\subsubsection{Laws regarding the environment}
From the environmental point of view, the first law that needs to be taken into account is the Legge 426/1998 defined the Siti di Interesse Nazionale (SIN), or Sites of National Interests, areas of the country that are considered particularly polluted and in need of recovery. Porto Marghera, with pollution problems regarding air, water and soil, is listed as one of the SIN.
The D. Lgs. 152/2006 (Legislative Decree), Italy’s Environmental Code, sets the legal guidelines for the actions for the decontamination of sites, as well as for the monitoring of polluted air, water and soil. The Ministry of Environment also defines plans and actors for the decontamination plans.
Regarding industrial emission with the aim to prevent and reduce pollution, Italy, and consequently Porto Marghera, refers to the Industrial Emissions Directive of the European Union (IED 2010/75/EU).

\begin{figure}[H]
    \centering
    \includegraphics[width=\textwidth]{images/01_SINperimeter.jpeg}
    \caption{Perimeter of the Site of National Interest Venezia-Porto Marghera. Map created by the authors. Shape file source: Ministry of Environment and Energetic Safety}
    \label{fig:perimeter}
\end{figure}

\subsubsection{Laws regarding the industrial area}
The industrial area of Porto Marghera is undergoing processes of economic development and urban transition, and is regarded as a ZLS or Zona Logistica Semplificata (Simplified Logistics Zone) as per D.L. 91/2017, decree that was initially created to enhance the development of ports in the South of the country and that was then further expanded to the North. The areas defined by the decree benefit from tax benefits, administrative simplification and fast-track investment processes.
Porto Marghera is also defined as an area of complex industrial crisis, and the Decreto MISE 9 June 2015 gives the framework for the project implemented in such zones.
The Legge 181/1989 makes the area eligible for financial incentives for business investments, as it is an area of industrial crisis.

\subsubsection{Laws regarding urban and territorial planning}
As for the urban and territorial planning, Porto Marghera is subject to the Legge Regionale Veneto n. 11/2004, which concerns regional urban planning, and to the Accordi di Programma (Program Agreements), agreements between Ministries, Regions, Municipalities and industries, in order to coordinate their actions.
There are also specific local development plans – Piano Particolareggiato dell’area di Marghera – adopted by the Municipality of Venice for regeneration of disused industrial areas and land use.

\subsubsection{Laws regarding the port}
The actual port of Porto Marghera is part of the Autorità del Sistema Portuale del Mare Adriatico Settentrionale, whose authority has been assigned by the Legge 84/1994, a reform of Italy’s ports.
The area is also subject to the Seveso Directive (D. Lgs. 105/2015) and to other permits regarding the safety of the workers and of the environment regulated by national bodies.

\subsubsection{Specific laws for Porto Marghera}
Porto Marghera also has policies and agreements that regard its specific area.

The Accordo di \textit{Programma per Porto Marghera}, first presented in 2015 and updated since, is an agreement between the Ministry of the Economic Development, the Ministry of Environment, the Veneto Region, the Municipality of Venice, the Porto Authority and other actors, and it main aims regard sustainable growth and development, environmental cleanup and investment attraction.

The \textit{Protocollo per le Linee Strategiche di Porto Marghera} (Protocol for the Strategic Lines of Porto Marghera), presented in 2021, concerns the reducing of environmental impacts through green and industrial transition, as well as an enhancement of social development. It is signed by the Veneto Region, the Municipality of Venice, trade unions and stakeholders.

\subsubsection{Specific laws for the Veneto Region}
The Veneto Region is also part of the Covenant of Mayors, introduced by the European Commission in the year 2008, with the aim to define the goals for climate and energy that the members of the European Union would have had to reach by the year 2020, and was then renewed in 2015 with new objectives for 2030.
The three main goals are  the reduction of 55\% of greenhouse gases emissions by 2030,  the strengthening of resilience and  the reduction of energetic poverty. The local governments therefore have to present an Action Plan for Sustainable Energy and Climate by the second year since adhering to the Covenant, with the actions that they are willing to partake in. 

\subsubsection{Specific laws for the city of Venice}
The city of Venice is under specific laws and regulations that apply to it, being a one of a kind case in the country of Italy and in the whole world.

With the Legge 16 aprile 1973 n. 71 the protection of Venice becomes an issue of national interest, with the State, the Region and the local authorities actively participating in the activities with such aim. Further improvement have been made with the Legge 29 novembre 1984, n. 798 (aimed at the architectural, urban, environmental and economic recovery of Venice between the years 1984 and 1986) and with the Legge 8 novembre 1991, n. 360. In said context, the Veneto Region had proposed the "\textit{Piano per la prevenzione dell’inquinamento e il risanamento delle acque del bacino idrografico immediatamente sversante nella laguna di Venezia}”, with the aim to expand the actions of prevention and decontamination of all the sources of pollution (civil, industrial, agricultural and zootechnical) and inside of the Drainage Basin of the Venice Lagoon. Said plan was updated in the year 2000, with the birth of the \textit{Piano Direttore 2000}, that established the most optimal strategies of decontamination for the waters coming into the Lagoon and for the Lagoon itself. The main guidelines of the Piano regard the prevention, reduction, auto-decontamination and, as a last resort, diversion.
The sectors of application of the Legge Speciale are the sewage and depuration, aqueducts, the Drainage Basin territory, agriculture and zootechnics, decontamination of contaminated areas, environmental monitoring and experimentation. Four provinces of the Veneto Region, including Venice (the others being Treviso, Padova and Vicenza) are a part of the territory of the Drainage Basin as defined by the D.C.R. n. 23 del 7 maggio 2003.

With the Legge Regionale n. 12 del 27 maggio 2024, the Veneto Region defined the regional management of the strategic environment evaluation, the environmental impact assessment, the environmental effects assessment and the integrated environmental authorization.

\begin{figure}[H]
    \centering
    \includegraphics[width=0.9\textwidth]{images/01_bacinoscolante.jpg}
    \caption{Perimeter the Drainage Basin inside the Venetian Lagoon. Source: \cite{moseecosistema}}
    \label{fig:bacino}
\end{figure}

% -- Sezione 6 --
\newpage
\section{Issues} 
\subsection{Pollution and its impact}
\begin{minipage}{0.6\textwidth}
The most prominent issue that the area of Porto Marghera has to face is the extremely high quantity of pollutants that can be found in its water, soil and sediments, and its devastating consequences for both the environment and the inhabitants \cite{grilliVenezia}.

Water pollution is widespread and varies according to the location of the different industries: in the oil area, PCBs are the main pollutants, with a higher concentration in the waste water than in the water table; the area of the Chemical Peninsula has a high concentration of polycyclic aromatic hydrocarbons and chlorinated organic compounds in the waste water, and chlorobenzenes in the water table.
Heavy metals, such as As and Pb, but also PCBs and polycyclic aromatic hydrocarbons are the main pollutants of the sediments of the water canals, in a much higher concentration than the one registered in the rest of the Venice lagoon.
As far as the soil is concerned, the level of pollution of 85$\%$ of the area of Porto Marghera is considered higher than the limit defined by national law.
\end{minipage}
\hspace{20pt}
\begin{minipage}{0.32\textwidth}
\centering
\includegraphics[width=\textwidth]{images/01_pollutantsphoto.png}
\captionof{figure}{Photo of the diffusion of pollutants, visible on the coast. Extracted by \cite{mag_acque}}
\end{minipage}

Such an alarming situation is the result of the lack of control of emissions of pollutants into soil and groundwater, of the air pollution produced by the industries over the years and by the use of production waste for coastal advance.

The impacts of this situation have been assessed by the SENTIERI (Studio Epidemiologico Nazionale dei Territori e degli Insediamenti Esposti a Rischio da Inquinamento) study on Porto Marghera and other SINs, and have been published in 2011.
The exposure of the population to these substances causes higher than usual rates of mortality for diseases such as lung cancer and pleural mesothelioma for men and asthma, cardiovascular diseases, chronic lung diseases and ovarian cancer for women.

\subsection{Slow bureaucracy in the remediation process}
Given the conditions of the pollution in the area of Marghera, the remediation of contaminated soil and groundwater is required. However, the expected process of decontamination is formed of 4 steps that found some difficulties in its application and have been stalled for years: 
\begin{enumerate}[leftmargin=*,noitemsep]
\item Soil and groundwater sampling: detailed soil sampling to determine the concentrations and types of pollutants introduced.
\item Emergency Safety Measures (MISE): comprehensive containment of the area using embankments and sheet piling to prevent contaminants from leaching into the water.
\item Remediation: a complex of techniques to extract pollutants (usually time and cost demanding). In some cases this step might be downgraded to simple soil containment.
\item Land reallocation: establishing the precise use of the decontaminated land.
\end{enumerate}

The soil containment system consists of anti-erosion physical barriers that have a wide range of forms (figure \ref{fig:sezioneA} shows an example of the simplest one). More specifically, the construction works include first, sheet piling (a physical barrier composed of interlocking steel sheet piles), then a groundwater and storm-water collection and drainage system and finally a conveyance system to channel the water into the Progetto Integrato Fusina (see paragraph 2.5.1) , where the actual water treatment process takes place \cite{mag_acque}. 

\begin{figure}[H]
\centering
\includegraphics[width=0.95\textwidth]{images/01_primadopo.png}
\caption{Photos of before and after the soil containment works. Picture extracted from \cite{mag_acque}.}
\label{fig:primadopo}
\end{figure}
The difficulties of application of the works of soil containment not only stemmed from the complexity of the work itself but also from the fact that its realization was assigned to multiple entities, these being the Port Authority, the existing firms, the Provveditorato and the government. This division provoked inequality in the progress of the construction works and a general difficulty in having clear information about the work evolution. In 2015, according to the report on the works assigned to the Magistrato delle Acque (Provveditorato), the construction of the soil containment was at 94\%, with the most problematic parts missing \cite{mag_acque}. Nevertheless, this amount accounted only for the Provveditorato's part of the works. Soil containment works on private industrial land is still in progress \cite{monitoraggiocivico}.

\begin{figure}[H]
\centering
\includegraphics[width=\textwidth]{images/01_marginamentimap.png}
\caption{Map of the soil containment works. Created by Magistrato delle Acque. Extracted from \cite{mag_acque}.}
\label{fig:marginamenti}
\end{figure}

Concerning the actual remediation, the SIN system provides for the following \textit{iter}. After the quantification of pollution through sampling, the area is labeled as contaminated if C>CSR (i.e. contamination exceeds the contamination limit). In case of contamination, a remediation project must be presented, approved and finally it can be carried out.

The last update from the Ministry of the Environment and Energetic Safety about soil and groundwater decontamination is presented in figure \ref{fig:SINfalda} and \ref{fig:SINterreni}. 
It shows that the majority of the territory has an approved remediation project but not carried out, while the completely remediated land and groundwater still constitutes a minor part. 

\begin{figure}[H]
\centering
\includegraphics[width=0.8\textwidth]{images/01_sezioneA.png}
\caption{Example of a section of soil containment technique.}
\label{fig:sezioneA}
\end{figure}
\section{Existing debates}
\subsection{Soil decontamination}

In January 2025, a group of citizens of the municipality of Venice protested to denounce that  the environmental cleanup in the National Interest Site (SIN) of Venice–Porto Marghera has been stalled for years due to bureaucratic delays and uncertain funding \cite{ecogiustizia}. With the protest, citizens stated a list of main requests:
\begin{itemize}[leftmargin=*,noitemsep]
\item Immediate funding for urgent safety and cleanup works 
\item Simplify bureaucratic procedures that delay cleanup on private lands
\item Strengthen environmental monitoring by ARPAV
\end{itemize}
\subsection{Social inequality}
Porto Marghera’s position with respect to the city of Venice, its lack of connections and its history, profoundly impacted by the industrial crisis and the subsequent lack of employment make difficult living conditions for its inhabitants.
While the young demographic suffers from the lack of jobs, finding them in the tourism sector when possible, the immigrant component of the population struggles to fit in.
The population has manifested discontent through protests for years, lastly for the sudden closing of the Altuglas factory.

\begin{figure}[h]
\centering
\includegraphics[width=0.92\textwidth]{images/01_SINfalda.pdf}
\caption{}
\label{fig:SINfalda}
\end{figure}

\begin{figure}[!ht]
\centering
\includegraphics[width=0.92\textwidth]{images/01_SINterreni.pdf}
\caption{}
\label{fig:SINterreni}
\end{figure}
\clearpage

% --- CAPITOLO 2 ---
\chapter{Legislative and Planning Framework}
\section{Historical notes on Porto Marghera}
At the beginning of the XX century the port of Venice, which at the time was located in the insular area, had reached the maximum level of traffic, showing its structural limitations, such as the absence of rail and road connections to the mainland and the lack of land which turned into a severe physical constraint to any possible expansion. Therefore, the development of a commercial port and an adjacent industrial area on the mainland appeared to be a profitable opportunity. In addition to this, it was also considered the availability of a labour force at low cost, as at that time the area was mostly rural and economically depressed. 

In 1917 the Municipality of Venice, the “Società Porto Industriale” created by the entrepreneur Giuseppe Volpi and the Italian Boselli government signed the “Decreto Luogotenenziale 26 luglio 1917, n. 1191” in order to build the new port. In 1919 the Vittorio Emanuele III Canal, connecting the Giudecca to Marghera, was dug, while the Sade company started building the first thermal power plant. In 1920, the Società Porto Industriale started building infrastructures for the industrial and commercial port, the rail and road links needed for transporting the goods and raw materials for the works. The construction started in 1919 and in 1928 fifty-eight firms were already settled in the first industrial zone of Porto Marghera. Porto Marghera, which was conceived and designed as a coastal industrial area right from the start, experienced a time of notable growth. In 1925 there were 33 companies and 3,440 workers \cite{zazzara}. 

In 1925 a new Port Master Plan was drawn up providing for the extension of industrial areas up to the Brenta canal (Fusina), securing enough space required for the enlargement of the area throughout the sixties. Starting from the '30s, the metallurgy and non-ferrous material industry (aluminium and its alloys, zinc) grew vigorously and a large plant producing synthetic ammonia for nitrogen fertilizers was created. The area flourished in a very short time: not even the II World War stopped its growth. Even though some plants were hit during aerial bombing, they were soon rebuilt after the post-war adjustment. 
In the early '50s, because of the saturation of the first industrial area with 128 companies and 22,500 workers, the project for a second industrial area was designed, establishing productions of petrochemicals, refractory materials, precision carpentry, power plants and food oil refineries. This second industrial area rose mainly on areas taken away from the Lagoon through fills or embankments of the ground level, using industrial waste and scraps from factories based in the first industrial area and materials from the excavation of canals \cite{interreg}.

During the '60s, the development of Porto Marghera and the growth of industrial trades required a new stage of interventions. Following the increase in industrial trades, the San Leonardo oil port was built in 1966 and, three years later, the excavation of the Malamocco-Marghera canal was completed, in order to allow all oil tankers to reach the port of San Leonardo and the industrial areas through the Malamocco port entrance, avoiding the San Marco basin and the city's historical centre. At this point, in 1965, Porto Marghera was at its peak of employment, with 33.000 workers in the area. Porto Marghera was one of the major industrial concentrations of that time and thus had a great number of working class employees that took part in the great conflicts between capital and labour of the late 60s \cite{zazzara}. 

Since the '70s Porto Marghera fell into a crisis, due to the increase in raw material price and the decline of the raw chemical industry. This obliged many firms to downsize and to lay many workers off. The crisis continued all along the '80s and the '90s, reaching less than half of workers compared to the peak year: in 2000, only 12.727 people were employed in Porto Marghera \cite{interreg}. It was at this time during the 90s that the concern for the economically and environmentally critical condition started to arise and a lot of legislation and agreements between the parts were made. A brief recap of its evolution is here summarized: 

\textbf{1994-1996}: The environmental remediation and industrial reconversion project for Porto Marghera found its first definition in the Variation to the General Regulatory Plan (PRG) for Porto Marghera. This tool, later discussed in this work, not only was a new and precise urban planning tool for the area, but also an important set of indications for implementing planned interventions. 

\textbf{1998-1999}: The progressive emergence of the environmental issue (confirmed site pollution, risk of pollutant spills in the lagoon, and the presence of industrial plants posing risks to urban centers near Porto Marghera) occurred alongside a deepening structural crisis in the industrial sectors characterizing the hub (particularly petrochemicals). This led to a preventive search for a balance between environmental protection and the continuation and development of high-risk production processes. In this period Porto Marghera was placed among the Siti di Interesse Nazionale (SIN - Sites of National interest) of Italy. 
The Program Agreement for Chemistry in Porto Marghera (formally a DPCM of February 1999) proposed to establish and maintain optimal conditions for the coexistence of environmental protection and the productive development of the chemical sector. It conditioned the continued presence of industrial activities on the implementation of interventions that guaranteed their sustainability.

\textbf{2001}: A supplement to the Agreement (DPCM 15.11.2001) subsequently defined criteria for harmonizing the approval procedures for investment projects presented by the signatory companies of the first agreement. In particular, this supplementary act provided for the drafting of a "Masterplan for Remediation": the goal was to guide the creation of projects consistent with an environmental redevelopment program for the entire area, ensuring coherence, timeliness, and solutions tailored to the specific characteristics of the sites. This Master Plan for Porto Marghera received final approval in 2004.

\textbf{2004-2005}: The revitalization of a debate about the future of Porto Marghera, which occurred simultaneously with the debate over the environmental sustainability of petrochemical activity, has meant that the general theme of the Porto Marghera project has been reviewed comprehensively from the recognition of its economic value in Venice and the Veneto region. The "Intesa per Porto Marghera" signed in late 2005 between the Region, local authorities, business associations, and trade unions, provided for the development of activities aimed at defining a "development pact" for Porto Marghera based on fundamental aspects of production analysis and development capacity (launch of an environmental redevelopment process for the area, prevention of industrial risk, redevelopment of the chemical hub, construction of infrastructure necessary to decongest municipal and provincial arteries).

\textbf{2006-2007}: On December 14, 2006, the State, the Veneto Region, the Province and Municipality of Venice, social partners, and companies signed a new Program Agreement for Chemistry in Porto Marghera, which aimed to maintain conditions of managerial certainty for companies operating there, combining these with environmental protection needs starting from the integrated petrochemical cycle. This led to the "Protocol for sharing strategic guidelines for the redevelopment and development of Porto Marghera," released on September 21, 2007.

\textbf{2011-2015}: In May, 2011, a DPCM by the Ministry of the economic development included Marghera inside the areas of industrial crisis. This later led to the signing of a new Program Agreement for the environmental remediation and redevelopment of Porto Marghera between the ministry, the Port Authority and the local institutions, with the aim of simplifying the procedures of remediation. Always in 2012, the Progetto Integrato Fusina (PIF) was approved: the construction of a multifunctional platform to treat contaminated groundwater resulting from the safety measures implemented in the area. In 2014 the Veneto Region and the Municipality of Venice bought 110 hectares of dismissed industrial areas (previously owned by Syndial) for re-industrialization. This was followed by a new Program Agreement in 2015 that allocated 152 million euros for Porto Marghera. 

\textbf{2018-present}:
In January 2018, a new agreement was signed to set up a five-year contract between local institutions for the management of the remediation process in the Porto Marghera SIN. This was followed by an agreement signed on 23rd October 2018 by the Ministry of Economic Development, the Region, and the Municipality to provide for investment in reconversion measures for an amount of 27 million euros. This achievement was also be confirmed with the approval in September 2019 of a three-year extension to the contract (i.e., extension of the deadline until 2022) for completion of the redevelopment interventions provided for in the latest program agreement.

\begin{figure}[h]
\centering
\includegraphics[width=\linewidth]{images/02_timeline.png}
\caption{Timeline of the most important plans and laws for Porto Marghera. Source: Peron I., Potenzialità contese. Porto Marghera, una questione di metodo, scheda 28}
\end{figure}
% ----------------------- sezione overview -----------------------------------------------
\section{The legislative framework for Porto Marghera}
In this section, an overview of the existing legislation acting on Porto Marghera is given. This area is characterized by a complex system of entities that act upon it. Considering a hierarchical structure from the farthest actor to the nearest to the area, Porto Marghera sees at the same time: 
\begin{itemize}[leftmargin=*]
\item The European Directives
\item The Italian laws (that receive and apply the European Directives)
\item The Region with its regional plans and laws
\item The Port Authority, that is a separate entity from the Region (therefore on the same level as the region)
\item The Municipality of Venice and its local plans
\end{itemize}
This complex network results into the overlapping of the jurisdiction of the different entities, that to work together with a unified vision need to sign Program Agreements. However, there is quite a distance between the objectives of the different authorities and the effective action they can pursue unitarily. 

\begin{figure}[h]
\centering
\includegraphics[width=0.8\linewidth]{images/02_structure.png}
\caption{Visual representation of the hierarchy of the different entities that act upon Porto Marghera. Image created by the authors.}
\end{figure}

\subsection{On the European and national level}
\subsubsection{Funds and Programs}
The most important European funds and programs applied to Porto Marghera are the following. 

\textbf{ERDF: European Regional Development Fund}: The ERDF promotes regional economic growth and territorial cohesion. It finances infrastructure, innovation, environmental and energy projects. Its aim is to reduce regional disparities and support sustainable development. Actions focus on competitiveness, green transition, and digital transformation.

\textbf{PNRR: The National Recovery and Resilience Plan}: The PNRR is a strategic plan that aims to modernize and improve the Italian economy, making it more sustainable, innovative, and competitive. The Green Revolution and Ecological Transition component of Italy's PNRR has a total allocation of €55.52 billion (28.56\% of the total plan), instead restoration and protection of seabeds and marine habitats has €400 million \cite{pnrr}.

\textbf{Interreg Program}: Under the European funding programme \textit{Connecting Europe Facility}, the ports of Venice, Trieste, Koper e Ravenna, obtained a new co-funding for the implementation of the project called ACCESS2NAPA, for a total expenditure of over 14 million Euros. The North Adriatic Sea Port System Authority (NASPA) will contribute to the achievement of the growth objectives of the North Adriatic Ports Association (NAPA) by promoting two actions in particular, namely with the design of interventions in Porto Marghera for the strengthening of the last mile railway infrastructure (in the connection between via dell’Elettricità and the Petrochemical Peninsula) and road accessibility (the so-called Malcontenta node). NASPA will have a budget of € 880.000, 50\% co-financed and 36 months to complete the activities \cite{access2napa}.

\subsubsection{Directives}
\begin{enumerate}

\item \textbf{Directives 92/43/EEC (Habitats) and 2009/147/EC (Birds)} : It establishes the Natura 2000 network for biodiversity and habitat protection. All redevelopment projects must undergo an environmental assessment of impacts.\\
\textit{Porto Marghera is in the immediate vicinity of Natura 2000 sites, therefore they fully apply
Any redevelopment, remediation, port expansion or dredging in Porto Marghera must undergo a mandatory Natura 2000 Appropriate Assessment. Projects that risk significant impacts on protected species or habitats must be redesigned, mitigated, or compensated. If such impacts cannot be avoided, authorization can only be granted for imperative reasons of overriding public interest and in the absence of alternatives.}\\
$\rightarrow$ \textbf{The Presidential Decree 357/2003, transposes the Habitats Directive into Italian law}.

\item \textbf{Directive 2000/60/EC, Water Framework Directive (WFD)} : It establishes a common framework for water protection and management in the EU.\\
\textit{For Porto Marghera it includes the remediation. Since it is a SIN (Site of National Interest) the Italian State has direct responsibility for the site's remediation, regardless of any redevelopment projects.}\\

\item \textbf{Directive 2001/42/EC, Strategic Environmental Assessment (SEA)}: it ensures that environmental aspects are integrated into plans and programs at an early stage.\\
\textit{The SEA Directive (2001/42/EC) applies to strategic planning for the redevelopment of Porto Marghera, ensuring that environmental priorities are integrated upstream into zoning and industrial reconversion choices. The EIA Directive (2011/92/EU), on the other hand, applies to each specific project implementing the plan.  So applies to the masterplan.}\\
$\rightarrow$ \textbf{Transposed into Legislative Decree 152/2006.}

\item \textbf{Directive 2004/35/EC, Environmental Liability Directive (ELD)}: Introduces the “polluter pays” principle and the obligation to prevent or repair environmental damage.\\
\textit{In the absence of a clearly identifiable or solvent polluter, Directive 2004/35/EC stipulates that the State becomes responsible for remediation as a last resort. Porto Marghera was therefore classified as a SIN in 1998, transferring responsibility for remediation to the Ministry of the Environment (now MASE). Since then, the operations have been financed by national and European public funds (ERDF, LIFE, etc.), and not by the companies that originally carried out the remediation.}\\
$\rightarrow$ \textbf{European obligations (WFD + ELD) are transposed into Italian law (Legislative Decree 152/2006).}

\item \textbf{Regulation (EC) No 850/2004, POPs Regulation}: Aims to eliminate and restrict Persistent Organic Pollutants (POPs) such as dioxins, PCBs, PAHs and other long-lasting contaminants.\\
\textit{POPs are historically present in Porto Marghera due to petrochemical and metallurgical activities, therefore this applies to strict remediation obligations. Any redevelopment project must ensure that excavated contaminated materials are safely treated, disposed of or destroyed, following POPs elimination objectives.}\\
$\rightarrow$ \textbf{Directly applicable in Italy (EU Regulation) with national monitoring by the Ministry and ARPA Veneto}

\item \textbf{Regulation (EC) 1907/2006 (REACH)}:  it regulates the registration, evaluation and authorization of chemical substances in the EU to ensure high protection of human health and the environment.\\
\textit{In Porto Marghera, REACH applies to remaining industrial facilities and to the management of contaminated materials during remediation (soil, sludge, groundwater).
Any reuse, treatment, or transportation of polluted sediment and soil must comply with REACH obligations, particularly regarding hazardous substances and exposure prevention.
REACH strengthens control over chemicals historically released in the area (chlorinated solvents, hydrocarbons, metals).}\\
$\rightarrow$ \textbf{ Implemented in Italy through national enforcement under the Testo Unico Ambientale (D.Lgs. 152/2006) and specific REACH compliance monitoring by ISPRA/ARPA.}

\item \textbf{Directive 2008/50/EC, Ambient Air Quality Directive}: it sets limits for key air pollutants (NO$_2$, SO$_2$, PM$_10$, etc.)\\
\textit{The Veneto Region must integrate these obligations into its regional air quality plan (and potentially local plans for highly polluted areas like Marghera). Local authorities (municipalities, ARPA) are required to monitor air quality according to the national criteria of Decree 155. If an industrial zone exceeds the limit values, corrective measures must be adopted through an action plan (closure, emission reduction, traffic control, etc.). Citizens have a right to information: the data must be publicly accessible, which can serve as a tool for exerting pressure.}\\
$\rightarrow$ \textbf{In Italy, this is Legislative Decree 155/2010.}

\item \textbf{Directive 2010/75/EU, Industrial Emissions Directive (IED)}: it regulates major industrial installations to minimize emissions to air, water, and soil. Useful for remaining or new industrial facilities in Porto Marghera that require compliance or closure.\\
\textit{In Porto Marghera, existing industrial facilities are required to comply with the Industrial Emissions Directive (IED); otherwise, they risk suspension or termination of their activities. For any new installation, an Integrated Environmental Permit (IPPC/AIA in Italy) must be issued by the regional authority. This permit defines the specific Emission Limit Values (ELVs) for the plant, based on the Best Available Techniques and Associated Emission Levels (BAT-AELs) established at EU level, while also considering local environmental conditions.}\\
$\rightarrow$ \textbf{In Italy, Legislative Decree No. 152/2006 (“Testo Unico Ambientale”) implements the IED.}

\item\textbf{Directive 2011/92/EU, Environmental Impact Assessment}: it requires an environmental impact assessment for major public and private projects. Applies to infrastructure, remediation, and redevelopment projects in the port area.\\
\textit{Any significant redevelopment project in Porto Marghera must comply with the Directive and undergo a full Environmental Impact Assessment (EIA), given the sensitive environmental context and the presence of high-risk industrial activities. The EIA includes cumulative impact analysis, public participation, alternatives assessment and monitoring obligations. Projects may be rejected if impacts are considered unacceptable or insufficiently mitigated. General studies pre change.}\\
$\rightarrow$ \textbf{In Italy this is contained in the decree 152/2006 part 2 among other things}

\item \textbf{Directive 2012/18/EU, SEVESO III Directive}:it prevents and mitigates major industrial accident risks involving dangerous substances.\\
\textit{Several facilities in Porto Marghera are classified as SEVESO establishments → they must implement safety management systems, risk assessment and emergency planning.
Redevelopment cannot reduce safety standards: any land-use change must undergo a compatibility check with industrial risk constraints. Public right to information on risks is strengthened (risk transparency obligation)}. \\
$\rightarrow$ \textbf{Implemented in Italy through Legislative Decree 105/2015, coordinated by Prefecture and Regional authorities.}

\end{enumerate}

\subsection{On the regional level}
\subsubsection{A general overview of the plans of the region}
\begin{figure}[h]
\centering
\includegraphics[width=\linewidth]{images/02_PTRC.png}
\caption{Hierarchy of the regional plans and its territorial validity. Image extracted from the official website of the Veneto region.} 
\label{fig:gerarchia}
\end{figure}
Porto Marghera is part of the Municipality and Province of Venice, which in turn is part of the Veneto Region. The latter has to provide the whole territory with a series of plans \cite{regioneveneto}, as all Regions in Italy must do. In particular, the Veneto Region must arrange: 
\begin{itemize}
\item \textbf{Piano territoriale regionale di coordinamento (PTRC)}: it represents the regional tool to govern the territory and its use. It indicates the objectives and the main lines of organization and structure of the regional territory, as well as the strategies and actions aimed at their realization. It is structurally above a series of more specific tools that apply on smaller scales as in image \ref{fig:gerarchia}.

\textbf{The latest version was approved by the Giunta Regionale on the 06/30/2020}

\item \textbf{Piano Regionale di Tutela e Risanamento dell’Atmosfera (PRTRA)}:  it represents the regional tool to regulate activities that create pollutant emissions in the atmosphere.

\textbf{The latest version was approved by the regional Council on 04/19/2016. Currently a new version is under development, since the update approval of the Giunta Regionale of 11/11/2021. }

\item \textbf{Piano di Tutela delle Acque (PTA)}: it is the regional tool that regulates the protection and preservation of hydrological resources. It establishes the procedures to protect all types of water bodies and the methods to guarantee a sustainable use of water. It is divided into three parts: a descriptive document, a document with the plan directions and finally a document with the technical regulations.

\textbf{The latest version was approved by the deliberazione del Consiglio Regionale n.107 on 11/05/2009, and with the last modification approved with the D.G.R.V. n. 1023 on the 07/11/2018.}

\item \textbf{Piani di Gestione delle Acque dei Distretti idrografici delle Alpi Orientali e del fiume Po}: parallel tool regulated by the hydrological district. 

\item \textbf{Piano Regionale per la Bonifiche delle Aree Inquinate (PRBAI)}:  it is the functional instrument for the analysis of critical situations and the identification of priority interventions, with which the Region, in implementation of current legislation, carries out a sustainable management of its territory and its resources. It is important to note that this plan does not apply to Porto Marghera, since the presence of the SIN regulation. 

\end{itemize}

\subsection{Port Authority level}
The Port Authority in Italy is a separate entity with respect to the region and it refers directly to the national regulations. In other words, it is parallel to the region, thus it has its own set of laws for the territory of its property. In the following subsection an overview of its most important regulations is provided. 

\begin{itemize}
\item \textbf{The Piano Regolatore Portuale dell’Autorità Portuale di Venezia}: approved by the Ministero dei Lavori Pubblici with the Decreto n.319 of 15/05/1965. With the Riforma della Legislazione Portuale of D.Lgs.169/2016 the new Authorities of the Harbour System have been established, in this specific context with the Autorità di Sistema Portuale del Mare Adriatico Settentrionale (AdSPMAS), regarding the ports of Venice and Chioggia, now seen as a unified system. 
\item \textbf{Three-year operational plan}: The Three-Year Operational Plan is a planning document issued by the North Adriatic Sea Port Authority, required by the Law No. 84/1994. It is revised annually and it is prepared every three years. It regards the planning of the port activities, actions and interventions in order to achieve previously set objectives.

\end{itemize}

\subsection{On the Province level (Città metropolitana di Venezia)}
As mentioned above, Porto Marghera forms part of the Province of Venice, whose specific name is “Metropolitan city of Venice” since law 56/2014. This authority must arrange a series of important plans and regulations on its territory both in the environmental and urban field. 

\textbf{Il Piano Territoriale Generale Metropolitano}:
Since the switch from the Province of Venice to the Metropolitan City of Venice in 2014, the previously existing Piano Territoriale di Coordinamento Provinciale (PTCP) was renamed as “Piano Territoriale Generale (PTG) della Città Metropolitana di Venezia” keeping the contents unvaried. Therefore, the PTG dates back to 2019, but it actually contains the same contents of the latest version of PTCP, which in turn dates back to 2014. 

\textbf{Il Piano Territoriale di Coordinamento Provinciale}
The PTCP outlines development and governance strategies for the Province of Venice. Adopted in 2008, the Plan is defined as open, flexible, and concerted, born out of a broad participatory process aimed at addressing a rapidly changing economic and regulatory environment.

\textbf{Piano di Assetto del Territorio}
The Piano di Assetto de Territorio (PAT) outlines strategic choices for the structure and development of the territory, identifying specific characteristics and constants of a geological, geomorphological, hydrogeological, landscape, environmental, historical, monumental, and architectural nature. The PAT is a “structural plan,” or a planning document that:
\begin{itemize}[leftmargin=*, noitemsep]
\item adopts the general guidelines of higher-level instruments (PTRC, PTCP, PALAV) and municipal instruments relating to the wider area (Strategic Plan, Urban Mobility Plan).
\item outlines major choices for the territory and strategies for sustainable development;
\item defines the functions of the different parts of the municipal territory;
\item identifies areas to be protected and enhanced due to their environmental, landscape, and historical-architectural importance;
\end{itemize}

%---------------sezione Accordi di Programma ----------------------------------------------
\section{"Accordi di Programma" for Porto Marghera}
Given the legislative evolution for Porto Marghera presented in paragraph 2.1, a section of explanation with a bit more detail about the \textit{Accordi di Programma} seems much needed. Thus, the following sub-paragraphs present the objectives of the most relevant Program Agreements signed between 1998 and 2012. 
\subsection{Accordo di Programma per la Chimica (1998)}
The Program Agreement for Chemicals in Porto Marghera was signed on October 21, 1998, and approved by DPCM of February 12, 1999. Its objective was to establish and maintain optimal conditions for coexistence between environmental protection and production development in the chemical sector.
The 1999 Chemicals Agreement set the following objectives:
\begin{itemize}[leftmargin=*, noitemsep]
\item Rehabilitate and protect the environment through cleanup, remediation, and site safety measures, reduction of emissions into the atmosphere and the lagoon, and prevention of major accident risks (SIMAGE)
\item Innovation, competition, employment: Inducing adequate industrial investment, with the aim of equipping existing plants with the best environmental technologies and making them competitive at European level, ensuring their long-term viability and ensuring the maintenance, revitalization, and qualification of employment.
\end{itemize}
A supplementary Act was signed in November 2001 by Ministries, Region, local institutions, trade unions and the most important firms working on the area.

\subsection{Accordo di Programma per l'Idrogeno (2005)}
The Program Agreement for the Porto Marghera hub stems from the availability of hydrogen already present in the area, derived as a byproduct of petrochemical industrial processes. This resource represents a competitive advantage that enables immediate and low-cost procurement, essential for fueling research. The project leverages the consolidated expertise of local companies, already united in the "Hydrogen Park – Marghera per l’Idrogeno" consortium, which serves as a strategic driver for the district's development and guarantees excellent expertise across the entire gas supply chain.

The four-year initiative is designed as an open-air innovation laboratory aimed at testing cutting-edge solutions for the production, storage, and use of hydrogen. The central objective is the integration of fuel cells in both stationary electricity generation and sustainable mobility. Through this synergy between industry and research, the Agreement aims to transform Porto Marghera into a national hub for the energy transition and the use of zero-emission technologies. In addition to this, the Program Agreement for Hydrogen led to the creation of plants for CO$_2$ catchment and reuse. The Program Agreement was later extended with the Addendum of December 2009.

\subsection{Accordo di Programma Vallone Moranzani (2008)}
The "Accordo di Programma per la gestione dei sedimenti di dragaggio dei canali di grande navigazione e la riqualificazione ambientale, paesaggistica, idraulica e viabilistica dell'area di Venezia - Malcontenta – Marghera" (also known as "Accordo Moranzani", was signed in 2008 by a number of entities. The "Moranzani" Program Agreement arose from the need to identify a permanent disposal site for dredged sediment from the port canals, an alternative to the site initially envisaged by the Progetto Integrato Fusina. More is presented in paragraph 2.5.1.

\subsection{Accordo di Programma per le Bonifiche (2012)}
Following reflections on the administrative burden, highlighted as one of the main critical issues in the remediation process, a key administrative act was adopted that represents a new beginning in the Porto Marghera reclamation process: on April 16, 2012, a Program Agreement for the remediation and environmental redevelopment of the Porto Marghera and surrounding areas was signed by the Ministry of the Environment, the Venice Water Authority, the Veneto Region, the Province of Venice, the Municipality of Venice, and the Venice Port Authority. The signed Program Agreement had two fundamental objectives:
\begin{itemize}[leftmargin=*, noitemsep]
\item to accelerate and simplify the remediation procedures for the Porto Marghera, supporting businesses in accessing credit for the implementation of the interventions
\item to define an initial list of new projects to be implemented in the area with simplified procedures, open to further participation
\end{itemize}

%----------------sezione detailed -------------------------------------------------------------------
\section{Detailed study of the relevant plans for Porto Marghera}
In this section, more insights are provided about some of the planning tools presented in the previous part. More specifically, the analysis delves into four regulatory plans:
\begin{itemize}[leftmargin=*, noitemsep]
  \item \textbf{Piano Regolatore Portuale per il Porto di Venezia} (1965): its the most important regulatory tool for the area managed by the Port Authority and it is still valid today, in spite of its outdatedness.
  \item \textbf{Variante al Piano Regolatore Generale per la zona di Porto Marghera} (1999): the very first serious regulatory plan created by the Municipality of Venice that affirmed its decisional power on the area. It is also the first official document where the critical condition of the environment in Porto Marghera is confirmed.
  \item \textbf{Masterplan per la bonifica dei siti inquinati} (2004): its the only tool that assessed in a technical way how to operate the remediation of the polluted soil and groundwater. 
  \item \textbf{Piano di Assetto del Territorio} (2014): its the most recent regulatory plan for the area of Venice that gives a picture of the evolution of Porto Marghera during the beginning of the new century in addition to showing the expectations for the future of the area.
\end{itemize}
%---------------- Porto -------------------------------------------------------------------------------
\subsection{Piano Regolatore Portuale per il Porto di Venezia}
The Port Regulatory Plan is the planning tool for the system of ports that are under the control of the Port System Authority, in the specific case of Porto Marghera of the North Adriatic Sea Port Authority. It is defined as such by the Port Reform Law 85/1994 and its subsequent amendments and integrations, such as the Legislative Decree No. 169/2016 of August 4, 2016 and the Legislative Decree No. 232/2017 of December 31, 2017, known as “Port Corrections”.
The current PRP dates back to the year 1965, but a new plan is currently undergoing drafting.
\subsubsection{Structure}
The Piano Regolatore Portuale (PRP) is made up of two different levels:
\begin{itemize}[leftmargin=*, noitemsep]
\item The System Strategic Planning Document (DPSS): it is the document that describes the complete vision of the port area managed by the Port Authority of the Adriatic Sea.
\item The Port Regulatory Plans: these are the specific plans for each port and port areas, such as Porto Marghera.
\end{itemize}
The current 1965 plan for the port industrial area includes the site plan, the excavation section and the use destinations.
\begin{figure}[h]
\centering
\includegraphics[width=0.77\linewidth]{images/02_portplan.png}
\caption{Planimetry of the Port Regulatory Plan (1965), available through the Port Authority of the North Adriatic Sea} 
\label{fig:portplan}
\end{figure}
\subsubsection{The new plan}
The Port Authority of the North Adriatic Sea is currently drafting two new RPPs, one for Venice, which will include Porto Marghera, and one for Chioggia, which has a plan dating back to 1908.

As of 2025, the DPSS is going to finish its draft and be approved. The new plan is being thought out by a technical group from AdSP and by a group of enterprises, such as Rina, StudioPaolaViganò, Acquatecno, Mtbs.
The DPSS defines the main functions as:
\begin{itemize}[noitemsep]
\item Port functions, such as warehouses, docks and cargo terminals
\item Port hinterland areas functions, such as logistics, industries and interactions with the city
\item Infrastructure, such as railways.
\end{itemize}

\subsubsection{Objectives}
\begin{itemize}[leftmargin=*, noitemsep]
\item Environmental regeneration of the brownfield areas of Porto Marghera, that are mostly non-efficient or disused
\item Sustainability, with more efficient land and soil use and the reduction of the use of new soil.
\item Implementation of new terminals, such as a container terminal in the Montesyndial area, through a series of strategic investments
\item Implementation of new railway connections to improve freight transport, more specifically in the Via della Chimica hub.
\end{itemize}

\subsubsection{Relationship with other plans}
The PRP has to operate with the Master Plan per la bonifica dei siti inquinati di Porto Marghera, as the actions described in the Master Plan have to take place in the area which is regulated by the PRP.

For the same reason, within the Accordo di programma per la riconversione e riqualificazione dell’area di crisi industriale complessa di Porto Marghera, there are relationships between the AdSP and the Ministero dello Sviluppo Economico, the Veneto Region, and the Municipality of Venice.
The area division as planned with the PRP is also utilized by the Piano di Raccolta e Gestione dei Rifiuti of the Port Authority.

\subsubsection{Main issues}
The fact that the current PRP dates back to 60 years ago, which brings the disadvantage of it not being specifically designed for the modern needs of sustainability and regeneration, as well as new functions.

The PRP has to be coordinated with infrastructure investments, land use plans and remediations plans, in order to plan for an efficient industrial conversion. Furthermore, there has to be balance between the port function and the urban regeneration function, making sure that there are areas that efficiently carry on with the classic port functions, and other ones that are converted to city areas.

The revisions to the plan need a wide number of steps, both technical and administrative, such as the conference of services, the VAS and the VINCA, which takes a considerable amount of time.

\begin{figure}[H]
\centering
\includegraphics[width=\linewidth]{images/02_newportplan.png}
\caption{Most recent planimetry as defined by the Decreto n. 865 of 27/12/2022, to define areas that have lost the port function and are considered areas of port-city interaction. Extracted from the official website of Port Authority.} 
\label{fig:portplan}
\end{figure}
%---------------------------- Venice Masterplan --------------------------------
\subsection{Masterplan for Porto Marghera}
The Variante to the General Master Plan PRG of Porto Marghera (1999) represents one of the most important urban planning instruments adopted by the City of Venice and approved by the Veneto Region, specifically designed to reorganize and redevelop a large industrial and port area. Historically, the planning of Porto Marghera had been dominated by the State and the Port Authority due to the area's particular legal regime and to the undefined jurisdiction of the Municipality of Venice, which went through the many legislative changes that followed the creation of the Italian Republic. As a matter of fact, since the construction of Porto Marghera in 1917, the area witnessed a great variety of historical events (such as the II World War, the fascist regime and the aforementioned birth of the Italian Republic) during which territorial regulation wasn’t a priority. 
\subsubsection{Assessed problems}
\textbf{1. No previous planning}: Porto Marghera has not, until 1995, been the subject of explicit planning action by the Municipality, which has limited itself in its urban planning instruments to merely sanctioning its designated land use, without addressing the merits of its physical internal organization. The plans elaborated in the past for the industrial zone—in 1925 and 1960, respectively for the first and second zones—were drawn up and sanctioned by the State and subsequently modified repeatedly by the Provveditorato al Porto (Port Authority) through its own Port Master Plan (Piano Portuale). This was due to the particular legal regime under which the zone itself had been constituted at the time, which made the Municipality's jurisdiction uncertain until 1995. It’s only with the Variante al Piano Regolatore Generale per la zona di Porto Marghera (VPRG) that the Municipality of Venice, in accordance with the PALAV, can affirm its authority and its planning jurisdiction on the industrial area.
This implies a series of problems that are the result of the lack of planning:
\begin{itemize}[noitemsep]
\item Urban isolation
\item Inadequate infrastructure
\item Environment degradation
\end{itemize}

\textbf{2. Economic and industrial crisis}: The VPRG seeks to drive a restructuring process designed to overcome the industrial contraction experienced since the ‘70s. The VPRG states that Porto Marghera continues to perform vital functions for the local economy. Its importance has not been undermined by previous restructuring cycles, since it guarantees employment and it gives diversification to Venice’s economy. However, at the time of the VPRG, Porto Marghera is in the middle of an economic crisis: the area has, in fact, been included among the "Aree di crisi" by the Government's Task Force for the Economy to counter the ongoing "deindustrialization. The causes of this crisis are multiple and complex, but it is possible to highlight mainly that: 
\begin{itemize}[noitemsep]
\item there are structural difficulties in the traditional ("historical") sectors of the hub, such as basic chemistry, aluminum, and coke processing, which were the main industrial activities to develop at the time of Porto Marghera’s birth.
\item there is absence of new and significant entrepreneurial initiatives
\item Poor control over the ripple effects of the crises faced by the hub's large basic companies
\end{itemize}
At the level of public governance, negative trends have been fueled by a series of unresolved issues and critical decision-making failures, notably the failure to dredge the industrial canals and absence of a unitary project for the industrial hub's reconversion. This complex combination of factors and the excessive bureaucracy, has prevented the creation of necessary synergies for development. 

\subsubsection{Planning solutions}
The VPRG’s proposed solutions for overcoming the aforementioned problems are listed in the following part: 

\textbf{1. No previous planning $\rightarrow$ General reorganization of the area, creation of reclamation and civil protection plans}: 
\begin{itemize}[leftmargin=*, noitemsep]
\item new zoning on the basis of studies on the problems of the area, on the existing infrastructure for gas and water and the presence of cultural heritage sites. 
\item redevelopment of the Venice Port:  the restoration of 50 ha occupied by abandoned industrial plants, with the idea of relocating logistics enterprises inefficiently located in the area. 
\item creation of new streets for Via dell’Elettricità: the main objective is reducing heavy traffic caused by the harbor
\end{itemize}
\textbf{2. Economic and industrial crisis $\rightarrow$ Creation of space for innovation and new small enterprises}
\begin{itemize}[leftmargin=*, noitemsep]
\item Creation of space for small and mid-sized companies: a project prepared by the Municipal Administration in order to entrust territorial lots to chosen enterprises. 
\item Establishment of “Parco Scientifico Tecnologico”: This project was born in 1992 under the protection of the European Community, with the participation of public institutions and private companies interested in technological innovation. The project establishes a gradual recovery of abandoned areas in the 1st industrial zone. The main objective is connecting this area with the University of Venice, in order to create a direct connection between the enterprises and the technological innovations.
\end{itemize}

\subsubsection{Strategy}
The “piano di intervento” considered by the VPRG is constituted of 5 main points: 
\begin{enumerate}[leftmargin=*]
\item The strategic use of available resources:
\item The promotion of memoranda of understanding between the Municipality of Venice and other entities with jurisdiction over the area (Veneto Region, Province of Venice, Port Authority/Port Superintendence) or stakeholders (State Railways, landowners, etc.). Two other parties are especially considered: the "Provveditorato di Porto" and the existing companies. 
\item The definition of infrastructure interventions of territorial significance (new southern access to the port area; separation of urban traffic from industrial traffic; connection between the industrial area and Via Torino; etc.).
	\begin{itemize}[noitemsep]
		\item a new set of streets to connect Porto Marghera with the outside
		\item reorganization of the streets inside the hub
		\item expansion of the Terminal Fusina
		\item street connection between the university and the industrial hub
	\end{itemize}
\item A zoning of the industrial hub's areas functional to the valorization of Porto Marghera's specific potential.
\item The definition of terms and methods of land use (Technical Implementation Standards) able to ensure the flexibility necessary for the remediation processes that are intended to be promoted and, at the same time, capable of providing the certainties required by entrepreneurial forces.
\end{enumerate}
\subsubsection{Evaluation of achieved results}
The VPRG succeeded in realizing some of the projects included in its objectives: in this subsection an assessment of the most relevant achievements is made:
\begin{itemize}
\item The creation of the "Parco Scientifico Tecnologico VeGa": In 1993, the company Venice Gateway for Science and Technology (VeGa) was instituted. VeGa is a "Parco Scientifico Tecnologico",  a compound of innovative firms of high technology. It was originally founded by 34 public and private partners to promote the development of the Porto Marghera industrial area through the creation of research centers, activities, and services. It started to be build in the late 90s, on the territory that was previously occupied by "Enichem Agricoltura", an abandoned area since 1986. In its first 10 years of activity, VeGa promoted the redevelopment of 35.000 m$^2$ of land through the construction of new buildings (mainly for office and university use) thanks to the EU structural funds. Nowadays it hosts companies working in ICT, green economy and environmental sustainability. 
\item Parco San Giuliano: Parco San Giuliano was a former marshland area that was used as a dumping ground for industrial sludge and waste during the peak years of production in Porto Marghera. In accordance with the objective of the PRG of creating green spaces, it was transformed into a park in 2003 with the aim of conserving and safeguarding the lagoon's habitat, fauna, and flora, thereby combating environmental degradation. Nowadays it is still an important green area for the inhabitants of the city of Mestre.
\item SIMAGE system: SIMAGE is the acronym of "Sistema Integrato per il Monitoraggio Ambientale e la Gestione delle Emergenze" a prevention system to manage industrial risk in Porto Marghera. It consists of a monitoring network made up of sensors placed directly inside the industrial plants. Its functionality over the years has been guaranteed by ARPAV, EZI (Porto Marghera Industrial Zone Authority), APV (Port Authority of Venice), and companies of the ENI group in Marghera. Its creation is one of the achievements of the VPRG, which included it inside its plan of civil protection, and of the Accordo di Programma per la Chimica of 1998. It is still working today and its functioning is currently regulated through the latest Accordo di Programma for the triennium 2023-2026.
\end{itemize}
\begin{figure}[H]
\centering
\includegraphics[width=\linewidth]{images/02_achievements.jpeg}
\caption{Location of the aforementioned achievements. Map created by the authors with QGIS.} 
\end{figure}

%--------------------------- Remediation Masterplan ----------------------------
\subsection{Masterplan per la bonifica}
The Masterplan per la bonifica dei siti inquinati di Porto Marghera (2004) is the main planning instrument dedicated to the coordination and regulation of environmental remediation interventions within the Porto Marghera National Priority Site (Sito di Interesse Nazionale – SIN). It was developed to provide a unified strategic and operational framework for the clean-up of contaminated soils, groundwater and sediments, ensuring coherence between remediation processes, territorial planning tools and redevelopment strategies.

\subsubsection{Main structure}
The Masterplan is structured as an integrated technical and planning document composed of several complementary components.
First, it includes a territorial and environmental classification framework, based on site characterization data, which subdivides Porto Marghera into homogeneous areas according to contamination typologies, risk levels and hydrogeological conditions.

Secondly, the plan defines a zoning system for remediation priorities, identifying areas that require urgent intervention and those that can be addressed in later phases, depending on environmental risk and redevelopment perspectives.

Finally, the Masterplan contains a set of technical guidelines and operational procedures, regulating the different phases of remediation, from preliminary investigations to monitoring after intervention, and defining the institutional responsibilities of the involved actors.

This structure allows the Masterplan to act both as a strategic coordination tool and as a technical reference framework for individual remediation projects.

\subsubsection{Existing issues}
The Masterplan addresses a series of structural and operational issues that characterize remediation processes in Porto Marghera.
One of the main challenges is the complexity and heterogeneity of contamination patterns, which involve different pollutants, environmental matrices and spatial distributions. This complexity requires differentiated remediation approaches and limits the applicability of standardized technical solutions.
Another major issue is the strong interaction between soil, groundwater and surface water systems, due to the presence of canals and hydraulic connections with the lagoon. This condition increases the risk of pollutant dispersion and complicates containment and treatment strategies.
In addition, the fragmentation of land ownership and institutional responsibilities represents a critical governance challenge. The involvement of multiple public authorities and private stakeholders increases administrative complexity and can slow down decision-making and implementation processes.
Finally, remediation activities are affected by high economic costs and long implementation timelines, which may delay redevelopment projects and reduce the attractiveness of private investments in the area.

\subsubsection{Objectives}
The general objective of the Masterplan is to ensure the systematic and coordinated remediation of contaminated areas in Porto Marghera, reducing environmental and health risks while enabling the future reuse of industrial brownfields. More specifically, the plan aims to
\begin{itemize}[leftmargin=*, noitemsep]
\item reduce environmental and health risks associated with contaminated soils, groundwater and sediments in Porto Marghera;
\item enable the safe reuse and redevelopment of brownfield areas, removing environmental constraints to territorial transformation;
\item restore environmental quality and ecological functionality in heavily impacted industrial zones;
\item ensure long-term environmental safety, preventing future pollutant dispersion towards the lagoon and surrounding urban areas;
\item support the structural reconversion of the industrial district by providing a stable environmental framework for new economic activities.
\end{itemize}

\subsubsection{Strategies and actions}
The Masterplan defines a set of coordinated remediation strategies aimed at ensuring environmental safety while supporting the progressive redevelopment of Porto Marghera. The main strategic guidelines can be summarized as follows:
\begin{itemize}[leftmargin=*, noitemsep]
\item The adoption of a risk-based remediation approach, prioritizing interventions according to exposure pathways, environmental sensitivity and potential impacts on human health, rather than relying exclusively on concentration threshold values.
\item The phased remediation process, articulated through different levels of intervention: Messa in Sicurezza d’Emergenza (MISE), aimed at the immediate containment of active pollution sources; Messa in Sicurezza Permanente (MISP), focused on long-term containment and isolation of contaminated matrices; Bonifica, consisting of the removal or treatment of contaminated soils and groundwater where technically and economically feasible.
\item The integration between remediation and land-use planning, ensuring that clean-up strategies are compatible with current and future land uses defined by urban, port and territorial planning instruments.
\item The coordination between soil, groundwater and sediment management, particularly in canal areas and waterfront zones, where contamination processes are strongly interconnected with hydraulic dynamics and dredging activities.
\item The promotion of technically flexible remediation solutions, encouraging the adoption of site-specific approaches, including in-situ treatments, hydraulic barriers, soil containment systems and selective excavation techniques, depending on local environmental conditions.
\end{itemize}

In addition, the Masterplan establishes technical planning instructions and regulatory rules governing site characterization methodologies, project approval procedures, monitoring requirements and post-remediation control phases. These provisions are supported by a comprehensive cartographic system, including contamination maps, remediation priority zoning, hydrogeological vulnerability layers and land-use compatibility maps, which facilitate coordination between environmental remediation and territorial planning processes.

\subsubsection{Way of action and implementation}
The implementation of the Masterplan is based on a multi-level governance framework involving national, regional and local institutions. The Ministry of Environment plays a central coordination role, while the Veneto Region, the Municipality of Venice and the Port Authority are responsible for the integration of remediation actions with territorial planning and port activities. Environmental monitoring and control functions are carried out by specialized agencies such as ARPAV (Agenzia Regionale per la Prevenzione e Protezione Ambientale del Veneto).

Operationally, remediation projects are implemented through programmatic agreements and administrative procedures that define responsibilities, funding mechanisms and execution timelines. 

\subsubsection{Evaluation}
Despite the structured framework introduced by the Masterplan in 2004, the progress of remediation activities in Porto Marghera has been relatively slow. By 2016, although remediation procedures had been activated on more than 90\% of the area, only about 14\% of the industrial district had effectively completed the clean-up process (see figure \ref{fig:SINfalda} and \ref{fig:SINterreni}). A large share of the sites remained at intermediate administrative or characterization stages. This situation highlights the difficulty of translating strategic planning objectives into concrete remediation results, mainly due to technical complexity, long authorization procedures and high intervention costs.
%-------------------------- PAT--------------------------------
\subsection{Piano di Assetto del Territorio}
The Piano di Assetto del Territorio (PAT), as defined in Article 13 of Regional Law 11 of 2004, establishes the objectives and sustainability conditions for eligible interventions and transformations and is drawn up by Municipalities based on ten-year forecasts. Venice's PAT was approved on 10/10/2014 and it is still valid: it refers to the PRG of 1999 by evaluating the evolution of the territory throughout the years and by setting the objectives for the future years. 
\subsubsection{Structure}
The PAT of the municipality of Venice is subdivided into three main parts: first, a knowledge framework (\textit{quadro conoscitivo}), secondly, a strategic environmental evaluation (\textit{Valutazione Ambientale Strategica (VAS)}) and finally the recollection of technical regulations together with the cartography (see figure \ref{fig:convertibility} for a re-design of the convertibility map of the area, one of the most interesting products of the PAT).  For the goals of this work, it is relevant to consider only what is mentioned regarding the environment and Porto Marghera specifically.
\subsubsection{Objectives}
The objectives of the PAT of Venice are stated in an assigned paragraph, where it is underlined that all of its objectives are in accordance with national and regional legislation, as clearly explained in the previous paragraphs. All objectives are divided into subgroups according to their common theme. These subgroups are: 
\begin{itemize}[leftmargin=*, noitemsep]
\item Environmental protection
\item Protection of cultural, historical and architectural heritage
\item Settlement system
\end{itemize}
Porto Marghera has a set of objectives of its own inside the section of "business activities", located inside the "settlement system part". Its condition of decommissioning of the plants is considered, but it is still regarded as one the core of economic activity in the municipality. Marghera's set of goals reads: 
\begin{enumerate}[leftmargin=*, noitemsep]
\item consolidation and possible expansion of productive activities (especially shipbuilding and new environmentally sustainable technologies)
\item physical and functional redevelopment of the area (introduction of new urban functions especially in the area near Via Fratelli Bandiera which could represent a potential extension of the city and its services)
\item productive reconversion of disused or underutilized areas with the expansion of port-related functions
\item improvement of accessibility from the south (connection with "Romea commerciale" and SP81)
\end{enumerate}
In addition, as a project choice, in the PAT it is stated that Porto Marghera is considered as a place of development of logistic integrated platform and/or as an opportunity for innovation and research. Instead, regarding the environmental clean-up, the PAT declares complete accordance with the \textit{"Protocollo di condivisione delle linee strategiche per la riqualificazione e lo sviluppo di Porto Marghera"} signed on the 21/09/2007 by the veneto Region, the Province of Venice, the Municipality of Venice, the ARPAV, the "unione degli Industriali"  and trade union organization. 
\subsubsection{Strategy: Ambito Territoriale Omogeneo di Porto Marghera}
The PAT subdivides the whole territory in "Ambiti Territoriali Omogenei (ATO)", which means areas that have common territorial features. For the ATO of Porto Marghera, the PAT considers the following planning choices :
 \begin{itemize}[leftmargin=*, noitemsep]
    \item the consolidation and strengthening of port functions, including the possible location of a new cruise ship terminal
    \item the identification of redevelopment and/or conversion areas across much of the district, provided they are not already involved in implementation plans, given the potential Porto Marghera expresses within the territorial and infrastructural context 
    \item the potential functional reconversion, while remaining consistent with the \textit{Accordo di Programma per la Chimica}, primarily aimed at establishing innovative and environmentally sustainable industrial production
     \item the identification of the areas between via dell'Elettricità and via F.lli Bandiera as strategic for the improvement of urban and territorial quality, both because they are in direct contact with the residential areas of Marghera (to the point of suggesting a small residential quota), and because their redevelopment/transformation is functional to improving accessibility from the South
    \item the enhancement of the waterfront through a project aimed at mitigating the impact of the industrial skyline through environmental and landscape redevelopment works.
\end{itemize}
Finally, it is interesting to note that in the part about the numerical sizing of the PAT, in the non-residential part, the additional settlement load (\textit{carico insediativo aggiuntivo}) is mainly attributed to Porto Marghera, with a value of 4.550.000 m$^2$ on a total of 7.146.000 m$^2$ distributed on the total area of the municipality of Venice. 
\subsubsection{Evaluation}
The PAT succeeds in establishing a set of goals that consider the population and its possible use of former industrial areas. However, it lacks in considering the population's real needs, that should be assessed with participatory processes. 

When it comes to evaluating what has actually been accomplished, the list remains quite short: no changes where made to the port, the reconversion of many buildings is still unaccomplished and no landscape redevelopment works were started. Nevertheless, some steps forward were made: about reconversion, in 2017 Venice Heritage Tower was created; a new cultural space was born from a former cooling tower \cite{venicetower}. 
Regarding the environment, the PAT does not really assess the problem directly and mainly focuses on the industrial reconversion of the area. 
\begin{figure}[H]
\centering
\includegraphics[width=\linewidth]{images/02_venicetower.png}
\caption{Picture of the renovation of the Venice Heritage Tower. Extracted by \cite{venicetower}} 
\label{fig:tower}
\end{figure}

%--------------------------------------sezione projects--------------------
\newpage
\section{Projects}
\subsection{Progetto Integrato Fusina}
The Progetto Integrato Fusina (PIF) is a complex project that concerns the construction of a plant for the collection, treatment and safe discharge of dangerous water of various origin. The PIF is presented as a high-profile engineering work of strategic important that serves a preeminent public function. Its core objectives are: 
 \begin{itemize}[leftmargin=*, noitemsep]
 \item Pollution reduction within the drainage basin of the Venice Lagoon by drastically limiting discharges, even those that have been treated.
 \item The remediation of contaminated sites in Porto Marghera, where the PIF serves as the key component for the water cycle.
 \item The optimization of water resource management by implementing extensive recycling of water used for industrial purposes.
 \end{itemize}
Technically, the project plans a treatment within a single functional platform of industrial discharges, contaminated groundwater resulting from the safety measures implemented at the Porto Marghera site, runoff water from potentially contaminated sites and all domestic sewage and rainwater from Mestre and Marghera. This should sum up to a total of 160.000 m$^3$ per day of treated wastewater, that is later put under refinement in a phytodepuration wetland ("Cassa di Colmata A"). When water is renewed, it is destined to industrial use in cooling plants, saving hundreds of thousands liters of drinkable water \cite{pif}. 
Furthermore, the project includes the construction of a final discharge into the open sea via an underwater pipeline. This pipeline transfers treated water from the final section of the plant to a point located approximately 10 km offshore from the Venice Lido, adhering to limits that are even stricter than those imposed by the European Community.
In addition, the project includes also the creation of a didactic - ludic park of 10 hectares on the South of Porto Marghera, in accordance with the objectives of the Accordo "Moranzani". 
\begin{figure}[H]
\centering
\includegraphics[width=0.85\linewidth]{images/02_pif.jpg}
\caption{Map of water collection channels of the PIF. Extracted by \cite{pif}} 
\label{fig:portplan}
\end{figure}

\subsection{Veritas SpA Eco Park}
In the southern part of Porto Marghera, in the same area of the development of the Progetto Integrato Fusina, a relevant example of Industrial Symbiosis (IS) is operating since 2017. An Industrial Symbiosis is a production system that engages traditionally separate entities in a collective approach involving physical exchange of by-products to substitute raw material input \cite{is}. The IS in Porto Marghera is an example of urban and industrial symbiosis network of national relevance: it enables a high percentage of materials recycling thanks to the collaboration between the public waste collection company Veritas and other connected industries that work in the sorting and recycling of materials.

The Ecodistretto di Porto Marghera was born in 2017, after the interruption of the former waste-to-energy plant. Its creation was driven by two different companies, Ecoprogetto Venezia S.r.l. and Eco-Ricicli Veritas, which converged into one in 2022, thus becoming Eco+Eco S.r.l. . This company forms part of the Veritas group, a municipally owned S.p.A. and one of the largest multiutility in Italy. The former Ecoprogetto Venezia S.r.l. was an integrated facility for processing unsorted waste; meanwhile, Eco-Ricicli Veritas operated as a platform for sorting waste from separate collection. Now, the two plants are still divided, but together they provide an environmentally-safe management of waste for approximately 890.000 people \cite{ecodistretto}. The Ecodistretto also welcomes third-party companies engaged in recycling materials at the early stages of the value chain.
\begin{figure}[H]
\centering
\includegraphics[width=\linewidth]{images/02_ecoprogetto.png}
\caption{Map of the industries engaged in the Eco Park. Extracted from \cite{ecodistretto}} 
\label{fig:ecodistretto}
\end{figure}


\subsection{VeGa waterfront}
VeGa's Venice Waterfront project is the first part of a development project for the northern macro-area of Porto Marghera. Twenty years after the construction of the Venice Science and Technology Park, Expo 2015 became a driving force for the VeGa project, which reclaimed a disused area along the Brentella canal, the first of four planned development scenarios.
The pretext for the regeneration of this initial site was the \textit{Venice Green Dream 50x50}, a collateral event at the 2012 Architecture Biennale, which triggered the reclamation process of a portion of the area.
Thanks to the extraordinary legislative tools provided by Expo, the area was reclaimed, the route connecting it to the Porto Marghera train station was redeveloped, and the\textit{ Expo Aquae Pavilion} was built, with its experimental showcase for phytoremediation.

\begin{minipage}{0.65\textwidth}
\vspace{0pt}
The project, as described by VeGa, is part of a broader vision: \textit{the green tree strategy}, a metaphor that sees Venice islands as the roots of a tree, and its mainland as the green branches and leaves, the symbol of development opportunities. This urban-environmental regeneration strategy does not appear to have any local implications. The leaves  have no geographical connection. The Green Tree Strategy and the strategic development of Venice's port system trace the boundaries of many "Porto Margheras", which are irreconcilable. Both visions, while interesting, should perhaps clarify and question the nature of the points where the geography of flows (and the geography of economic interests) meets the geography of places. 

{\footnotesize (Translated from: Peron I. (2016), Potenzialità contese. Porto Marghera, una questione di metodo.) }
\end{minipage}
\hspace{10pt}
\begin{minipage}{0.3\textwidth}
\centering
\includegraphics[width=\linewidth]{images/02_treestrat.pdf}
\captionof{figure}{Representation of the Green Tree Strategy by VeGa.}
\end{minipage}

\subsection{Hydrogen Valley in dismissed area (ex Sapio)}
Thanks to PNRR funds, in 2022 the Italian Ministry of the Environment and Energetic Security (MASE) published a funding notice to promote the construction of "hydrogen valleys" in dismissed industrial areas \cite{mase_hydrogen}. Porto Marghera adhered to this notice and so the project for the Venetian hydrogen valley began. 

The chosen dismissed plant that is subject of the renovation is the Sapio plant, located in the western part of Porto Marghera (see image ...). The project considers also using part of the already existing equipment of the previous firm: as a matter of fact, the Sapio used to work in the mixing of gases for petrochemical production. Furthermore, together with the Sapio reconversion, a photovoltaic plant is build in the area of Fusina, in order to sustain the hydrogen production using renewable resources.

The project plans the hydrogen production mainly for industrial use, in order to sustain the decarbonization of heavy plants of the area, but hydrogen is also destined to logistical use for the public transport in the mainland. The total nominal capacity  of the electrolytic cell is planned to be 5 MW, which correspond to a hourly capacity around 1.000 Nm$^3$/h of hydrogen. \cite{hydrogentech}

\section{Other proposals}
The following proposals are the result of the latest efforts made by citizens, activists, researchers and professionals to deal with the issues of Porto Marghera. As it is apparent, there is not a wide variety of them, a clear sign of the great difficulty that all of the subjects have to face trying to make changes to the current situation.

\subsection{100 idee per Porto Marghera}
100 idee per Venezia is the condensed effort of a participatory process that involved about 250 citizens of Venice since 2023, published as a book by La Toletta edizioni in 2024. The projects presented concern the entire city of Venice, and some sections are focused on the issues of regeneration and requalification of the area of Porto Marghera. The volume contains ideas formed after workshops, talks and seminars, and it could serve as a starting point for understanding the needs of the population.


\subsection{The Marghera Gamble}
The Marghera Gamble is a project by architects Martina Bertani and Charles André, exploring future scenarios for Porto Marghera. The discussion brought together voices such as: Gabriella Chiellino (environmental entrepreneur), Cristiana Colli (journalist and curator), Massimiliano De Martin (Venice Deputy Mayor for Environment and Urban Planning), Stefano Zeli of Terrapreta (collective specialising in soil regeneration), and Jane da Mosto for WahV, sharing perspectives on sustainable territorial development based on regenerating the lagoon system and associated values for society. The debate took place on the 20th of September 2025 within the Biennale Architettura, in the ENS public programme: Relational landscape: development without growth in Venice.

\subsection{Ecogiustizia subito}
Ecogiustizia Subito is a campaign promoted by  ACLI, AGESCI, ARCI, Azione Cattolica Italiana, Legambiente e Libera. The associations underline the need for funds for infrastructure for safety and environmental remediation for sites such as Nuovo Petrolchimico and Fusina; the need for a higher quantity of scientific and environmental analyses and for stricter controls on the management of times, procedures and right use of the resources for the environmental remediation; the need for the promotion of community participation for the inhabitants to define projects of economic and social redevelopment focusing on the industrial conversion of the SIN and on the creation of new job opportunities with green economy.

\section{Final evaluation}
Porto Marghera presents four main regulatory plans that affect the management of its territory: the Piano Regolatore Portuale per il Porto di Venezia (1965), the Variante al Piano Regolatore Generale per la zona di Porto Marghera (1999), the Masterplan per la bonifica dei siti inquinati (2004) and the Piano di Assetto del Territorio (2014), each, as previously covered, in both their assets and liabilities.

The general evaluation of the contribution of said plans to the territory of Porto Marghera, in both its port area and its residential one, and to the activities carried out on it, those being of commerce, of environmental regeneration or of production, cannot be said to be overwhelmingly positive.

It would be unfair to say that progress hasn’t been made: a new Port Regulatory Plan is currently in the works, to better serve a port area that has most certainly changed since 1965; projects like the Parco Scientifico Tecnologico VeGa, the Parco San Giuliano and the SIMAGE system have been successfully implemented and are an important part of the life of the area; remediation activities have been activated in 90\% of the targeted area; the Venice Heritage Tower was created.

However, for an area that has been the direct focus of such numerous debates, and that has continuously been signalled as an area of crisis from the economic, environmental, industrial points of view, as well as from the quality of life of its inhabitants.

Especially analysing the state of the environmental regeneration that should be carried out, only the 14\% of the industrial district has been actually regenerated, as of 2016. Not only this is a problem as it allows pollution to persist, but it also further prevents any other changes that could take place, as they would need remediation first. This is also aggravated by the fact that, due to the lack of participation, the needs of the population are often neglected, both on the long and short term, and most citizens are often uninformed of the condition of the industrial fabric.

Furthermore, it is important to underline that Porto Marghera was never completely given up on during the last decades, as the projects carried out testify. As much as scattered, one must recognize that innovative solutions are present: the system of recollection of wastewater PIF was completed, the urban-industrial symbiosis network was built and is operative, the green hydrogen valley is underway and the debate on Porto Marghera never extinguished. Nevertheless, this scattering slows the general renovation process and results in a physical oxymoron: polluted and dismissed areas are placed right next to “wannabe” centres of innovation, that most of the times seem misplaced and end up being under-achieving (VeGa \cite{altrochemestre} and the Venice Tower). 

In conclusion it can be argued that the plans do exist, however much outdated, and innovative projects are carried out, but this doesn’t seem to be sufficient for the needs of the territory. 

% --- CAPITOLO 3 ---
\chapter{Planning solutions}
\section{The existing problems of Porto Marghera}
As seen up to now, Porto Marghera suffers from a wide range of problems, creating a complex system that cannot be analyzed in a simplistic way. In this section a summary of the existing problems is made.

\begin{itemize}[leftmargin=*, noitemsep]
\item \textbf{Soil and groundwater pollution at an extreme level}: the construction of Porto Marghera and the activities that operated on it since the XX century have caused pollution of soil, groundwater and sediments of the lagoon, creating a negative one-of-a-kind situation.
\item \textbf{Slow bureaucracy}: the works for pollution decontamination have been stalled for years and very few hectars of land actually have a completely carried out decontamination. The multiplicity of entities that exist on the same land contributed in slowing the process more.
\item \textbf{The outcome of the remediation is unsure}: while being an area full of potential, located in a place that is economically important, the polluted condition of the area does not attract private investors that are discouraged in starting the remediation process due to its complexity and unsureness of success.
\item \textbf{Partially an active zone}: the area is still partially active since it hosts port, logistics companies, chemical companies and oil companies, but this makes the decontamination process more difficult because public entities have to discuss also with private entities.
\end{itemize}

It is clear that Porto Marghera’s case is unique, therefore it is very difficult to find examples in other locations where problems were worked out in a perfect way. However, some examples that have partially similar characteristics to Porto Marghera can be found and some inspiration can be taken from them.

\section{National and international virtuous examples}
\subsection{Emscher Park}
The case of the Emscher Park in Germany is one of the most complex yet successful cases of brownfield renovation, considering that it regards the vast and largely polluted area of the Ruhr district. The strength of the program was that the various objectives established were divided in the span of the ten years of activity of the project (1989-1999), and that it was carried out through single-focused projects that stayed between the previously defined guidelines.

Including main focuses such as the restoration of the hydrographic system, new forms of housing and cultural activities, the conservation of cultural heritage and the reconstruction of the landscape, the Emscher Park case gave a new life to the rust belt, and became and example to follow all across Europe.

\subsubsection{The Planungsgesellschaft IBA Emscher Park GmbH (Internationale Bauausstellung (IBA) or International Architecture Exhibition, German device for urban engineering and architecture)} 
One of the key points of the Emscher Park operation was the definition of an authority with no legal power but with the aim to coordinate the action of all the parties involved, both public and private – including 17 different Municipalities, and supporting partners such as Emschergenossenschaft and Lippeverband, Kommunalverband Ruhrgebiet, Deutsche Bahn, RAG AG and many others \cite{rhur_iuav}.
The society, officially inactive since 1999, was organised in a board of directors, with members such as political, economic, social and environmental authorities, and a steering committee headed by the minister of urban planning and transportation, and made up of the representatives of the region, of the main municipalities, of TU Dortmund, of the professional orders and of the freelancers of the involved professions. Its employees were less than thirty, and its headquarters were located in Gelsenkirchen.
Its main characteristic was the total lack of legal power, that made it impossible for it to fine authorities or to force them to take part into initiatives. Its aim was to be a meeting platform for dialogue and exchange of ideas between the different parties involved, that could each come up with their own projects.

\textbf{Application for Porto Marghera}: One of the main problems that Porto Marghera has to face is the presence on its territory of many different legal actors, such as the Municipality of Venice, the Port Authority of the Northern Adriatic Sea, the Veneto Region and more. This plethora of actors, along with the extended length of the bureaucratic procedure, has proved to be quite disadvantageous to the making of actual plans for the remediation of Porto Marghera. Therefore, the Emscher Park case could be a valid example to follow, with the ad hoc creation of a non-legal authority aimed at the coordination of the various ideas and projects for Porto Marghera.

The presence of said authority would ensure that all of the ideas from the different parties are shared and heard, and at the same time it would maintain a cooperative and non-threatening environment, with the impossibility of giving sanctions or forcing the actors to take part into a specific project. With shared ideas and a more punctual organization of them, the future of Porto Marghera could be made different than its present.

\subsubsection{Guidelines and single-focused projects} 
The project for Emscher Park followed seven main guidelines developed by the IBA, identified as:
\begin{itemize}[leftmargin=*, noitemsep]
\item Reconstruction of landscape – the Emscher Landscape Park
\item Ecological restoration of the river Emscher system
\item Rhein-Herne Canal – an adventure space
\item Industrial cultural heritage as national treasure
\item Working in the park
\item New forms of houses and housing
\item New options for social, cultural and sports activities
\end{itemize}
These general principles were the base that had to be integrated into the single-focused projects that the actors had to come up with, each registered in a short, medium or long term period of time.
This allowed the plans to be followed more strictly, and to have a clearer vision of what the use destination of each part of the park should have been. 

\textbf{Application for Porto Marghera}: Porto Marghera lacks a clear vision of what it wants to be in the future. It is apparent that remediation works are needed, but an overall masterplan can be too large of a project for an area that needs such a conspicuous amount of work.

Main guidelines to follow and single-focused projects could be an efficient solution to this problem, as they would give a more defined and detailed vision of what each part of the area would and could look like. The remediation works wouldn’t just exist in a vacuum but would have a clear objective as to why it is done, apart from the obvious environmental constraint. The short, medium and long term division for the single-focused project would allow the overall plans to proceed much faster and the first results to be seen in a relatively closer window of time, to make room for future adjustments. The definition of the guidelines would be carried out by the aforementioned ad hoc authority, while the single-focused projects would be proposed by single actors.

\subsubsection{Population participation} 
What distinguishes the Emscher Park project, amongst other things, is the fact that the population of the area had such a central part in the processes of creation of what would become the park itself. In fact, the wishes and desires of the inhabitants on what they wanted the area to look like, what they wanted to celebrate and preserve, which functions were needed in their opinion, how the landscape had to transform to fulfill its full potential in their minds. Numerous projects were proposed and the population’s contribution to each varied from one to another.
It is fundamental to understand that without the population’s willingness to participate and without the plan to include it, the results wouldn’t have been as successful.

\textbf{Application for Porto Marghera}: Porto Marghera is a heavily polluted area that because of this issue and of its industrial past, doesn’t leave much space for the population and its needs, sometimes even basic ones. It is noted, for example, that a problem in previous years was the one of the lack of public housing, as well as the lack of jobs. Emscher Park serves as an example for how the ideas and the desires of the population helped shape its final form.

The same could be applied to Porto Marghera, for example letting the population decide, within the list of the abandoned industrial buildings, which ones to restore to keep as a part of the industrial heritage of the area for a process of valorization, and which ones to demolish or restore to have a new function, according to the ones needed the most by the population itself.
The inhabitants could also be proposing their own ideas for projects to realize on the areas and buildings subject to remediation, and such ideas could be explored in dedicated workshops.

\begin{minipage}{0.55\textwidth}
\centering
\includegraphics[width=\textwidth]{images/03_emscher.png}
\captionof{figure}{Emscher park, by Peron I.}
\end{minipage}
\hspace{20pt}
\begin{minipage}{0.39\textwidth}
\centering
\includegraphics[width=\textwidth]{images/03_genova.png}
\captionof{figure}{Porto di Genova, by Peron I.}
\end{minipage}

\subsection{Genova's harbor}
Genova is a perfect example of redevelopment of a dismissed area and reconnection of the city with the sea. Historically, Genova was always characterized by a dualism: the port and the city. The geography of the area was a great constraint: the mountains on the north, the valleys of the sides and the sea at south obliged the city to tight itself inside the available space. Inside this narrow area, the municipality of Genova and the Port Authority coexisted and regulated strictly separate zones. 
\subsubsection{Previous situation}
The area was terribly divided into two parts and the city didn’t have access to the sea. Starting in the 1960s, Genoa's “old port” experienced a gradual decline in activity due to the construction of new ports in the West and the relocation of main activities to the areas of Voltri (bulk cargo) and Multedo (crude mineral oils). With the transformation of maritime trade and the spread of container use, the old port increasingly lost its role in the economic balance of the city, resulting in the abandonment of several buildings. In addition to this, the construction of the Sopraelevata street in the late ‘60 strengthened the sense of separation between the two parts of Genova \cite{storiaportoantico}. 
\subsubsection{What was done}
The difficult situation witnessed a change when Renzo Piano designed the “Porto Antico”: this was the first time for Genova that an area was taken away from the port jurisdiction and given back to the city to be destined to urban settlements. In 1995 the Porto Antico S.p.A was created ad hoc to manage the whole area. In its project, Renzo Piano also planned the redevelopment of dismissed buildings present there: the idea was reconverting them into areas that were missing inside the historical city centre (such as areas for cultural events or commercial areas). It is estimated that Porto Antico created 1000 jobs and is one of the most visited areas of Genova, especially because of its famous aquarium \cite{mic_portoantico}. 
\subsubsection{Current situation}
After more than 20 years, Porto Antico still attracts visitors from different parts of Italy. A great number of cultural events are organized yearly in the area, giving also to citizens job and leisure opportunities. 

\subsection{Rotterdam's harbor}
Rotterdam is a strong example of how a large industrial port can be transformed while remaining economically active. Historically, the port developed mainly for industrial and logistical efficiency, with little attention to its relationship with the city or the environment. Over time, port activities moved westward toward the North Sea, creating a physical and functional separation between Rotterdam and its harbor.
\subsubsection{Previous situation}
During the 20th century, Rotterdam’s port was dominated by heavy industries such as oil refining, petrochemicals, and bulk transport. The port expanded over a very large area, reaching more than 40 km from the city centre \cite{rotterdam}. While this growth was economically successful, it caused serious environmental damage and left several inner-port areas abandoned. The port also increasingly conflicted with protected natural areas, later included in the Natura 2000 network.
\subsubsection{What was done}
Rotterdam chose an integrated approach combining port development, environmental protection, and urban renewal. The construction of Maasvlakte 2 (an expansion of the port), considered a project of national importance, was accompanied by strong ecological compensation measures. Coastal dunes were recreated, marine habitats were restored, fishing was limited in sensitive areas, and compensation zones were made much larger than the areas that were damaged \cite{rotterdam2}.

In parallel, the city redeveloped former port areas close to the centre, such as Kop van Zuid and Stadshavens. Old docks and industrial zones were converted into mixed-use areas with housing, offices, cultural spaces, and innovation hubs, reconnecting the city with the waterfront \cite{rotterdam3}.
\subsubsection{Current situation}
Today, Rotterdam remains one of Europe’s most important ports while also being a leader in sustainability and innovation. Requalified waterfront areas attract residents, visitors, and new economic activities. Environmental restoration and monitoring are still ongoing, showing a long-term commitment to ecological balance.
\subsubsection{Similarities with Porto Marghera}
Like Rotterdam, Porto Marghera developed as a large industrial and petrochemical area, creating strong separation between the port and the city and causing environmental degradation. Both areas include abandoned industrial zones and sensitive ecosystems. The Rotterdam experience indicates that ecological compensation in lagoon environments can be technically achievable in areas surrounding port infrastructure, without necessarily interrupting port activities, offering relevant insights for the redevelopment of Porto Marghera.

\begin{figure}[H]
\centering
\includegraphics[width=\linewidth]{images/03_rotterdam.png}
\caption{Map of the evolution of Rotterdam in history \cite{rotterdam}}
\end{figure}

\section{Project proposals}
For a territory characterized by such complexity, a general masterplan isn’t the right choice. Instead, the redevelopment of a number of areas inside Porto Marghera could be the starting point for a new use and look of it, following the examples of the projects that were carried out in the last decade (paragraph 2.5).
In addition, considering the high level of pollution both for soil and for the aquifer, the outcome of a possible remediation is unknown at the start. For this reason, the land use creation process cannot be disassociated to the remediation process. In other words, the land use must be decided according to the outcome of the remediation process, that is also the achievable level of remaining pollution. In the following paragraphs, the project proposals of this work are presented. A first part delves into the remediation techniques and explains the vision of the landuse evolution in time according to remediation (taking inspiration from the proposal of PhD Irene Peron \cite{peron}). Instead, a second part shows the project proposals about landuse and building reconversion for Porto Marghera. 

\subsection{About remediation}

\subsubsection{The techniques}
There is a wide range of techniques that can be taken to assess the pollution problem. In the presented scheme, the different solutions are linked with its effectiveness and consequent use limitations.

\underline{MISE (messa in sicurezza emergenziale)} [Emergency safety measures]: 
Any immediate or short-term intervention, to be implemented in the event of sudden contamination of any kind, aimed at containing the spread of primary sources of contamination, preventing their contact with other matrices present on-site, and removing them, pending any further remediation interventions or operational or permanent safety measures.

\underline{MISO (messa in sicurezza operativa)} [Operational Safety Measures]
The set of interventions carried out on an active site intended to guarantee an adequate level of safety for people and the environment, pending further permanent safety measures or remediation to be implemented upon the cessation of activity. These also include contamination containment interventions to be implemented on a transitional basis until the execution of remediation or permanent safety measures, in order to prevent the spread of contamination within the same matrix or between different matrices. In such cases, suitable monitoring and control plans must be prepared to verify the effectiveness of the solutions adopted.

\underline{MISP (messa in sicurezza permanente)} [Permanent Safety Measures]:
The set of interventions aimed at permanently isolating pollution sources from the surrounding environmental matrices and guaranteeing a high and definitive level of safety for people and the environment. In such cases, monitoring and control plans must be provided, as well as land-use restrictions in accordance with urban planning regulations

\underline{Remediation}: The set of interventions aimed at eliminating pollution sources and polluting substances, or reducing their concentrations present in the soil, subsoil, and groundwater to a level equal to or lower than the Risk-based Threshold Concentrations (CSR). Remediation techniques are divided into two main categories: \textit{in-situ} techniques (pollution is treated in its location) and \textit{ex-situ} techniques (the polluted matrix is removed and treated separately). 

In accordance with ARPA Veneto chosen remediation techniques, this work wants to highlight the need of in-situ remediation. In fact, ex situ remediation proves to be more costly than in-situ actions: there is a need to find a site (landfill) where the waste can be treated, resulting in a greater environmental and health impact stemming from the risks of transport and disposal. Landfilling may be an option for small areas characterized by a specific and localized source of pollution, but it proves impracticable in situations as vast and complex as that of Porto Marghera. 

\begin{figure}[H]
\centering
\includegraphics[width=\linewidth]{images/03_remediation.png}
\caption{Visualization of the different techniques to assess the protection of the environment. Image extracted from Peron I. (2016). \cite{peron}}
\end{figure}

\subsubsection{The vision}
This work does not aim at projecting new ways to carry out the remediation process, but it tries to propose an alternative vision on the \textit{iter} of decontamination of the soil. Since the Masterplan on remediation, the slowness of the procedures deriving from the complexity of the matter have left a large quantity of terrains completely unusable by the citizens or new companies: obviously, this was due to the contaminated condition of the land, but there might be a different way to give public access to the areas, even if limited, and this is what the vision here presented consists of. 

The basic concept here proposed is that the intended land use of a contaminated soil needs to be temporary and legally capable of change according to the progress of the remediation. This approach stems from the fact that when a remediation process starts, the outcome (i.e. the final condition after the operation) is unknown: this implies that a terrain might not have the qualities required for the intended land use, even though the planned remediation was completely carried out. Instead, if a piece of land doesn't have a fixed landuse but a changing-with-time landuse, its final destination can be assessed later accordingly. 

Thus, the project vision is tightly interwoven with the progress and monitoring of the site's remediation process; a time-schedule example with a monitoring window of 5 years articulated across four temporal scenarios is here presented. In the example the chosen remediation technique is phytoremediation. It is given for granted that the necessary MISE works have already been implemented (see paragraph 1.6.2 about soil containment).
\begin{enumerate}[leftmargin=*, noitemsep]
\item Initial scenario(t=$t_0$): at this point the area faces its maximum level of contamination, the use of the site is highly restricted and the remediation process is at its starting point. The whole area is decided to be treated with phytoremediation. The existing dismissed buildings are still in abandoned condition.
\item First scenario (t=5 years): another sampling is carried out and the level of contamination has decreased, thus the access to the area is consented but limited (e.g. the area is equipped to become a park but with lifted walkable paths - even if placed outside the urban area it can function as a recreational area for workers in their breaks).
\item Second scenario (t=10 years): a further sampling is carried out and the level of contamination has decreased, up to a level at which it is permitted to give the area to logistic use (e.g. the dismissed building can start to be restored to the necessary use)
\item Third scenario (t=15 years): a further sampling is carried out and the level has decreased enough to let the area be used for commercial use (e.g. the dismissed building can be restored to welcome offices)
\end{enumerate}

It is important to state clearly that the residential use of the remediated land is not considered in the majority of situations, due to the fact that the pollution level is so high that any kind of remediation could never completely restore the complete salubrity of soil and groundwater \cite{grilliVenezia}. In addition to this, the land located in the artificial macro-islands is not adequate to residential use because of infrastructure and location (it is too far from the urban facilities and the connection to the mainland happens only through a small number of bridges).

\subsection{About landuse}
The idea for a project in Porto Marghera, a hub full of potential but at the same time extremely difficult to coordinate in terms of actors and environmental interventions, would be focused on green hydrogen, environmental regeneration and training of workers and students.
The main objectives of the project would be to build and further expand a centre for the distribution and use of green hydrogen for transport, industry and port, and to make the port operations more sustainable with the aid of electric infrastructure and to add facilities. The brownfields would be used for the cultural and historical valorization of the area, and according to the needs of the inhabitants of Porto Marghera. The whole project should be enriched by numerous workshops and training activities for students and workers, to make them more aware of the processes of environmental regeneration, with a hands-on approach.

This idea is mainly applied to the closest areas to the urban fabric that are currently in state of abandonment. This choice is justified by the fact that there is an unmotivated separation between industrial area and city are along Via Fratelli Bandiera, therefore using the abandoned spaces on the eastern side of the area would constitute a first reconnection between the two parts, following the virtuous example of Genova's harbor. An outline of the most interesting dismissed areas to consider is presented in figure \ref{fig:dismessi}. A more detailed presentation of this is given in the first part of this section.

In the second part, the project identifies the areas on the macro-islands that are currently empty or abandoned in order to choose them as the place where to apply the remediation vision presented in the previous section.

In the following paragraphs, a list of ideas is presented together with a map of the possible location of said project-idea. The plan involves only buildings shown in figure \ref{fig:dismessi}, instead for the use of the brownfields an additional set of abandoned areas in the macro-isles is considered.

\begin{figure}[h]
\centering
\includegraphics[width=\linewidth]{images/03_dismessi.jpeg}
\caption{Map of dismissed industrial buildings (highlighted in orange) near the city center of Marghera. The map was created by the authors using QGIS.}
\label{fig:dismessi}
\end{figure}


\subsubsection{The green hydrogen hub}
First of all, the project would be compatible with the already existing Hydrogen Valley Venezia in Porto Marghera, under the funding of the PNRR (see paragraph 2.5.4).
The idea would be to continue with the building of green hydrogen plants based on photovoltaic panels and other renewable energy sources considered most suitable for the space. This would allow for more production, which would make the green hydrogen available not only for transport and industry, which are the main objectives of the Hydrogen Valley Venezia, but also for heavy goods vehicles and naval vehicles.
This would be a perfect opportunity for workers and Marghera inhabitants to learn more about green hydrogen in workshops, and for university students to write dissertations and to fulfill internship duties. 

A possibile placement of the hydrogen hubs is presented in figure \ref{fig:hydrohubs}, considering dismissed buildings that could be restored and are the closest to the green hydrogen plant Sapio, in the southern part of Marghera.

\begin{figure}[h]
\centering
\includegraphics[width=\linewidth]{images/03_hydrohubs.jpeg}
\caption{Map of a possible location of the hydrogen hubs considering two of the already highlighted dismissed buildings near the city center. The two buildings were chosen because of their vicinity to the Sapio green hydrogen plant (shown in the map). The map was created by the authors using QGIS.}
\label{fig:hydrohubs}
\end{figure}

\subsubsection{The docks and the monitoring}
The subsequent step would be linked to the ongoing project that aims at the electrification of the port dock driven by the Port Authority, in order to reduce the emissions of the ships that are anchored in the port.
To this, there could be the addition of systems of noise and air quality, that would be useful for the assessment of the results of the operations that have been carried out. The monitoring could be, again, done with the collaboration of students that could rejoice in doing internships. Furthermore, an information point could be considered, to spread the awareness among inhabitants of the neighborhood.

\subsubsection{The brownfields on the mainland}
Regarding the brownfields, there is a distinction to be defined between the areas located in the mainland and the areas located on the macro-islands. It goes without saying that a different purpose in land use has to be considered for the two zones given their different vicinity to the urban fabric, the infrastructure facilities and the dissimilar risk of pollution (both atmospheric and acoustic). Thus, this project considers that areas inside the macro-islands can only be destined to become green areas, commercial areas or in some cases cultural spaces, instead, for the brownfields on the mainland other destination of use can be considered.

\begin{figure}[H]
\centering
\includegraphics[width=\linewidth]{images/03_projectlanduse.jpeg}
\caption{Map of a possible destination of use of the remaining abandoned areas. The map was created by the authors using QGIS.}
\label{fig:newlanduse}
\end{figure}

Porto Marghera is, as already established, an area rich in cultural value for the impact its industrial nature has had since it was built, but this aspect lacks valorization. This is why the first proposal of use in the mainland would be for a Marghera Museum, a restored industrial building that would serve as a time capsule for how production worked through the years and what legacy it left. The museum would be articulated in different parts, each one related to the different type of industry that is being represented: there would be a part of historical retelling and historical documentation through pictures and artifacts, and a part where some of the most important mechanisms for production are built again in order for people to further inspect them and use them. Each section would also be filled with the words of the people that worked in the different industries. A part would then be dedicated to the historical accounts of the worker protests during the years, and the final section would be dedicated to Marghera as a part of the Venetian conglomerate but through the eyes of the people that live in it, with accounts from the past and from today. This last section would also contain a part dedicated to the ongoing projects in Porto Marghera, in order to keep the inhabitants accountable of its current development. The museum would be useful, again, for students of Ca'Foscari and IUAV Universities that could offer guided tours as part of their internships. In addition, the museum could be the perfect space where to hold different workshops to learn more about environmental regeneration and sustainability.

The considered building, located along Via Fratelli Bandiera (Figure \ref{fig:newlanduse}), is a place that has been in state of abandonment since 2000s and photos of Google Earth testify that it is a place that the citizens asked for its renovation (Figure \ref{fig:museum}).
\begin{figure}[H]
\centering
\includegraphics[width=\linewidth]{images/03_museum.jpg}
\caption{Photos of the building that could be renovated to become museum. Source: Google Earth Pro.}
\label{fig:museum}
\end{figure}
 
Another proposal of use of one of the dismissed buildings of the area is a new university hub, either for Univeristà Ca'Foscari, IUAV or H-FARM business school. The dismissed building located close the railway in the northen part of Marghera presents a great number of infrastructural advantages: vicinity to train station Venezia Mestre, close to the city center and near an empty green area that could become the park of the university hub itself. The architecture of the building is very basic, but suitable to host university classrooms.
\begin{figure}[H]
\centering
\includegraphics[width=\linewidth]{images/03_university.jpeg}
\caption{Photos of the building that could be renovated to become a new university hub with its own park. Source: Google Earth Pro.}
\label{fig:uni}
\end{figure}

When it comes to the housing, the aim would be to solve a problem that characterizes Porto Marghera and that has been lamented multiple times by its inhabitants \cite{ripensarevenezia}: the lack of public housing and student housing. 

Public housing: Marghera and Mestre suffer from the shortage of public housing (ERP — Edilizia Residenziale Pubblica) since many years \cite{ripensarevenezia}. The existent ERP are old and already full and none of the recent governments have tried to solve this problem. Therefore, it seems necessary to include in this project the renovation of dismissed buildings that could function as public housing. The considered building for this is the one presented in figure \ref{fig:housing}.

Shared student-elderly housing: In recent years, projects that encourage distant generations to share a home have surfaced all over the world, including on university campuses in Canada, California and the Netherlands, as the BBC article reports \cite{itergen-house}. A similar idea could be applied also in Porto Marghera, considering its closeness to the Universities in Venice. The shared housing between students and elderly is thought as an experience that would benefit both categories and would guarantee them a better living situation. Students would benefit from it as they would have rooms to rent at a reduced price, sensible to their needs since the vast majority of them doesn’t work or works lesser paid jobs, and in a location that allows them to get to Venice rather swiftly with public transport. The elderly, on the other hand, frequently lack company and assistance. In this project, each person would have their own bedroom and bathroom facilities, and the other spaces would be shared, in order to guarantee companionships and little helps from the students to the elderly and vice versa – though not in terms of medical assistance. The common areas could also be filled with monthly organized cultural activities. The spaces would have their set of rules, such as ones for silence, privacy and volunteering. The space would also have to have, in close proximity, facilities for social and medical aid, which could ultimately create new spots in the job market.

The chosen building for this idea is the same one used for the public housing, presented in figure \ref{fig:housing}, located very close to the train station of Venezia Mestre and near the other planned hub for university. The building structure is suitable for the shared housing because it has internal gardens, small internal spaces and overall a large area. 
\begin{figure}[H]
\centering
\includegraphics[width=\linewidth]{images/03_housing.png}
\caption{Photos of the building that could be renovated to become a shared student-elderly housing. Source: Google Earth Pro.}
\label{fig:housing}
\end{figure}

Finally, a proposal of landuse for one of the considered buildings is to build a kindergarten. As a matter of fact, the \textit{Livelli Essenziali di Prestazione} (Essential Performance Levels) of Porto Marghera regarding the offer of places of education before elementary school are very low compared with other cities in Italy (39 places for 100 newborns in 2020) \cite{lep}. Thus, the wide green area located along Via Fratelli Bandiera could be the perfect spot for the construction of a new small kindergarten with a big green area all around it. 
\begin{figure}[H]
\centering
\includegraphics[width=\linewidth]{images/03_asilo.jpeg}
\caption{Location of the empty abandoned area that could be used to build a small kindergarten with a big garden.}
\label{fig:asilo}
\end{figure}

The last part of the proposal for the brownfileds in the mainland is dedicated to the conversion of buildings into spaces for co-working and startup hubs, the aim of which would be to give an actual space for sharing ideas and innovation in a space like Porto Marghera that so desperately needs it. This could be done starting from the VeGa, expanding its area. 

\begin{figure}[H]
\centering
\includegraphics[width=\linewidth]{images/03_macroisole.jpeg}
\caption{Map of the areas inside the macro islands considered in the project for a new destination. Map created by the authors using QGIS}
\label{fig:macroisole}
\end{figure}

\subsubsection{The brownfields on the macroislands}
Considering now the areas in the macro islands, the main project proposal is to identify the empty areas and the dismissed areas (land that hosts abandoned buildings) as in figure \ref{fig:macroisole}. The former are to be treated with phytoremediation, thus blocking its use for at least 5 years, as in the example presented in paragraph 3.3.1. Meanwhile, the latter are to be remediated as well, but will include in the future the renovation of the currently abandoned buildings. The use destination has to be defined later, in accordance with the vision of the project, choosing among logistics, cultural or commercial use.

\subsubsection{The funding}
When it comes to the funding of the project, many routes could be chosen.
As already established, the PNRR gives funds for dock electrification and renewable energy production such as green hydrogen (already implemented through the MASE funding notice). Furthermore, the Veneto Region has calls for bids for applications of environmental regeneration and requalification of polluted sites. Green investors and PMI can benefit from funding from partnered banks as well.
European programs could also be a part of the funding, such as the Interreg for urban and industrial regeneration, and the Horizon Europe for research on renewable energy and hydrogen as a source of it.
Last but not least, private investors would also be included, especially involving them in startups and sustainable logistics hubs.

\subsubsection{The coordination of stakeholders}
In order to achieve this, taking into account the large number of actors present on the territory and cited in the previous sections, the suggestion would be the one of creating an authority that would coordinate the process.
The Municipality of Venice, the Veneto Region and the Port Authority of the Norther Adriatic Sea would be the institutional stakeholders, while the projects and training would be the responsibility of universities and research centres.
Private investors and PMI would act on actuation and investments.

\begin{figure}[H]
\centering
\includegraphics[width=\linewidth]{images/03_peron.png}
\caption{One of the dismissed buildings in the northen part of Porto Marghera, a possible place of renovation while keeping memory of the history of the area. Extracted from Peron I.(2016), Potenzialità contese. Porto Marghera: una questione di metodo. \cite{peron}}
\end{figure}


\chapter{Conclusion and annexes}
This work analyzed Porto Marghera under various aspects in the attempt of giving to the reader the clearest picture as possible. In the fist part of this work, the presentation of the area with its characteristics and issues was given, which was followed, in the second part, by the analysis of the legislative framework in which Porto Marghera finds itself. This step clearly showed that an extensive legislative work was done on the area, especially due to the increasing awareness of the environmental disaster that the whole place testified, and also that the area was never completely given up on, considering the active presence of projects for the area, even though not unitarily coordinated. Perhaps the lack of a general plan and regulation for the area was the beginning of the problem itself. However, given the present patch-worked condition of Porto Marghera, this work affirmed in the third part that it is now utopian to think about an all-inclusive masterplan on Porto Marghera. Thus, some separate project proposals are presented in the last part of this work, trying to keep a realistic sight on the condition of the area. 

All in all, this work led to the conclusion that the current  state of Porto Marghera is the result of a series of actions of humans guided often by their own self interest and the unconsciousness of the consequences of these. In addition to this, the lack of a common path to assess the issue prolonged the critical situation for years. Planning in such conditions becomes a matter to be dealt with carefully, by people that consider the common social, environmental and physical wellness before the private interest. Citizens have to be involved in the planning process since they are the first to feel the effect of it and the most important power they already have on this is the elective one: choosing consciously their own representatives inside the municipality is the most fundamental right and duty that a citizen can exercise.

\begin{figure}[H]
\centering
\includegraphics[width=0.8\linewidth]{images/03_pollution.png}
\caption{Photo of the dispersion of pollutants inside the lagoon. Image extracted from \cite{mag_acque}.}
\label{fig:pollution2}
\end{figure}

\begin{figure}[H]
\centering
\includegraphics[width=0.8\linewidth]{images/02_trasformabilitapat.png}
\caption{Graphical representation of the convertibility plan of the PAT, re-designed by Peron I. \cite{peron}. This map was used to project the proposals of chapter 3.}
\label{fig:convertibility}
\end{figure}

% --- Bibliography---
\nocite{*}
\bibliographystyle{plain}
\bibliography{bibliography}

\end{document} 